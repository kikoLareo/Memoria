\chapter{Introdución}
\label{chap:introducion}

\section{Contexto y motivación}
\label{sec:contexto}
¿Netflix?¿HBO?¿DAZN?¿Amazon Prime Video?¿Disney+? Me atrevería a afirmar que
(casi)todos los lectores de este documento tienen o han tenido acceso a alguna de 
estas aplicaciones en los últimos meses. Diría también que son unas de las principales
aplicaciones en las que uno piensa a la hora de descargar alguna aplicacion en alguno de sus dispositivos.
En muchos de estos dispositivos hasta vienen preinstaladas. Según un estudio
realizado por la Comisión Nacional de los Mercados y la Competencia (CNMC) el 58\% de los hogares 
españoles con acceso a Internet usaba algún servicio de vídeo en streaming a mediados del año 2023, 
frente al 37\% de mediados de 2019. \cite{CNMC}. Otras fuentes situan este porcentaje en un 81\% a principios
de 2023 \cite{Streaming2023} e incluso en un 95\% en junio de 2024 \cite{Streaming2024}. Estas cifras varian un poco dependiendo
de la fuente, pero todas coinciden en algo: el consumo de contenido audiovisual a través de plataformas de streaming
está en auge y ya supera a las plataformas televisivas tradicionales. Leyendo estos estudios y viendo la evolución de
este mercados se entiende el por qué de la constante aparición de nuevas plataformas de este estilo. Desde peliculas y series
hasta deportes, pasando por documentales, cursos, conciertos, etc. Estas plataformas se adaptan a cualquier estilo
de contenido y a cualquier tipo de usuario. 

No es únicamente el número de usuarios el que ha aumentado, sino también el tiempo que pasamos viendo contenido 
en estas plataformas, el número de plataformas distintas a las que accedemos y el número de dispositivos en los
que lo hacemos. La causa de esto es la facilidad de acceso a estas plataformas desde cualquier lugar y dispositivo. 
Y es que, ¿quién no ha empezado a ver una serie en la televisión del salón, ha continuado en la tablet de la cocina y 
ha terminado en el móvil de la cama? ¿O quién no ha parado la serie para cambiar de plataforma y ver un partido 
de fútbol? El consumo de contenido audiovisual nunca había sido tan sencillo y accesible. 

Sencillo para el consumidor claro, pero para llegar a esta versatilidad y facilidad de uso hay un gran trabajo detrás 
para lograr que utilizar estas plataformas sea igual sin importar el tamaño, la marca o el sistema operativo del dispositivo
ni el lugar en el que se encuentre el usuario. 


\section{Objetivo del trabajo}
\label{sec:objetivo}
\subsection{Desarrollo de una aplicación OTT multiplataforma y multicliente}
\label{sec:PlataformaOTT_introduccion}

Aquí surge el objetivo de este trabajo: desarrollar una aplicación multiplataforma para la visualización de contenido
audiovisual en streaming.

En este trabajo se va a abordar todas las fases de desarrollo de una aplicación OTT (Over The Top), con dos características
principales: multiplataforma y multicliente. Multiplataforma porque la aplicación esta orientada a poder adaptarse a cualquier 
dispositivo y sistema operativo. Multicliente porque el código no está pensado únicamente para un cliente en concreto,
sino que está pensado para poder ser utilizado por cualquier cliente que quiera tener su propia plataforma donde mostrar
su contenido.

Estos enfoques suponen varios retos a nivel de desarrollo, los principales: adaptación a las distintas necesidades de cada dispositivo 
donde se quiera dar soporte a la aplicación, con las características y limitaciones que conlleva utilizar las distintas plataformas, tecnologías
y sistemas operativos; y la adaptación a las distintas necesidades de cada cliente, tratando de mantener un equilibrios entre la personalización
de la aplicación y la reutilización del código. 


\subsection{Desarrollo e integración de una aplicación para la recogida y visualización de datos de uso}
\label{sec:Analitica_introduccion}

Cuando se desarrollan aplicaciones comerciales siempre hay un objetivo claro: atraer a los usuarios y mantenerlos el mayor tiempo posible
en la aplicación. Para ello, es necesario conocer a los usuarios, saber qué les gusta, qué no les gusta, qué les interesa, etc. Por ello,
las empresas cada vez más están invirtiendo en herramientas de analítica para poder recoger y analizar los datos de uso de sus aplicaciones.

Complementariamente a la aplicación OTT, en este trabajo se va a abordar el desarrollo de una aplicación de analítica de datos de uso de la
aplicación OTT. Esta aplicación recogerá datos que nos permitirá conocer el comportamiento de los usuarios en la aplicación, para poder
mejorar la experiencia de usuario y adaptar la aplicación a las necesidades de los distintos usuarios de las distintas plataformas.

Este desarrollo incluye el análisis y seleccion de las métricas necesarias para conocer el comportamiento de los usuarios en este tipo de 
plataformas, la integración de la API de analítica en la OTT y el desarrollo de una aplicación de visualización de los datos recogidos.
