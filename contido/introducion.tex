\chapter{Introdución}
\label{chap:introducion}

\section{Contexto y motivación}
\label{sec:contexto}


¿Netflix? ¿HBO? ¿DAZN? ¿Amazon Prime Video? ¿Disney+? Es muy probable que casi todos los lectores de este documento 
hayan utilizado alguna de estas aplicaciones en los últimos meses. De hecho, suelen ser de las primeras opciones que 
consideramos al descargar aplicaciones en nuestros dispositivos, y en muchos casos, incluso vienen preinstaladas. 
Según un estudio realizado por la Comisión Nacional de los Mercados y la Competencia (CNMC), el 58\% de los hogares 
españoles con acceso a Internet usaba algún servicio de vídeo en streaming a mediados del año 2023, frente al 37\% 
de mediados de 2019 \cite{CNMC}. Otras fuentes sitúan este porcentaje en un 81\% a principios de 2023 \cite{Streaming2023}
 e incluso en un 95\% en junio de 2024 \cite{Streaming2024}. Estas cifras varían un poco dependiendo de la fuente, 
 pero todas coinciden en algo: el consumo de contenido audiovisual a través de plataformas de streaming está en 
 auge y ya supera a las plataformas televisivas tradicionales. 

Leyendo estos estudios y observando la evolución de este mercado, se entiende el por qué de la constante aparición de 
nuevas plataformas de este estilo. Desde películas y series hasta deportes, pasando por documentales, cursos, conciertos, 
etc., estas plataformas se adaptan a cualquier estilo de contenido y a cualquier tipo de usuario.

No es únicamente el número de usuarios el que ha aumentado, sino también el tiempo que pasamos viendo contenido en 
estas plataformas, el número de plataformas distintas a las que accedemos, y el número de dispositivos en los que 
lo hacemos. La causa de esto es la facilidad de acceso a estas plataformas desde cualquier lugar y dispositivo. Y 
es que, ¿quién no ha empezado a ver una serie en la televisión del salón, ha continuado en la tablet de la cocina
 y ha terminado en el móvil de la cama? ¿O quién no ha parado la serie para cambiar de plataforma y ver un partido 
 de fútbol? El consumo de contenido audiovisual nunca había sido tan sencillo y accesible.

Sencillo para el consumidor, claro, pero para lograr esta versatilidad y facilidad de uso, hay un gran trabajo detrás 
para asegurar que estas plataformas sean funcionales sin importar el tamaño, la marca o el sistema operativo del 
dispositivo, ni el lugar en el que se encuentre el usuario. 

Aquí es donde surge la necesidad de desarrollar soluciones tecnológicas que permitan a las empresas ofrecer experiencias 
de usuario consistentes y personalizadas en múltiples dispositivos y sistemas operativos. El mercado creciente de plataformas 
OTT ha creado una demanda significativa para soluciones que no solo se adapten a esta diversidad tecnológica, sino que 
también permitan a diferentes clientes configurar sus propias plataformas con facilidad.



\section{Objetivo del trabajo}
\label{sec:objetivo}
\subsection{Desarrollo de una aplicación OTT multiplataforma y multicliente}
\label{sec:PlataformaOTT_introduccion}

En este contexto, el objetivo de este trabajo es desarrollar una aplicación multiplataforma para la visualización de contenido 
audiovisual en streaming, diseñada para adaptarse a una amplia variedad de dispositivos y sistemas operativos, así como para 
ser utilizada por múltiples clientes. 

El proyecto abordará todas las fases de desarrollo de una aplicación OTT (Over The Top), con dos características principales: 
multiplataforma y multicliente. Multiplataforma porque la aplicación está orientada a poder adaptarse a cualquier dispositivo 
y sistema operativo. Multicliente porque el código no está pensado únicamente para un cliente en concreto, sino para ser 
utilizado por cualquier cliente que quiera tener su propia plataforma donde mostrar su contenido.

Estos enfoques suponen varios retos a nivel de desarrollo, los principales siendo la adaptación a las distintas necesidades de 
cada dispositivo donde se quiera dar soporte a la aplicación, con las características y limitaciones que conlleva utilizar las 
distintas plataformas, tecnologías y sistemas operativos; y la adaptación a las distintas necesidades de cada cliente, tratando 
de mantener un equilibrio entre la personalización de la aplicación y la reutilización del código.

\subsection{Desarrollo e integración de una aplicación para la recogida y visualización de datos de uso}
\label{sec:Analitica_introduccion}

Además, se desarrollará una aplicación complementaria destinada a la recogida y visualización de datos de uso. En el entorno competitivo
actual, conocer a los usuarios y entender su comportamiento es crucial para mejorar la experiencia de usuario y mantener a los usuarios 
activos en la plataforma. Por ello, esta aplicación permitirá a las empresas analizar en detalle cómo interactúan los usuarios con la 
plataforma OTT, brindando información valiosa para la toma de decisiones.

El desarrollo incluye el análisis y selección de las métricas necesarias para conocer el comportamiento de los usuarios en este tipo de 
plataformas, la integración de la API de analítica en la OTT y el desarrollo de una interfaz intuitiva para la visualización de los datos recogidos.
