\section{Necesidades del cliente}
\label{sec:analisis_necesidades_cliente}

Como se ha comentado en la sección \ref{sec:analisis_requisitos_funcionales}, el analisis de requisitos
en esta aplicación es un tanto especial, ya que no existe un solo cliente sino que existen varios clientes
cada uno con sus necesidades y requisitos. El objetivo del proyecto es satisfacer y cumplir con esos 
requisitos y necesidades a partir de una única aplicación. 

Esto supone que además del estudio de viabilidad técnica, para cada requisito se ha estudiado la manera
de implementarlo de forma que, o bien sirva para todos los clientes, o en caso de ser un requisito 
muy especifico y necesario para un cliente, hacerlo de manera que no afecte al resto de clientes.

Un ejemplo de esto es la interfaz. La interfaz ya cuenta con las funcionalidades necesarias para mostrar
los colores, logos e imagenes de cada cliente. En cuanto a la presentación de los contenidos general, esta es muy similar
en cualquier OTT del mercado: listas horizontales con imagenes y textos de los distintos contenidos. Esto 
no supone ningún problema ya que son los clientes los que gestionan los textos y las imagenes y los responsables
de que estos esten en armonia para una interfaz atractiva. El reto está en elementos del menú dedicados a mostrar
de manera especial los contenidos. Un ejemplo de esto, las listas de destacados. Estas listas son aquellas que se sitúan,
por lo general, en la parte superior de las pantallas y muestran los contenidos más importantes. El problema está en que
cada cliente tiene una idea de como quiere que se muestren estos contenidos. Las opciones de configuración si se quiere
dar libertad al cliente son muchas. El cliente puede decidir si quiere botón de play, si quiere que se muestre el título,
el subtitulo, la duración, la fecha de publicación, etc. Para poder dar soporte a todas estas opciones lo que se ha hecho
es crear un sistema de "etiquetado" o "clasificación" tanto de los contenidos como de las listas en los que se encuentran
estos contenidos. De esta manera, en función del tipo de lista (también conocido como widget) y del tipo de contenido,
estos se mostrarán de una manera u otra. Esto tiene una serie de ventajas y de inconvenientes. Si hablamos de ventajas,
la principal es que cada cliente puede personalizar la interfaz a su gusto, diferenciandose del resto de clientes.
Si hablamos de inconvenientes, el principal es que para cada tipo de lista y de contenido se debe dar soporte en la aplicación.
 Esto supone un esfuerzo extra en el desarrollo y en la gestión de los contenidos. 

 Este ha sido el enfoque que se ha dado a la aplicación desde un principio debido a la arquitectura de los distintos
 componentes creados y del funcionamiento de otras plataformasque pertenecen a la empresa. El enfoque en el que se está 
 trabajando ahora mismo es en una mayor personalización de widgets y contenidos, sin necesidad de tener que crear un nuevo
 tipo de estos elementos cada vez que se requiera una personalización o presentación nueva. Lo mismo para las distintas
 pantallas. 