\section{Necesidades del cliente}
\label{sec:analisis_necesidades_cliente}

Dado que esta aplicación está destinada a múltiples clientes, cada uno con sus propias necesidades y requisitos, 
el proyecto se ha centrado en desarrollar una solución única capaz de satisfacer estas diversas demandas.

Para cada requisito, se ha realizado un análisis detallado para determinar la mejor forma de implementarlo. 
El objetivo ha sido garantizar que, siempre que sea posible, la solución desarrollada pueda ser utilizada por 
todos los clientes de la plataforma. Esto ha implicado diseñar funcionalidades que sean lo suficientemente 
flexibles para adaptarse a diversas necesidades sin requerir modificaciones adicionales.

En los casos donde un requisito es particularmente específico o esencial para un cliente, se ha buscado una 
implementación que no interfiera con las funcionalidades del resto de los clientes. Esto se ha logrado mediante 
la creación de configuraciones modulares y opciones personalizables, que permiten a cada cliente ajustar la 
aplicación según sus propias necesidades sin afectar la experiencia de otros usuarios de la plataforma.

Este enfoque ha permitido mantener una base de código única y consistente, evitando la necesidad de desarrollar 
versiones separadas para cada cliente. Al mismo tiempo, se ha asegurado que cada cliente pueda personalizar su 
experiencia sin comprometer la funcionalidad general de la plataforma.

Un ejemplo es la interfaz de usuario, que permite a los clientes personalizar colores, logos e imágenes. Aunque 
la presentación general de los contenidos sigue un diseño común en el mercado, como listas horizontales de contenidos 
con imágenes y texto, la personalización se ha centrado en elementos clave como las listas de destacados. Estas 
listas, que muestran los contenidos más importantes, pueden ser configuradas por cada cliente para incluir o 
excluir elementos como botones de reproducción, títulos, subtítulos, duraciones, y fechas de publicación.

Para soportar esta personalización, se ha desarrollado un sistema de "etiquetado" que clasifica tanto los 
contenidos como las listas, permitiendo una visualización flexible según las preferencias del cliente. Este
 enfoque permite a cada cliente diferenciar su interfaz, aunque implica un esfuerzo adicional en desarrollo y
  gestión de contenidos.

Actualmente, el proyecto se está orientando hacia una mayor personalización sin necesidad de crear nuevos tipos 
de elementos cada vez que se requiere una nueva configuración o presentación, mejorando así la flexibilidad de 
la plataforma.
