\section{Requisitos}
\label{sec:analisis_requisitos}

En software, se entiende por requisito a una condición o capacidad que debe cumplir un sistema para
resolver un problema o satisfacer una necesidad. Son las funciones, características y restricciones que
debe cumplir un producto final.

La identificación de requisitos es una etapa crucial en cualquier proyecto software ya que marca el
camino a seguir en el desarrollo del sistema. Si no se identifican correctamente los requisitos, el
sistema no trabajará correctamente y no cumplirá ni sus especificaciones ni las expectativas del
cliente. Por ello, es importante que los requisitos sean claros, precisos y completos. 

Para hacer una correcta identificación de los requisitos, hay que tener en cuenta y conoces los 
dos tipos de requisitos que nos podemos encontrar: los requisitos funcionales, que describen las
funciones que debe realizar el sistema, y los requisitos no funcionales, que describen las
características que debe tener el sistema. 

Además, es importante conoces de donde se deben obtener los requisitos. Si leemos el apartado 
7.2 de la norma ISO 9001:2000, Realización del producto. Procesos relacionados con el cliente, 
este nos dice que los requisitos relativos se obtienen de: 

\begin{itemize}
    \item Los requisitos especificados por el cliente.
    \item Requisitos no especificados por el cliente pero necesarios para el uso especificado o
    para el uso previsto, cuando sea conocido.
    \item Requisitos legales y reglamentarios relacionados con el producto.
    \item Requisitos adicionales determinados por la organización.
\end{itemize}

Todo los requisitos que se detecten en esta fase deben ser revisados por la organización para 
asegurarse de que son correctos, completos, se entienden y pueden ser suministrados. Este proceso
de revisión debe llevarse a cabo a lo largo de todo el proyecto cuando alguno de estos requisitos
sufra alguna modificación.

\subsection{Requisitos de la plataforma OTT}
\label{subsec:analisis_requisitos_plataforma_ott}

En esta sección se describirán los requisitos recogidos para la plataforma OTT. Los requisitos detallados a 
continuación son los relacionados con la creación de la aplicación multiplataforma y multicliente, que recibe
la información y determina como mostrar esta. Dentro de estos requisitos no se incluyen los relacionados 
con otros componentes de la plataforma, como la CDN, la gestión de usuarios, la gestión de contenidos, etc.

\subsubsection{Requisitos funcionales}
\label{subsubsec:analisis_requisitos_funcionales}

\paragraph{Requisitos especificados por el cliente}
\label{par:analisis_requisitos_funcionales_cliente}

En este caso los requisitos no son unicamente de un cliente, sino que como ya se ha comentado
en la introducción, se han recogido los requisitos de varios clientes. Ejemplos de requisitos
especificos de clientes:

\paragraph{Generales}
\label{par:analisis_requisitos_funcionales_generales}

\begin{itemize}
    \item El usuario debe tener una pantalla donde poder buscar contenido.
    \item El usuario debe disponer de una pantalla con los ajustes de la aplicación.
\end{itemize}

\paragraph{Radio Television de Castilla y León. CyLTv Play}
\label{par:analisis_requisitos_funcionales_cyltvplay}

\begin{itemize}
    \item El usuario debe tener disponible una pantalla donde se muestre todo el contenido en directo.
    \item El usuario debe tener disponible una pantalla donde se muestre todo el contenido a la carta.
    \item El usuario debe tener una pantalla disponible para cada delegación del canal La 8.
\end{itemize}

\paragraph{Oh!Jazz}
\label{par:analisis_requisitos_funcionales_ohjazz}

\begin{itemize}
    \item El usuario debe poder iniciar y cerrar sesión.
    \item Los usuarios no loggeados no pueden acceder a todos los contenidos.
    \item El usuario debe poder cambiar de idioma la plataforma.
\end{itemize}

\paragraph{Requisitos no especificados por el cliente}
\label{par:analisis_requisitos_funcionales_no_cliente}

Este tipo de plataformas tienen unos requisitos comunes que no han sido especificados por los clientes,
pero que son necesarios para el correcto funcionamiento de la plataforma. Estos requisitos son:

\begin{itemize}
    \item La interfaz debe recibir la información necesaria para saber qué y cómo mostrar los contenidos.
    \item La aplicación debe ser capaz de determinar, para que cliente está trabajando y tomar decisiones 
    en función de esto.
    \item La aplicación debe ser capaz de determinar, para que plataforma está trabajando y tomar decisiones
    en función de esto.
    \item La aplicación debe crear una interfaz de usuario que sea atractiva, intuitiva y fácil de usar.
    \item La aplicación debe crear una interfaz que se adapte a distintas resoluciones.
    \item La aplicacion debe ser capaz de reproducir los contenidos multimedia en cualquier sistema operativo.
    \item La aplicación debe estar personalizada con los logos y colores de cada cliente.
    \item La aplicación debe funcionar con los botones del mando a distancia en el caso de las televisiones.
    \item La aplicación debe mostrar los datos correctamente sin alterar la información.
\end{itemize}

\subsubsection{Requisitos no funcionales}
\label{subsubsec:analisis_requisitos_no_funcionales}

\paragraph{Requisitos de rendimiento}
\label{par:analisis_requisitos_no_funcionales_rendimiento}

\begin{itemize}
    \item La aplicación debe ser capaz de mostrar los contenidos en el menor tiempo posible.
    \item La aplicación debe ser capaz de reproducir los contenidos multimedia sin cortes.
    \item La experiencia de usuario debe ser lo más fluida posible.
    \item La aplicación debe ser capaz de adaptarse a distintas resoluciones.
\end{itemize}

\paragraph{Requisitos de seguridad}
\label{par:analisis_requisitos_no_funcionales_seguridad}

\begin{itemize}
    \item La aplicación debe ser segura y no permitir accesos no autorizados.
    \item La aplicación debe proteger los datos de los usuarios.
    \item La aplicación debe proteger los datos de los clientes.
\end{itemize}

\paragraph{Requisitos de usabilidad}
\label{par:analisis_requisitos_no_funcionales_usabilidad}

\begin{itemize}
    \item La aplicación debe ser intuitiva y fácil de usar.
    \item La aplicación debe ser atractiva y agradable visualmente.
    \item La aplicación debe ser accesible para todo tipo de usuarios.
\end{itemize}

\paragraph{Requisitos de mantenimiento}
\label{par:analisis_requisitos_no_funcionales_mantenimiento}

\begin{itemize}
    \item La aplicación debe ser fácil de mantener y actualizar.
    \item La aplicación debe ser fácil de escalar.
    \item La aplicación debe ser fácil de desplegar.
\end{itemize}

\paragraph{Requisitos de escalabilidad}
\label{par:analisis_requisitos_no_funcionales_escalabilidad}

\begin{itemize}
    \item La aplicación debe ser capaz de adaptarse a un aumento de usuarios.
    \item La aplicación debe ser capaz de adaptarse a un aumento de contenidos.
    \item La aplicación debe ser capaz de adaptarse a un aumento de clientes.
\end{itemize}

\paragraph{Requisitos de disponibilidad}
\label{par:analisis_requisitos_no_funcionales_disponibilidad}

\begin{itemize}
    \item La aplicación debe estar disponible en todo momento.
    \item La aplicación debe ser capaz de recuperarse de fallos.
    \item La aplicación debe ser capaz de recuperarse de caídas.
\end{itemize}

\paragraph{Requisitos de compatibilidad}
\label{par:analisis_requisitos_no_funcionales_compatibilidad}

\begin{itemize}
    \item La aplicación debe ser compatible con todos los sistemas operativos.
    \item La aplicación debe ser compatible con todos los navegadores.
    \item La aplicación debe ser compatible con todos los dispositivos.
\end{itemize}

\paragraph{Requisitos de internacionalización}
\label{par:analisis_requisitos_no_funcionales_internacionalizacion}

\begin{itemize}
    \item La aplicación debe ser capaz de adaptarse a distintos idiomas.
    \item La aplicación debe ser capaz de adaptarse a distintas culturas.
    \item La aplicación debe ser capaz de adaptarse a distintas zonas horarias.
\end{itemize}

\paragraph{Requisitos de personalización}
\label{par:analisis_requisitos_no_funcionales_personalizacion}

\begin{itemize}
    \item La aplicación debe ser personalizable con los logos y colores de cada cliente.
    \item La aplicación debe ser personalizable con los logos y colores de cada plataforma.
    \item La aplicación debe ser personalizable con los logos y colores de cada delegación.
\end{itemize}

\paragraph{Requisitos de adaptabilidad}
\label{par:analisis_requisitos_no_funcionales_adaptabilidad}

\begin{itemize}
    \item La aplicación debe ser capaz de adaptarse a distintas resoluciones.
    \item La aplicación debe ser capaz de adaptarse a distintos dispositivos.
    \item La aplicación debe ser capaz de adaptarse a distintos navegadores.
\end{itemize}





