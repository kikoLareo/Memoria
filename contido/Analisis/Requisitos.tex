\section{Requisitos}
\label{sec:analisis_requisitos}

En software, un requisito es una condición o capacidad que debe cumplir un sistema para resolver 
un problema o satisfacer una necesidad, incluyendo funciones, características y restricciones.

Identificar correctamente los requisitos es crucial para el éxito del proyecto, ya que orienta 
el desarrollo y asegura que el sistema cumpla con las especificaciones y expectativas del cliente. 
Existen dos tipos principales de requisitos: funcionales, que describen las funciones que debe realizar 
el sistema, y no funcionales, que detallan las características que el sistema debe tener.

Según la norma ISO 9001:2000, los requisitos se obtienen de varias fuentes, incluyendo las especificaciones 
del cliente, las necesidades implícitas para el uso previsto, los requisitos legales y los adicionales 
determinados por la organización. Todos los requisitos deben revisarse para asegurar su precisión y 
viabilidad a lo largo del proyecto.

\subsection{Requisitos de la plataforma OTT}
\label{subsec:analisis_requisitos_plataforma_ott}

Los requisitos recogidos para la plataforma OTT se enfocan en la creación de una aplicación multiplataforma y 
multicliente, excluyendo componentes como la CDN o la gestión de usuarios y contenidos.

\subsubsection{Requisitos funcionales}
\label{subsubsec:analisis_requisitos_funcionales}

\paragraph{Requisitos especificados por el cliente}
\label{par:analisis_requisitos_funcionales_cliente}

Los requisitos funcionales abarcan tanto los generales como los específicos de cada cliente, por ejemplo:

\begin{itemize}
    \item \textbf{Generales:} Pantallas para buscar contenido y configurar la aplicación.
    \item \textbf{CyLTv Play:} Pantallas para contenido en directo, a la carta y por delegaciones.
    \item \textbf{Oh!Jazz:} Inicio/cierre de sesión, restricciones para usuarios no registrados, y opción de cambio de idioma.
\end{itemize}

\paragraph{Requisitos no especificados por el cliente}
\label{par:analisis_requisitos_funcionales_no_cliente}

Requisitos necesarios para el correcto funcionamiento de la plataforma:

\begin{itemize}
    \item La aplicación debe determinar y adaptar la interfaz según el cliente y la plataforma.
    \item La interfaz debe ser intuitiva, atractiva, adaptarse a distintas resoluciones y funcionar con mandos a distancia en TVs.
\end{itemize}

\subsubsection{Requisitos no funcionales}
\label{subsubsec:analisis_requisitos_no_funcionales}

\paragraph{Rendimiento:} Mostrar contenidos rápidamente, sin cortes, y asegurar una experiencia fluida.

\paragraph{Seguridad:} Garantizar la protección de datos de usuarios y clientes, y evitar accesos no autorizados.

\paragraph{Usabilidad:} La aplicación debe ser intuitiva, visualmente atractiva y accesible para todos los usuarios.

\paragraph{Mantenimiento:} La aplicación debe ser fácil de mantener, escalar y desplegar.

\paragraph{Escalabilidad:} La aplicación debe adaptarse a aumentos en usuarios, contenidos y clientes.

\paragraph{Disponibilidad:} La aplicación debe estar disponible siempre y poder recuperarse de fallos.

\paragraph{Compatibilidad:} La aplicación debe ser compatible con todos los sistemas operativos, navegadores y dispositivos.

\paragraph{Internacionalización:} La aplicación debe adaptarse a diferentes idiomas, culturas y zonas horarias.

\paragraph{Personalización:} La aplicación debe permitir la personalización con logos y colores de cada cliente y plataforma.

\paragraph{Adaptabilidad:} La aplicación debe ajustarse a distintas resoluciones, dispositivos y navegadores.



\subsection{Requisitos de la aplicación de análisis de datos}
\label{subsec:analisis_requisitos_aplicacion_analisis}

La aplicación de análisis de datos tiene como objetivo proporcionar herramientas para la recopilación, 
visualización y análisis de datos de uso. A continuación, se describen los requisitos funcionales y no funcionales 
de la aplicación:

\subsubsection{Requisitos funcionales}
\label{subsubsec:analisis_requisitos_analisis_funcionales}

\begin{itemize}
    \item Permitir la visualización de estadísticas de uso en tiempo real.
    \item Ofrecer filtros para datos por criterios como fecha, tipo de contenido o dispositivo.
    \item Generar análisis detallados de métricas como visualizaciones, reproducciones y usuarios únicos.
    \item Permitir una integración futura datos de diferentes fuentes y consolidarlos en un formato unificado.
\end{itemize}

\subsubsection{Requisitos no funcionales}
\label{subsubsec:analisis_requisitos_analisis_no_funcionales}

\paragraph{Rendimiento}
\label{par:analisis_requisitos_analisis_no_funcionales_rendimiento}

\begin{itemize}
    \item Procesar grandes volúmenes de datos de manera eficiente, manteniendo una respuesta rápida.
\end{itemize}

\paragraph{Seguridad}
\label{par:analisis_requisitos_analisis_no_funcionales_seguridad}

\begin{itemize}
    \item Proteger los datos sensibles, garantizando su confidencialidad e integridad.
\end{itemize}

\paragraph{Usabilidad}
\label{par:analisis_requisitos_analisis_no_funcionales_usabilidad}

\begin{itemize}
    \item Ofrecer una interfaz intuitiva y accesible para la interpretación de datos.
    \item Proporcionar una experiencia de usuario fluida en todos los dispositivos soportados.
\end{itemize}

\paragraph{Mantenimiento}
\label{par:analisis_requisitos_analisis_no_funcionales_mantenimiento}

\begin{itemize}
    \item Facilitar la actualización e integración de nuevas métricas y funcionalidades.
    \item Escalar la aplicación para manejar un aumento en la cantidad de datos.
\end{itemize}


