\section{Estudio de las características de los distintos sistemas operativos, dispositivos y tecnologías}
\label{sec:analisis_estudio}

Cuando se realizó el primer estudio del proyecto y los distintos estudios de viabilidad se analizaron las 
características de los distintos sistemas operativos, dispositivos y tecnologías que se podían utilizar para
el desarrollo de la aplicación. En este apartado se va a realizar un estudio de las características de estos 
elementos y como estos encajan entre sí para el desarrollo de la aplicación.

\subsection{Sistemas operativos}
\label{subsec:analisis_estudio_sistemas_operativos}

Cuando se comenzó a hablar del proyecto, el objetivo principal era el de comenzar con el desarrollo enfocado
en televisores, pero sin perder la perspectiva del resto de dispositivos. Además, también se tenía una idea
más o menos clara de los sistemas operativos con los que comenzar: Android TV, Tizen, webOs y Apple TV. Como ya 
se ha mencionado en la sección \ref{subsubsec:analisis_estudio_viabilidad_tecnica_aplicacion_tv}, Apple Tv tuvo
que ser descartado, pero los otros tres sistemas operativos siguen siendo válidos. Se estudiaron las características
de estos sistemas operativos a través de la documentación oficial y de la (escasa) información que se podía encontrar
en la red. Las características tecnológicas básicas ya han sido comentadas en la sección de fundamentos tecnológicos
\ref{subsec:adaptabilidad_multiplataforma}. En este apartado vamos a extender un poco más estas características, a 
través del análisis realizado para el proyecto.

\paragraph{características de los sistemas operativos}
\label{par:analisis_estudio_sistemas_operativos_caracteristicas}
Como ya se ha comentado las tecnologías que se querían utilizar para el desarrollo de la aplicación eran JavaScript,
HTML y CSS. El primer análisis que se realizó fue el de la compatibilidad de estas tecnologías con los distintos
sistemas operativos. Aqui se descubrió que tanto Tizen como webOs son compatibles con estas tecnologías, por lo 
menos con las funcionalidades más básicas y necesarias, ya que existen funcionalidades más recientes o complejas
que, aunque poco a poco se van integrando, todavía no están disponibles en estos sistemas operativos. Por otro lado,
Android TV propuso un reto, ya que, por defecto, no es compatible con estas tecnologías. Sin embargo, estudiando un 
poco más a fondo, se descubrió que Android TV permite el desarrollo de aplicaciones utilizando tecnologías web a través
de una herramienta llamada Apache Cordova \ref{par:adaptabilidad_android_tv_conversión}.

\paragraph{Empaquetado y distribución de aplicaciones}
\label{par:analisis_estudio_sistemas_operativos_empaquetado}
Otro punto importante que fue objeto de estudio fue como se debian configurar y empaquetar las aplicaciones para
poder probar y distribuir en los distintos sistemas operativos. De nuevo, Tizen y webOs son muy similares en este
aspecto, ya que aunque tienen formatos de empaquetado diferentes (wgt e ipk respectivamente), la estructura de los
archivos son muy parecidos: ambas plataformas requieren un archivo HTML principal, un archivo de configuración  (config.xml y package.json respectivamente) principal 
de la aplicacion como nombre, version, etc. , el código de la aplicación (JavaScript, HTML y CSS) y los recursos necesarios
(imágenes, fuentes, etc.). Por otro lado, la estructura de Android Tv es distinta ya que la estructura principal de la aplicación
es la de Apache Cordova y todo el código de este necesario para convertir la aplicación a un formato compatible con Android TV.
Dentro de esta estructuta encontramos: 
\begin{itemize}
    \item \textbf{www}: Directorio que contiene el código de la aplicación (JavaScript, HTML y CSS).
    \item \textbf{config.xml}: Archivo de configuración de la aplicación.
    \item \textbf{platforms}: Directorio que contiene las plataformas en las que se ha empaquetado la aplicación.
    \item \textbf{plugins}: Directorio que contiene los plugins de Cordova que se han utilizado en la aplicación.
\end{itemize}

\paragraph{Herramientas de desarrollo}
\label{par:analisis_estudio_sistemas_operativos_herramientas_desarrollo}
Por último, se estudiaron las herramientas de desarrollo que se podían utilizar para el desarrollo de la aplicación.
El desarrollo llevado a cabo ha sido a través de un código común desarrollado en Visual Studio Code. Sin embargo,
la ejecución de las pruebas y la generación de los archivos de empaquetado no se podía realizar de la misma manera.
Para cada SO se han creado unas plantillas con la estructura necesaria en cada caso y en las que se adjuntaba el código común
y se añadía el código específico para cada SO (un archivo de configuración básico necesario para unas pocas funcionalidades específicas de cada SO).

Cada SO ofrece distintas opciones para la ejecución, pruebas y empaquetado de las aplicaciones.

\begin{itemize}
    \item \textbf{Tizen}: Tizen Studio es la herramienta oficial de Samsung para el desarrollo de aplicaciones para Tizen. 
    Esta herramienta ofrece muchas funcionalidades para el desarrollo como la gestión de dispositivos, emuladores, etc. 
    Además, permite la generación de los archivos de empaquetado y la instalación de la aplicación en un televisor Samsung con Tizen OS.
    \item \textbf{WebOs}: LG ofrece una CLI (Command Line Interface) para la ejecución de las funcionalidades necesarias para la creación, 
    empaquetado y distribución de aplicaciones para WebOS. Además, también está disponible una extensión para Visual Studio Code que facilita aún más el desarrollo.
    \item \textbf{Android TV}: Las aplicaciones para Android Tv se pueden empaquetar y distribuir a través de Android Studio, que ofreces todas las funcionalidades
    necesarias para el desarrollo de las aplicaciones, desde escribir código hasta depurar e inspeccionar la aplicación. Sin embargo, como en este 
    proyecto no se está utilizando Android de forma nativa, si no que se esta adaptando a través de Cordova, Android Studio deja de tener todas las funcionalidades
    disponibles, debido a que ni tiene la estructura ni el lenguaje de programación que se utiliza por defecto en Android. La funcionalidad más afectada
    por esto es la de ejecución ya que no permite crear el apk correctamente porque no reconoce el código.  Por este motivo, el empaquetado e instalación 
    de la aplicación en un televisor Android se ha realizado a través de un terminal utilizando los comando ofrecidos por Cordova. 
\end{itemize}


\subsection{Dispositivos}
\label{subsec:analisis_estudio_dispositivos}

Para el estudio de los dispositivos, los puntos en los que se centró el análisis fueron la capacidad de los 
dispositivos para ejecutar la aplicación, la resolución de pantalla y el tamaño de la pantalla.

El segundo punto arrojó una respuesta muy clara: la aplicación debía ser adaptativa. La aplicación debía ser capaz
de adaptarse a cualquier resolución de pantalla, ya que se iba a ejecutar en televisores de distintas marcas y modelos, en 
móviles, en Pc, etc.

La filosofía de diseño que se tomó desde un principio fue la filosofia de diseño responsivo. Esta filosofía permite adaptarse
automáticamente al tamaño de la pantalla del dispositivo en el que se visualiza. Para ello, se utilizan unidades relativas
como porcentajes o vh y vw en lugar de unidades absolutas como px. De esta manera, la aplicación se adapta a cualquier
resolución de pantalla, ya sea un televisor, un móvil o una tablet. Por ejemplo, para este proyecto las pruebas se han realizado
en dos televisores (Samsung y Lg) que tienen una resolución de 1920x1080 ambas, y con una TCL(Android TV) que tiene una resolución
de 920*540. La aplicación se ha adaptado correctamente a los tres dispositivos, mostrando la misma información y funcionalidades
en todos ellos. 

Por otro lado, el primer punto fue un estudio clave, ya que la capacidad actual varia enormemente entre los distintos 
dispositivos sobre los que queremos trabajar, incluso entre dispositivos de la misma familia como los televisores,
existen diferencias de eficiencia y capacidad entre los distintos modelos y SO. Unas primeras pruebas de eficiencia
demostrarón que las funcionalidades que en un portatil funcionaban sin ningún tipo de problema ni sobrecarga para el dispositivo,
a la hora de probar en las televisiones los movimientos se veían pesados y lentos. Como no es posible estudiar todos los 
modelos de los diferentes dispositivos de las distintas marcas y la aplicación busca llegar al mayor número de dispositivos posible, 
el estudio se realizó sobre las herramientas y funcionalidades del desarrollo más eficientes.

Gracias a este estudio se consiguió mejorar la eficiencia y fluidez de la aplicación en los dispositivos de menor capacidad
buscando cual era la manera más eficiente de realizar las distintas funcionalidades. Por ejemplo, se demostró que utilizar
la propieda transform: translate() en lugar de la propiedad top o left para mover los elementos de la aplicación era mucho más
eficiente y fluido. 

Se encontró otra manera de aumentar la eficiencia, por ejemplo, con una propiedad de CSS que permite acelerar el renderizado
de la aplicación. Esta propiedad es la de will-change. Esta propiedad permite al navegador saber que un elemento va a cambiar
y que debe prepararse para ello. De esta manera, el navegador puede optimizar el renderizado de la aplicación y mejorar la
experiencia del usuario.


\subsection{Tecnologías}
\label{subsec:analisis_estudio_tecnologias}

En cuanto al estudio de las tecnologías, este es constante a lo largo de todo el desarrollo del proyecto. Si bien también 
lo es el estudio de los sistemas operativos y dispositivos en casos concretos, el estudio de las tecnologías es algo que se
realiza de manera constante, ya que no solo las tecnologías cambian y evolucionan, sino que es muy importante para un proyecto
buscar, estudiar y analizar las distintas tecnologías para sacarles el máximo partido y mejorar la eficiencia y la calidad de la
aplicación. Así, a lo largo de todo el desarrollo se han buscado las mejores funciones que nos ofrece JavaScript que nos pudieran 
ayudar con cada una de las funcionalidades a implementar. También se estudian las mejores librerias disponibles para utilizar en la
aplicación de análisis de datos, a través de Node.js, una potentisima herramienta con un catalogo de librerias muy amplio, el cual
si se estudia a fondo puede mejorar nuestros desarrollos de manera exponencial. Por otro lado, estudiar las mejores estructuras
y elementos que nos ofrece HTML y Css, y las mejores combinaciones entre estas tecnologías,  va a ser clave tanto para la eficiencia 
de la aplicación como para la presentación de la interfaz.



