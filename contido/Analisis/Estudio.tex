\section{Estudio de las características de los distintos sistemas operativos, dispositivos y tecnologías}
\label{sec:analisis_estudio}

Cuando se realizó el primer estudio del proyecto y los distintos estudios de viabilidad, se analizaron las 
características de los distintos sistemas operativos, dispositivos y tecnologías que se podían utilizar para
el desarrollo de la aplicación. En este apartado se realizará un estudio de las características de estos 
elementos y cómo se integran para el desarrollo de la aplicación.

\subsection{Sistemas operativos}
\label{subsec:analisis_estudio_sistemas_operativos}

Al inicio del proyecto, el objetivo principal era enfocarse en televisores, sin perder de vista otros dispositivos. 
Se tenía una idea clara de los sistemas operativos con los que empezar: Android TV, Tizen, webOS y Apple TV. Como 
se mencionó en la sección \ref{subsubsec:analisis_estudio_viabilidad_tecnica_aplicacion_tv}, Apple TV fue descartado, 
pero los otros tres sistemas operativos siguieron siendo viables. Se estudiaron las características de estos sistemas 
a través de la documentación oficial y la información disponible en la red. Las características tecnológicas básicas ya 
han sido comentadas en la sección de fundamentos tecnológicos \ref{subsec:adaptabilidad_multiplataforma}. En este apartado 
se ampliará el análisis realizado para el proyecto.

\paragraph{Características de los sistemas operativos}
\label{par:analisis_estudio_sistemas_operativos_caracteristicas}
Las tecnologías seleccionadas para el desarrollo de la aplicación fueron JavaScript, HTML y CSS. El primer análisis fue 
el de la compatibilidad de estas tecnologías con los sistemas operativos. Tanto Tizen como webOS son compatibles con estas tecnologías, 
aunque algunas funcionalidades avanzadas aún no están disponibles de forma nativa. Android TV presentó un reto, ya que no es compatible 
de forma nativa con estas tecnologías, pero se encontró una solución mediante Apache Cordova (ver \ref{par:adaptabilidad_android_tv_conversion}).

\paragraph{Empaquetado y distribución de aplicaciones}
\label{par:analisis_estudio_sistemas_operativos_empaquetado}
Otro aspecto importante fue cómo empaquetar y distribuir las aplicaciones. Tizen y webOS son similares, aunque tienen formatos de empaquetado 
diferentes (wgt e ipk, respectivamente). Ambos requieren un archivo HTML principal, un archivo de configuración (config.xml y package.json), 
el código de la aplicación (JavaScript, HTML y CSS) y los recursos necesarios (imágenes, fuentes, etc.). Android TV utiliza la estructura de 
Cordova, que incluye los siguientes directorios:

\begin{itemize}
    \item \textbf{www}: Contiene el código de la aplicación.
    \item \textbf{config.xml}: Archivo de configuración.
    \item \textbf{platforms}: Directorio con las plataformas en las que se ha empaquetado la aplicación.
    \item \textbf{plugins}: Contiene los plugins de Cordova utilizados.
\end{itemize}

\paragraph{Herramientas de desarrollo}
\label{par:analisis_estudio_sistemas_operativos_herramientas_desarrollo}
Se estudiaron las herramientas de desarrollo para cada sistema operativo. El desarrollo principal se realizó en Visual Studio Code, con 
plantillas específicas para cada SO. Las herramientas utilizadas fueron:

\begin{itemize}
    \item \textbf{Tizen}: Tizen Studio de Samsung permite la gestión de dispositivos, emuladores y el empaquetado de aplicaciones.
    \item \textbf{webOS}: LG ofrece una CLI y una extensión para Visual Studio Code, facilitando el desarrollo, empaquetado y distribución.
    \item \textbf{Android TV}: Aunque se podría usar Android Studio, la adaptación con Cordova requirió utilizar comandos en el terminal para empaquetar e instalar las aplicaciones.
\end{itemize}

\subsection{Dispositivos}
\label{subsec:analisis_estudio_dispositivos}

Se analizaron dos aspectos principales: la capacidad de los dispositivos para ejecutar la aplicación y la resolución de pantalla.

El estudio de resoluciones mostró la necesidad de que la aplicación fuera adaptativa, capaz de ajustarse a cualquier resolución de 
pantalla. Se utilizó un diseño responsivo, empleando unidades relativas (vh, vw, porcentajes) en lugar de px, lo que permitió que la 
aplicación se adaptara a diferentes dispositivos y resoluciones, como 1920x1080 (Samsung y LG) o 920x540 (TCL Android TV).

El rendimiento varía entre dispositivos. Las pruebas iniciales revelaron que, en televisores con menos capacidad, los movimientos 
resultaban lentos. Para mejorar la eficiencia, se optimizaron funciones como el uso de \texttt{transform: translate()} en lugar de 
\texttt{top} o \texttt{left}, y la utilización de \texttt{will-change} para mejorar el renderizado.

\subsection{Tecnologías}
\label{subsec:analisis_estudio_tecnologias}

El estudio de las tecnologías fue un proceso constante durante el desarrollo del proyecto. La elección de las mejores funciones de 
JavaScript, el análisis de bibliotecas de React y el uso eficiente de HTML y CSS fueron clave para mejorar tanto la eficiencia 
como la presentación de la interfaz.

En cuanto a la aplicación de análisis de datos, se estudiaron diversas librerías disponibles en React, lo que permitió mejorar 
exponencialmente el desarrollo y rendimiento de la plataforma de análisis. La combinación de estas tecnologías permitió optimizar 
la calidad y la eficiencia de la aplicación OTT y su plataforma analítica complementaria.
