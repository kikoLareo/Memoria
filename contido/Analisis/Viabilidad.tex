\section{Estudio de viabilidad}
\label{sec:analisis_estudio_viabilidad}

Desde el inicio, uno de los principales interrogantes fue si era factible desarrollar una aplicación única 
que pudiera generar versiones para distintos sistemas operativos y dispositivos. Este apartado analiza la 
viabilidad técnica y económica del proyecto, abordando las preguntas clave sobre la posibilidad de crear 
una solución híbrida entre dispositivos móviles y televisores, la existencia de tecnologías adecuadas y 
la eficiencia en su implementación.

\subsection{Viabilidad técnica}
\label{subsec:analisis_estudio_viabilidad_tecnica}

La \href{https://es.wikipedia.org/wiki/Viabilidad_t%C3%A9cnica#:~:text=Condici%C3%B3n%20que%20hace%20posible%20el,leyes%20de%20la%20naturaleza%20involucradas.}{viabilidad técnica} 
de un proyecto se refiere a la condición que hace posible llevar a cabo una idea o proyecto desde el punto de 
vista tecnológico. Para evaluar la viabilidad de este proyecto, es crucial determinar si existe una herramienta 
o tecnología que permita trabajar con un número suficiente de sistemas operativos y dispositivos para que el 
proyecto sea rentable. A continuación, se detalla este estudio.

En cuanto a infraestructura, la empresa ya dispone de los componentes necesarios y completamente funcionales 
para soportar la aplicación. Tanto el CMS, como la base de datos y los servidores de aplicaciones están operativos 
y son capaces de manejar la carga de trabajo que la aplicación requerirá, lo que hace que la viabilidad técnica 
en este aspecto sea positiva.

\subsubsection{Adaptación a web}
\label{subsubsec:analisis_estudio_viabilidad_tecnica_aplicacion_web}

Las aplicaciones web ofrecen flexibilidad y portabilidad, permitiendo el uso de diversas tecnologías y 
lenguajes de programación. Este entorno facilita el desarrollo, ya que la aplicación puede aprovechar 
tecnologías comunes para adaptarse a otros dispositivos, sin que la viabilidad técnica sea un obstáculo significativo.

\subsubsection{Adaptación a televisores}
\label{subsubsec:analisis_estudio_viabilidad_tecnica_aplicacion_tv}

El caso de los televisores es muy distinto. Aunque los televisores inteligentes actuales cuentan con un navegador
web, estos no son ni tan potentes ni tan cómodos de utilizar como los de los dispositivos móviles o los ordenadores, 
por lo que, aunque una persona pueda acceder a la aplicación desde su televisor, no es la opción ideal para 
estos dispositivos. La opción que aplica para este proyecto es la de crear una aplicación nativa para televisores.
Estas aplicaciones se ejecutan directamente en el sistema operativo del televisor, lo que requiere compatibilidad
con la tecnología utilizada para desarrollar la aplicación.

Los sistemas operativos que se estudiaron en un primer momento para este proyecto fueron los más utilizados en 
televisores inteligentes: Android TV, Tizen, webOS y Apple TV. La primera opción fue utilizar tecnologías como 
JavaScript, HTML y CSS. Tras analizar Tizen y webOS, se confirmó que ambas opciones permiten el desarrollo con 
estas tecnologías, lo que hacía viable la implementación técnica. Sin embargo, Android TV y Apple TV no permiten 
el desarrollo con estas tecnologías de manera nativa, ya que utilizan sus propios entornos de desarrollo. Para 
superar esta limitación en Android TV, se optó por \href{https://cordova.apache.org/}{Apache Cordova}, una 
herramienta que permite crear aplicaciones híbridas usando HTML, CSS y JavaScript. En el caso de Apple TV, no se 
encontró una solución viable en esta etapa para adaptar el código a sus tecnologías propietarias.

Aunque no se encontró una solución para Apple TV, la capacidad de desarrollar para Android TV, junto con Tizen 
y webOS, cubre aproximadamente un 27\% del mercado mundial de televisores inteligentes \cite{smarttv_marketshare}. 
Aunque no es un porcentaje muy alto, es suficiente para considerar el proyecto como viable.

Con el proyecto ya en marcha y versiones funcionales de la aplicación disponibles, se están estudiando nuevas 
implementaciones para otros sistemas operativos y dispositivos. Entre las opciones futuras está Vidaa OS 
(Hisense), que actualmente posee el 7.8\% del mercado mundial, posicionándose como el segundo sistema operativo 
más utilizado. Los primeros estudios sobre Vidaa OS han sido positivos, indicando que es compatible con las 
tecnologías elegidas, por lo que se planea continuar con el análisis y eventualmente integrar este sistema operativo.

En cuanto a Apple TV, se planifica desarrollarlo por separado y explorar cómo adaptar el código de la aplicación 
a su tecnología en el futuro, si es posible.

\subsubsection{Adaptación a dispositivos móviles}
\label{subsubsec:analisis_estudio_viabilidad_tecnica_aplicacion_movil}

Inicialmente, el enfoque no está en generar aplicaciones móviles nativas, sino en asegurar que la página web 
sea responsive para ofrecer una buena experiencia en dispositivos móviles. A medio plazo, se contempla la 
posibilidad de desarrollar para Android, con una probable exclusión inicial de iOS debido a las limitaciones tecnológicas.

\subsubsection{Conclusión}
\label{subsubsec:analisis_estudio_viabilidad_tecnica_conclusion}

El proyecto es técnicamente viable. A pesar de la limitación con Apple TV, el soporte para Android TV, Tizen 
y webOS, junto con la flexibilidad del desarrollo web, asegura que el proyecto puede avanzar con éxito.

\subsection{Viabilidad económica}
\label{subsec:analisis_estudio_viabilidad_economica}

La viabilidad económica se centra en evaluar los costos y beneficios del proyecto para determinar su rentabilidad.

\subsubsection{Costes}
\label{subsubsec:analisis_estudio_viabilidad_economica_costes}
Los costes que conlleva el desarrollo de este proyecto son los siguientes:

\begin{itemize}
    \item \textbf{Costes de desarrollo:} Incluyen el desarrollo de la aplicación, donde el único gasto significativo ha sido 
    el del desarrollador, ya que muchas herramientas y sistemas de prueba son gratuitos o ya existentes. Otros gastos, 
    como diseñadores o licencias, han sido cubiertos por los clientes.
    
    \item \textbf{Costes de mantenimiento:} La aplicación requiere un mantenimiento continuo, que incluye la solución de 
    problemas y la incorporación de nuevas funcionalidades para adaptarse a nuevos clientes y plataformas. Estos costos 
    se alinean con los costos de desarrollo.
\end{itemize}

\paragraph{Costes de desarrollo}
\label{par:analisis_estudio_viabilidad_economica_costes_desarrollo}

Como ya se indicó, el único gasto significativo en el desarrollo de la aplicación ha sido el del desarrollador. Para hacer el 
cálculo de este costo, según un artículo de la Universidad Alfonso X el Sabio \cite{coste_desarrollador},
el sueldo promedio de ingeniero informático puede oscilar entre los 20.450 y los 76.800 euros brutos por año, correspondiendo los
salarios más bajos a los desarrolladores junior y los más altos a los senior. Para este cálculo, vamos a utilizar por lo tanto 
los salarios más bajos, que son los más representativos para el perfil del desarrollador que ha trabajado en este proyecto. 
El coste por hora sería de 10 euros, y el total de horas trabajadas hasta el momento es de más o menos 800 horas. Por lo tanto, el coste
total del desarrollo sería de 8.000 euros. En caso de hablar de un desarrollador con más experiencia (entre 3 y 9 años), el salario al 
que puede aspirar ronda los 36.000 euros brutos al año, lo que supondría un coste de 18 euros por hora y un total de 14.400 euros.

\subsubsection{Beneficios}
\label{subsubsec:analisis_estudio_viabilidad_economica_beneficios}

Los ingresos provendrán de la creación y mantenimiento de la aplicación, con la posibilidad de cobrar por 
funcionalidades adicionales según lo requiera el cliente. Se realiza un primer cobro por la creación de la
aplicación teniendo en cuenta los costos de desarrollo, y a partir de ese momento se cobrará por siguientes fases de diseño 
de la aplicación. La cantidad a cobrar dependerá de la complejidad de la actualización y del tiempo que lleve implementarla.

\subsubsection{Conclusión}
\label{subsubsec:analisis_estudio_viabilidad_economica_conclusion}

Con las 800 horas trabajadas hasta el momento, ya hay un cliente que va a tener su aplicación en funcionamiento y en los marketplaces
este mes. Este tiempo de desarrollo y los costes ya estaban contemplados en el presupuesto inicial, por lo que
el proyecto es económicamente viable. Además, la reutilización del código y el cobro de mantenimiento y actualizaciones
aseguran que el proyecto sea rentable a largo plazo.

Con el tiempo y la mejora de la aplicación, se espera que el proyecto sea aún más rentable, teniendo en cuenta la capacidad de atraer
a nuevos clientes y seguir expandiendo la plataforma a nuevos dispositivos y sistemas operativos sin la necesidades de crear 
proyectos desde cero ni un departamento de desarrollo grande y específico para cada uno.

\subsection{Conclusión final}   
\label{subsec:analisis_estudio_viabilidad_conclusion}

En resumen, el proyecto es viable tanto técnica como económicamente. La adaptación a diferentes plataformas 
ha sido probada con éxito, y los costos previstos son manejables frente a los beneficios esperados, lo que 
justifica la continuación del desarrollo.
