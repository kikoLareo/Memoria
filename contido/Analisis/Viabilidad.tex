\section{Estudio de viabilidad}
\label{sec:analisis_estudio_viabilidad}

Cuando se comenzó a trabajar con la idea de crear una aplicación que pudiera generar aplicaciones
para distintos sistemas operativos y dispositivos a partir de un mismo código, una de las primeras
preguntas que surgieron fue si era posible hacerlo. Para obtener una respuesta a esta pregunta, era 
necesario saber si era viable una aplicacion hibrida entre dispospositivos móviles y televisores, 
si existia una tecnología que permitiera hacerlo en una cantidad de sistemas operativos y dispositivos
interesante y si era posible hacerlo de manera eficiente y económica. En este apartado se va a dar respuesta
a estas preguntas, analizando la viabilidad técnica y económica del proyecto.

\subsection{Viabilidad técnica}
\label{subsec:analisis_estudio_viabilidad_tecnica}

La \href{https://es.wikipedia.org/wiki/Viabilidad_t%C3%A9cnica#:~:text=Condici%C3%B3n%20que%20hace%20posible%20el,leyes%20de%20la%20naturaleza%20involucradas.}{viabilidad técnica} 
de un proyecto se refiere a la condición que hace posible realizar un proyecto o llevar a cabo una idea
desde el punto de vista de la tecnología. Para estudiar la viabilidad de este proyecto es necesario saber
si existe una herramienta o tecnología que permita trabajar en una cantidad de sistemas operativos y dispositivos
lo suficientemente grande como para que compense realizar el proyecto. A continuación se detallará este estudio. 

En cuanto a infraestructura, la empresa ya dispone de los componentes necesarios, totalmente funcionales, 
para dar soporte a la aplicación. Tanto el CMS, como la base de datos y servidores de aplicaciones están
en pleno funcionamiento y son capaces de soportar la carga de trabajo que supondría la aplicación, por lo que
la viabilidad técnica en este aspecto es positiva.

\subsubsection{Adaptación a web}
\label{subsubsec:analisis_estudio_viabilidad_tecnica_aplicacion_web}

La aplicación web es una aplicación que se ejecuta en un navegador web. Este tipo de aplicaciones permiten 
una gran flexibilidad y portabilidad, no solo a la hora de ser explotadas por los usuarios, sino también a la
hora de ser desarrolladas. Estas aplicaciones permiten utilizar una enorme variedad de tecnologías y lenguajes
de programación, por lo que este requisito no es determinante a la hora de estudiar la viabilidad técnica del proyecto
ya que podremos utilizar casi cualquier tecnología que nos permita adaptarnos a los otros dispositivos.

\subsubsection{Adaptación a televisores}
\label{subsubsec:analisis_estudio_viabilidad_tecnica_aplicacion_tv}

El caso de los televisores es muy distinto. Aunque los televisores inteligentes actuales cuentan con un navegador
web, estos no son ni tan potentes ni tan comodos de utilizar como los de los dispositivos móviles o los ordenadores, 
por lo que aunque una persona pueda acceder a la aplicación desde su televisor, no es la opción que queremos para 
los televisores. La opción que aplica para este proyecto es la de crear una aplicación nativa para televisores.
Estas aplicaciones son aplicaciones que se ejecutan en el sistema operativo del televisor, por lo que es necesario
que el sistema operativo del televisor sea compatible con la tecnología que utilicemos para desarrollar la aplicación.

Los sistemas operativos que se estudiaron en un primer momento para este proyecto son los más utilizados en 
televisores inteligentes: Android Tv, Tizen, webOs y Apple Tv. El estudio comenzó con buscar que tecnologías 
permite cada opción de modo nativo. La opción que se buscaba era la de utilizar JavaScript, HTML y CSS para
el desarrollo. Analizando Tizen y webOs encontramos que ambas opciones permiten el desarrollo a través de 
estas tecnologías, por lo que la viabilidad tecnica comenzaba a ser posible. El problema llegó con Android Tv
y Apple Tv. Estos sistemas operativos no permiten el desarrollo de aplicaciones a través de estas tecnologías,
si no que tienen sus propias tecnologías. Buscando soluciones encontramos una opción para Android Tv, que es
\href{https://cordova.apache.org/}{Apache Cordova}. Apache Cordova es una herramienta que permite desarrollar
aplicaciones híbridas para distintos sistemas operativos a través de HTML, CSS y JavaScript. En el caso de Apple Tv
no hubo suerte y no se encontró una solución viable con la que conseguir desarrollar con la tecnología escogida y 
"traducir" o "adaptar" el código a la tecnología de Apple Tv como se hizo con Android Tv. 

Sin embargo, aunque no se encontró una solución para Apple Tv, el hecho de que se encontrara una solución para
Android Tv y que las otras dos opciones permitieran el desarrollo con las tecnologías escogidas supone 
trabajar sobre un 27\% del mercado mundial de televisores inteligentes \cite{smarttv_marketshare}. Aunque no es
un porcentaje muy alto, es un porcentaje lo suficientemente grande como para que el proyecto sea viable.

Ahora que el proyecto ya está en marcha y ya existen versiones completamente funcionales, se están estudiando
las implementaciones de nuevos sistemas operativos y dispositivos. Entre estas opciones está Vidaa OS (Hisense), que 
tiene el 7.8\% del mercado mundial ahora mismo, posicionandose como el segundo sistema operativo más utilizado. El 
primer estudio que se hizo sobre este sistema operativo fue positivo, ya que indicó que se podía desarrollar con las
tecnologías escogidas, por lo que a corto plazo se continuará con el estudio y se comenzará con la integración
si el estudio de viabilidad técnica es positivo.

Por otro lado, Apple Tv se desarrollará por separado, y se buscará la manera de adaptar el código de la aplicación 
a la tecnología de Apple Tv e Ios en un futuro, si fuera posible. 

\subsubsection{Adaptación a dispositivos móviles}
\label{subsubsec:analisis_estudio_viabilidad_tecnica_aplicacion_movil}

En una primera etapa del proyecto, los objetivo no están centrados ahora mismo en que este proyecto generé aplicaciones
para dispositivos móviles, pero si la adaptación de la página web para que sea \href{https://es.wikipedia.org/wiki/Dise%C3%B1o_web_adaptable}{responsive}
y permita una buena experiencia de usuario en dispositivos móviles. Además, si que existe está implemetación como objetivo
a medio plazo por lo que el código permite una rapida integración de esta funcionalidad. Sin embargo, todo indica que
al igual que para los televisores, el sistema operativo Ios no permitirá el desarrollo con las tecnologías escogidas, por
lo que por el momento se centraría en Android y en la adaptación de la página web para dispositivos móviles.



\subsubsection{Conclusión}
\label{subsubsec:analisis_estudio_viabilidad_tecnica_conclusion}

La conclusión tras el estudio de viabilidad técnica en las diferentes plataformas es que el proyecto es viable.
Aunque no se ha encontrado una solución para Apple Tv, el hecho de que se haya encontrado una solución para Android Tv
y que las otras dos opciones permitan el desarrollo con las tecnologías escogidas, además de la gran flexibilidad que
ofrece el desarrollo web, hacen que el proyecto sea viable técnicamente.


\subsection{Viabilidad económica}
\label{subsec:analisis_estudio_viabilidad_economica}

La \href{https://es.wikipedia.org/wiki/Viabilidad_econ%C3%B3mica#:~:text=La%20viabilidad%20econ%C3%B3mica%20es%20la,de%20un%20proyecto%20o%20inversi%C3%B3n.}{viabilidad económica}
de un proyecto es la capacidad que tiene un proyecto de generar beneficios económicos. Para estudiar la viabilidad
económica de este proyecto es necesario estudiar los costes que conlleva el desarrollo del proyecto y los beneficios
que se pueden obtener. A continuación se detallará el estudio realizado, sin mostrar cifras concretas ya que es información
confidencial.

\subsubsection{Costes}
\label{subsubsec:analisis_estudio_viabilidad_economica_costes}

Los costes que conlleva el desarrollo de este proyecto son los siguientes:

\begin{itemize}
    \item Costes de desarrollo: el proyecto requiere de un equipo que se encargue de desarrollar la aplicación. En estos costes 
    se incluyen los costes de desarrollar la aplicación, los diseñadores, herramientas... Para este proyecto el único coste
    que se ha tenido ha sido el de los desarrolladores, en este caso 1 persona únicamente. El resto de costes o bien no han sido
    necesarios (las herramientas son gratuitas, el sistema de pruebas también...) o bien son gastos del cliente (diseñadores, licencias del marketplace...).
    o bien son gastos que ya existían (servidores, base de datos...).
    \item Costes de mantenimiento: el proyecto requiere de un mantenimiento continuo para asegurar su correcto funcionamiento.
    En este caso, la aplicación está en continuo desarrollo y mejora ya que a medida que entren nuevos clientes y se adapte a nuevos
    sistemas operativos y dispositivos, se irán añadiendo nuevas funcionalidades y mejoras para todos los clientes. Dentro de este
    desarrollo entra también el mantenimiento de la aplicación, que se encarga de solucionar los problemas que puedan surgir. Por lo tanto, 
    los costes derivados del mantenimiento son los mismos que los costes de desarrollo. 

\end{itemize}

\subsubsection{Beneficios}
\label{subsubsec:analisis_estudio_viabilidad_economica_beneficios}

Los ingresos previstos para este proyecto vienen derivados de la contratación de los servicios de la empresa. La empresa
cobra una cantidad fija por la creación de la aplicación y una cantidad fija por el mantenimiento de la misma. 

Dentro de este cobro están incluidos los costes de desarrollo y mantenimiento y se indica que toda aquella funcionalida extra que no está contemplada
ni en el desarrollo ni en el contrato, se cobrará aparte. 

\subsubsection{Conclusión}
\label{subsubsec:analisis_estudio_viabilidad_economica_conclusion}

Uno de los baneficios de este proyecto es la reducción de costes, ya que no es necesario realizar un desarrollo para todos los clientes, 
evitando los gastos de desarrollo y mantenimiento de una aplicación en cada proyecto de cada cliente. Esta reducción de costes es debido
a la reutilización del , reducción del tiempo de desarrollo, reducción de personal necesario... 

Por lo que si el estudio y la previsión de gastos y beneficios es correcta, y esto se refleja correctamente en el contrato, el proyecto
es viable económicamente.


\subsection{Conclusión final}   
\label{subsec:analisis_estudio_viabilidad_conclusion}

Tras el estudio de viabilidad técnica y económica, se puede concluir que el proyecto es viable. La viabilidad técnica
se ha demostrado a través de la creación de la aplicación y la adaptación a los distintos sistemas operativos y dispositivos
y la viabilidad económica se ha demostrado a través de los costes y beneficios previstos. Por lo tanto, se puede afirmar que
el proyecto es viable y que se puede continuar con su desarrollo.

