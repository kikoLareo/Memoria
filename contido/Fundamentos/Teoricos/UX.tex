\subsection{Experiencia de usuario (UX)}
\label{sec:fundamentos_teoricos_ux}

A diferencia de los medios tradicionales, donde el contenido se emite en un horario fijo y de una forma específica, 
sin permitir que el usuario interactúe con él, los servicios OTT ofrecen a los usuarios la posibilidad de navegar 
por un catálogo de contenidos, seleccionar lo que desean ver, consultar su información, su estructura (temporadas, 
episodios, etc.), y acceder al contenido en cualquier momento y lugar, sin restricciones. Esta flexibilidad y 
control que los usuarios desean sobre el contenido es tanto una de las principales ventajas de los servicios OTT 
como uno de los mayores desafíos para los diseñadores.

La experiencia de usuario (UX) \cite{UX} se define como el conjunto de factores y elementos que afectan la 
interacción del usuario con la interfaz de un sistema, dispositivo o aplicación. Estos factores son clave 
para determinar si un usuario disfruta de la experiencia de uso de un producto o servicio. Es fundamental 
prestar atención a estos aspectos en el diseño de cualquier aplicación, y más aún en el caso de una plataforma 
OTT, donde los usuarios buscan una experiencia fluida, personalizada, intuitiva y atractiva.

\subsubsection{Principios de diseño de UX}
\label{subsec:fundamentos_teoricos_ux_principios}

El diseño de una experiencia de usuario efectiva se basa en una serie de principios y prácticas que deben 
adaptarse a cada proyecto, sus necesidades y características. En el caso de las aplicaciones OTT, dos de los 
principios fundamentales son la usabilidad y la simplicidad. La plataforma debe ser fácil de usar, intuitiva 
y accesible para todo tipo de usuarios, independientemente de su nivel de experiencia o conocimientos técnicos. 
Esto garantiza que cualquier usuario pueda utilizar nuestras plataformas. Otros principios importantes incluyen 
la consistencia y la adaptabilidad entre dispositivos y plataformas, la accesibilidad para asegurar que todos 
los usuarios puedan disfrutar de la plataforma, y la personalización para ofrecer una experiencia única y 
relevante a cada usuario.

\subsubsection{Elementos clave de la UX en una plataforma OTT}
\label{subsec:fundamentos_teoricos_ux_elementos}

Las plataformas OTT son aplicaciones con una serie de componentes clave que son esenciales para su funcionamiento. 
Estos componentes no solo deben estar presentes, sino que también deben funcionar de manera óptima para garantizar 
una buena experiencia de usuario. Los componentes más importantes son:

\begin{itemize}
    \item \textbf{Contenidos:} El elemento central de la plataforma, en torno al cual gira toda la experiencia de usuario. Los contenidos deben ser fáciles de encontrar, navegar y consumir, y deben estar presentados de forma atractiva y organizada.
    \item \textbf{Agrupaciones de contenidos:} Ya sea en listas, series, temporadas, categorías, géneros, etc., las agrupaciones de contenidos permiten a los usuarios explorar y descubrir nuevos contenidos de forma sencilla y atractiva.
    \item \textbf{Metadata:} Información sobre cualquiera de los elementos que conforman la plataforma, como descripciones, fechas, títulos, actores, etc. La metadata proporciona el contexto necesario para que los usuarios entiendan y valoren los contenidos.
    \item \textbf{Páginas:} Cada menú, serie, temporada, episodio, etc., debe tener su propia página con la información y funcionalidades necesarias para facilitar el uso de la plataforma.
    \item \textbf{Botones y controles:} Los botones y controles de reproducción, pausa, adelanto, retroceso, etc., deben ser intuitivos y accesibles para que los usuarios puedan interactuar con el contenido de forma sencilla.
\end{itemize}

\subsubsection{Elementos clave para la UX de la plataforma de análisis de datos}
\label{subsec:fundamentos_teoricos_ux_elementos_analisis}

La base que se siguió para la creación de la plataforma de análisis de datos fue la de ofrecer una experiencia de usuario
sencilla y efectiva, que permitiera a los usuarios visualizar y analizar los datos de una manera clara.

La plataforma de análisis de datos es una herramienta de gran utilidad para los usuarios, ya que les permite
visualizar y analizar los datos de una manera sencilla y efectiva. Algunos de los elementos clave para la
experiencia de usuario de esta plataforma son:

\begin{itemize}
    \item \textbf{Visualización de datos:} La plataforma debe permitir a los usuarios visualizar los datos de forma clara y efectiva, utilizando gráficos, tablas y otros elementos visuales para facilitar la comprensión de la información.
    \item \textbf{Análisis de datos:} Los usuarios deben poder analizar los datos de manera sencilla, utilizando herramientas como filtros, agrupaciones, páginas, etc., para extraer información relevante y tomar decisiones informadas.
    \item \textbf{Generación de análisis:} La plataforma debe permitir a los usuarios generar análisis personalizados con los datos seleccionados, para conseguir una mayor profundidad y detalle en la información.
    \item \textbf{Interfaz de usuario:} La interfaz de usuario debe ser intuitiva y fácil de usar, con un diseño limpio y atractivo que facilite la navegación y el uso de la plataforma.
\end{itemize}