\subsection{Experiencia de usuario (UX)}
\label{sec:fundamentos_teoricos_ux}

A diferencia de los medios tradicionales, donde el contenido se emite en un horario fijo y de una forma
especifica, sin darle al usuario la posibilidad de interactuar con el contenido, los servicios OTT permiten
a los usuarios navegar por el catálogo de contenidos, seleccionar lo que desean ver, su información, 
su genetica(temporadas, episodios...) y cualquier otra información relevante, y ver el contenido como y cuando
quieran, sin restricciones de tiempo o lugar. Esta flexibilidad y control que desean los usuarios sobre el contenido
es tanto una de las mejores características de los servicios OTT como uno de los mayores desafíos para los diseñadores.

La experiencia de usuario (UX) \cite{UX} se define como los factores y elementos relativos a la interacción del usuario
con la interfaz de un sistema, dispositivo o aplicación. Estos factores son claves para definir si un usuario disfruta
o no de la experiencia de uso de un producto o servicio. Es fundamental prestar atención a estos factores en el diseño
de cualquier aplicación, y más aún en el caso de una plataforma OTT, donde los usuarios buscan una experiencia fluida,
personalizada, intuitiva y atractiva.

\subsubsection{Principios de diseño de UX}
\label{subsec:fundamentos_teoricos_ux_principios}

El diseño de un UX efectivo se basa en una de principios y prácticas que deben adaptar a cada proyecto, a sus necesidades
y características. En el caso de este tipo de aplicaciones dos de estos prinpios son la usabilidad y la simplicidad.
La plataforma para el consumo de los contenidos debe ser fácil de usar, intuitiva y accesible para todo tipo de usuarios,
independientemente de su nivel de experiencia o conocimientos técnicos. Esto nos asegurará que cualquier usuario podrá 
hacer uso de nuestras plataformas. Otros principios importantes son la consistencia y la adaptabilidad entre dispositivos
y plataformas; la accesibilidad, para garantizar que todos los usuarios puedan disfrutar de la plataforma; y la personalización,
para ofrecer una experiencia única y relevante a cada usuario.

\subsubsection{Elementos clave de la UX en una plataforma OTT}
\label{subsec:fundamentos_teoricos_ux_elementos}

Las plataformas OTT son aplicaciones con una serie de componentes clave que no pueden faltar ya que son necesarios para 
el funcionamiento de la misma. De la misma manera, estos componenetes deben comportarse de una manera determinada. Los
componenetes más importantes son: 

\begin{itemize}
    \item \textbf{Contenidos} Elemento central de la plataforma sobre el que gira toda la experiencia de usuario. Los contenidos
    deben ser fáciles de encontrar, navegar y consumir, y deben estar presentados de forma atractiva y organizada.
    \item \textbf{Agrupaciones de contenidos} Ya sea en listas, series, temporadas, categorías, géneros, etc. Las agrupaciones
    de contenidos permiten a los usuarios explorar y descubrir nuevos contenidos de forma sencilla y atractiva.
    \item \textbf{Metadata} información sobre cualquiera de los elementos que conforman la plataforma: descripciones, fechas,
    títulos, actores, etc. Aporta al usuario el contexto necesario para entender y valorar los contenidos.
    \item \textbf{Páginas} Cada menú, serie, temporada, episodio, etc. debe tener su propia página con la información y funcionalidades
    necesarias para hacer uso de la plataforma.
    \item \textbf{Botones y controles} Los botones y controles de reproducción, pausa, adelanto, retroceso, etc. deben ser intuitivos
    y accesibles para que los usuarios puedan interactuar con el contenido de forma sencilla.
\end{itemize}

