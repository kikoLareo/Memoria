\subsection{Importancia de la Analítica de Datos en Plataformas OTT}
\label{subsec:fundamentos_analitica_importancia}

La analítica de datos juega un papel crucial en las plataformas OTT, ya que permite a las empresas obtener 
una comprensión profunda del comportamiento del usuario y, en consecuencia, optimizar la experiencia de usuario (UX). 
A través de la recopilación y análisis de datos, las plataformas pueden identificar patrones de consumo, preferencias 
del usuario, y áreas de mejora, lo que a su vez facilita la toma de decisiones informadas sobre la oferta de contenido 
y el desarrollo de nuevas funcionalidades.


\subsubsection{Herramientas y Frameworks: Matomo}
\label{subsubsec:fundamentos_analitica_matomo}

Para la analítica de datos en esta plataforma OTT, se ha utilizado Matomo, una herramienta de código abierto que permite 
la recopilación y análisis de datos de manera eficaz y segura. Matomo fue seleccionado por su capacidad para integrarse 
fácilmente con la plataforma y su flexibilidad en la configuración de métricas específicas según las necesidades del cliente.

Matomo permite la captura de datos de usuario en tiempo real, incluyendo métricas clave como el tiempo de visualización, 
las tasas de abandono, y el comportamiento de navegación. Además, ofrece una interfaz intuitiva para la visualización de 
estos datos, facilitando la interpretación y toma de decisiones.

\subsubsection{Integración con la Plataforma OTT}
\label{subsubsec:fundamentos_analitica_integracion}

La integración de Matomo con la plataforma OTT se realiza a través de APIs que permiten la recopilación automatizada de 
datos de usuario desde diversos puntos de interacción dentro de la aplicación. Estos datos son procesados y almacenados 
para su posterior análisis, permitiendo a los administradores acceder a informes detallados y dashboards interactivos 
que ofrecen una visión completa del uso y rendimiento de la plataforma.

\subsection{Aplicación de los Principios de UX en la Analítica de Datos}
\label{subsec:fundamentos_ux_analitica}

Al igual que en la plataforma OTT principal, los principios de UX son fundamentales en la aplicación de análisis de datos. 
Es esencial que la interfaz de la aplicación de analítica sea intuitiva, con una presentación clara y accesible de los datos, 
lo que permitirá a los administradores tomar decisiones rápidas y efectivas. La interactividad y la capacidad de personalizar 
informes son aspectos clave que garantizan que los usuarios puedan filtrar y segmentar los datos según sus necesidades 
específicas, optimizando así la utilidad de la herramienta.
