\subsection{OTT (Over-the-Top)}
\label{sec:fundamentos_teoricos_ott}

\subsubsection{Definición, funcionamiento y clasificación de los servicios OTT}
\label{sec:ott_definicion_funcionamiento_clasificacion}

En radiodifusión, un servicio OTT (siglas en inglés de \textit{over-the-top}) consiste en la transmisión de audio, vídeo y otros 
contenidos a través de internet sin la implicación de los operadores tradicionales en el control o la distribución de los mismos \cite{OTT}.

Estos servicios han transformado radicalmente la forma en que los usuarios consumen contenido audiovisual y cómo los proveedores 
lo distribuyen. La clave de esta transformación radica en la capacidad de acceder a contenidos de manera inmediata, sin necesidad 
de esperar a que se emitan en televisión, radio, cine u otros medios tradicionales. Además, los servicios OTT permiten a los usuarios 
disfrutar de contenidos en cualquier lugar y momento, siempre que cuenten con una conexión a internet.

Su funcionamiento se basa en un catálogo de contenidos que los usuarios pueden explorar según sus preferencias. Cada catálogo 
incluye una variedad de películas, series, documentales y otros tipos de contenido, presentados con descripciones detalladas, 
géneros, calificaciones de usuarios y recomendaciones basadas en el historial de visualización. Los usuarios pueden navegar por 
el catálogo, seleccionar el contenido que desean ver y comenzar la transmisión de manera inmediata, beneficiándose de la capacidad 
de pausar, rebobinar o adelantar el contenido según su conveniencia. Este sistema flexible no solo permite a los usuarios tener control 
total sobre lo que ven y cuándo lo ven, sino que también facilita la entrega personalizada de contenido, adaptándose continuamente a los 
gustos individuales de cada usuario.

El contenido ofrecido por los servicios OTT es muy variado, abarcando desde películas y series hasta música, deportes, noticias, cursos 
educativos y recursos generados por los usuarios. Esta diversidad permite que los servicios OTT se adapten a diferentes preferencias y 
necesidades, ofreciendo una experiencia personalizada para cada tipo de usuario. Además, el acceso a datos masivos sobre los hábitos de 
consumo de los usuarios ha permitido a las plataformas OTT optimizar sus catálogos y estrategias de marketing, logrando una conexión más 
profunda y relevante con su audiencia.

En función del modelo de negocio y la forma en que se ofrece el contenido, los principales tipos de servicios OTT son:
\begin{itemize}
\item \textbf{SVOD (Subscription Video on Demand):} Servicios que ofrecen acceso ilimitado a un catálogo de contenido a cambio de una suscripción mensual o anual.
\item \textbf{AVOD (Advertising Video on Demand):} Servicios que proporcionan contenido gratuito a los usuarios a cambio de la visualización de anuncios publicitarios.
\item \textbf{TVOD (Transactional Video on Demand):} Servicios que permiten a los usuarios alquilar o comprar contenido de manera individual.
\end{itemize}

\subsubsection{Evolución y tendencias de los servicios OTT}
\label{sec:ott_evolucion_tendencias}

El crecimiento de los servicios OTT ha sido impulsado por varios factores clave, como la mejora de las conexiones a internet, que ha permitido una 
transmisión más rápida y de mayor calidad, y los avances tecnológicos en la compresión de video y la optimización de redes, que han mejorado 
significativamente la experiencia del usuario. Estas aplicaciones deben ser diseñadas para escalar eficientemente, manejando grandes volúmenes 
de tráfico durante eventos en vivo o lanzamientos populares, lo que plantea desafíos en términos de rendimiento. Además, la personalización se 
ha convertido en un diferenciador esencial, impulsada por algoritmos de inteligencia artificial que analizan los patrones de consumo para ofrecer 
contenido relevante, mejorando la experiencia del usuario y fomentando la lealtad a la plataforma. La proliferación de dispositivos móviles y la 
preferencia por el contenido instantáneo han consolidado a los servicios OTT como la opción preferida frente a los medios tradicionales, reflejando 
una evolución en los hábitos de consumo hacia la conveniencia y la personalización.

El entorno competitivo también ha impulsado innovaciones en la oferta de servicios OTT. Las empresas han tenido que diferenciarse mediante la 
creación de contenido original exclusivo, el uso de inteligencia artificial para mejorar la personalización y la implementación de modelos de 
negocio híbridos que combinan suscripciones, publicidad y transacciones para maximizar el alcance y la rentabilidad. Estas estrategias no solo 
buscan atraer y retener a los usuarios, sino también garantizar la sostenibilidad del servicio en un mercado cada vez más fragmentado.
