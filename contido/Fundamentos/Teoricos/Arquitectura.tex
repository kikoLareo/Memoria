\subsection{Arquitectura}
\label{subsec:fundamentos_teoricos_arquitectura}

Se entiende Arquitectura Software como los modelos y estándares que sirven de base
para el diseño e implementación de sistemas de software. Indica la estructura, funcionamiento
e interacción de los componentes de software \cite{ArqSoftware}.

El diseño de la arquitectura de un sistema de software para una aplicación OTT es clave para 
garantizar el correcto funcionamiento de la misma y la entrega eficiente, escalable y confiable 
de los contenidos. Existen varias posibilidades para el diseño de la arquitectura de una aplicación
de estas características como la Arquitectura Monolítica, la Arquitectura Orientada a Servicios (SOA)...
En este caso, se ha optado por una arquitectura basada en microservicios. 

\subsubsection{Arquitectura basada en microservicios}
\label{subsec:fundamentos_teoricos_arquitectura_microservicios}

La arquitectura basada en microservicios es un enfoque para el diseño de aplicaciones que consiste en
un conjunto de pequeños servicios, los cuales se ejecutan en su propio proceso y se comunican con 
mecanismos ligeros (normalmente una API de recursos HTTP como es el caso de este proyecto) \cite{Microservices}.
Cada microservicios está especializado en una tarea concreta y trabaja de forma independiente. 

De este modo, tendremos las funcionalidades de la plataforma OTT distribuidas en diferentes microservicios,
aislando las funcionalidades y permitiendo que cada uno de ellos pueda ser desarrollado, desplegado y
escalado de forma independiente. Esto facilita la evolución de la plataforma, ya que se puede mejorar 
cada microservicio con la confianza de que si se hace correctamente no afectará al resto de la plataforma.
Lo mismo ocurre con la tolerancia a fallos, ya que si la arquitectura está bien diseñada, un fallo en un
microservicio no debería afectar al resto de la plataforma, o deberia hacerlo lo menos posible.

Ejemplos de microservicios más comunes en una plataforma OTT son:
\begin{itemize}
\item \textbf{Gestión de usuarios:} encargado de la gestión de los usuarios de la plataforma,
incluyendo el registro, autenticación y autorización de los mismos. 
\item \textbf{Gestión de contenidos:} encargado de la gestión y almacenamiento de los contenidos de la plataforma,
de los metadatos de los contenidos y de los ficheros multimedia.
\item \textbf{Orquestador:} encargado de orquestar las peticiones de los usuarios a los diferentes microservicios
de la plataforma.
\item \textbf{Recomendaciones:} encargado de la generación de recomendaciones personalizadas para los usuarios
de la plataforma.
\end{itemize}


\subsubsection{Componentes de la arquitectura}
\label{subsec:fundamentos_teoricos_arquitectura_componentes}

En una plataforma OTT, además de los microservicios, existen varios componentes clave que conforman la 
infraestructura técnica y que son esenciales para su funcionamiento. Estos componentes se encargan de 
soportar las operaciones críticas, desde la ingesta y distribución de contenido hasta la experiencia de 
usuario final. Algunos de los componentes más importantes son:

\begin{itemize}
\item \textbf{CDN (Content Delivery Network):} Red de distribución de contenidos que permite la entrega
rápida y eficiente de contenidos multimedia a los usuarios finales. La CDN almacena copias de los contenidos
en servidores distribuidos geográficamente, lo que reduce la latencia y mejora la velocidad de carga de los
contenidos.
\item \textbf{Ingesta y gestión de contenidos:} Componente encargado de recibir y procesar y gestionar todos 
los contenidos que van a estar disponibles en la plataforma, permitiendo a los administradores de contenido
subir, editar, etiquetar, organizar y , en general, gestionar los contenidos de la plataforma y sus metadatos, 
preparando estos para su distribución.
\item \textbf{Gestión de usuarios:} Encargado de toda gestión relacionada con los usuarios de la plataforma,
incluyendo el registro, autenticación, autorización, gestión de perfiles y preferencias, etc.
\item \textbf{Interfaz de usuario:} Es la parte visible de la plataforma, donde los usuarios interactúan con
la misma. Es el componente encargado de recibir la información de los microservicios y presentarla de forma
amigable al usuario.
\item \textbf{monitorización y análisis:} Componente encargado de monitorizar el rendimiento de la plataforma
y de analizar los datos generados por los usuarios para mejorar la experiencia de usuario y la eficiencia de la
plataforma.
\item \textbf{Otros componentes:} Existen otros componentes que pueden ser necesarios en la plataforma: 
componentes de seguridad, monetización y publicidad, gestión de pagos, etc.
\end{itemize}






