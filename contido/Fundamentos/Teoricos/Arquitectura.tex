\subsection{Arquitectura}
\label{subsec:fundamentos_teoricos_arquitectura}

La arquitectura de software se refiere a los modelos y estándares que sirven de base para el diseño e 
implementación de sistemas de software. Define la estructura, el funcionamiento y la interacción de los 
componentes de software \cite{ArqSoftware}.

El diseño de la arquitectura de un sistema de software para una aplicación OTT es crucial para garantizar 
su correcto funcionamiento, así como una entrega eficiente, escalable y confiable de los contenidos. 
Existen varias opciones para diseñar la arquitectura de una aplicación de estas características, como 
la Arquitectura Monolítica o la Arquitectura Orientada a Servicios (SOA). En este caso, se ha optado por 
una arquitectura basada en microservicios.

\subsubsection{Arquitectura basada en microservicios}
\label{subsec:fundamentos_teoricos_arquitectura_microservicios}

La arquitectura basada en microservicios es un enfoque de diseño de aplicaciones que consiste en un conjunto 
de pequeños servicios, los cuales se ejecutan de forma independiente y se comunican mediante mecanismos 
ligeros (como una API de recursos HTTP, como en este proyecto) \cite{Microservices}. Un microservicio se 
encarga de una única funcionalidad de la aplicación, lo que le permite operar de manera autónoma y comunicarse 
con otros microservicios según sea necesario.

En esta arquitectura, las funcionalidades de la plataforma OTT se distribuyen en diferentes microservicios, 
lo que permite que cada uno sea desarrollado, desplegado y escalado de forma independiente. Esto facilita 
la evolución de la plataforma, ya que cada microservicio puede ser mejorado sin afectar al resto del sistema. 
Además, una buena arquitectura de microservicios mejora la tolerancia a fallos; si uno falla, el resto de la 
plataforma debería continuar funcionando correctamente o verse mínimamente afectado.

Ejemplos de microservicios comunes en una plataforma OTT son:
\begin{itemize}
\item \textbf{Gestión de usuarios:} Encargado de la gestión de usuarios, incluyendo registro, autenticación y autorización.
\item \textbf{Gestión de contenidos:} Responsable de la gestión y almacenamiento de los contenidos, así como de los metadatos y archivos multimedia.
\item \textbf{Orquestador:} Coordina las peticiones de los usuarios a los diferentes microservicios de la plataforma.
\item \textbf{Recomendaciones:} Genera recomendaciones personalizadas para los usuarios.
\end{itemize}

\subsubsection{Componentes de la arquitectura}
\label{subsec:fundamentos_teoricos_arquitectura_componentes}

Además de los microservicios, una plataforma OTT incluye varios componentes clave que conforman la infraestructura 
técnica esencial para su funcionamiento. Estos componentes soportan las operaciones críticas, desde la ingesta y 
distribución de contenido hasta la experiencia de usuario final. Algunos de los componentes más importantes son:

\begin{itemize}
\item \textbf{CDN (Content Delivery Network):} Red de distribución de contenidos que permite la entrega rápida y eficiente de contenidos multimedia a los usuarios finales. La CDN almacena copias de los contenidos en servidores distribuidos geográficamente, reduciendo la latencia y mejorando la velocidad de carga.
\item \textbf{Ingesta y gestión de contenidos:} Componente que recibe, procesa y gestiona todos los contenidos que estarán disponibles en la plataforma. Permite a los administradores subir, editar, etiquetar y organizar los contenidos y sus metadatos para su distribución.
\item \textbf{Gestión de usuarios:} Responsable de todas las funciones relacionadas con los usuarios, incluyendo registro, autenticación, autorización, gestión de perfiles y preferencias.
\item \textbf{Interfaz de usuario:} Es la parte visible de la plataforma, donde los usuarios interactúan con ella. Recibe la información de los microservicios y la presenta de manera amigable.
\item \textbf{Monitorización y análisis:} Componente encargado de monitorizar el rendimiento de la plataforma y de analizar los datos generados por los usuarios, con el fin de mejorar la experiencia de usuario y la eficiencia del sistema.
\item \textbf{Otros componentes:} Incluye componentes de seguridad, monetización, publicidad, gestión de pagos, entre otros, que pueden ser necesarios según las características de la plataforma.
\end{itemize}
