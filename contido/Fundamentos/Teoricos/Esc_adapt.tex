\subsection{Escalabilidad, adaptabilidad y multiplataforma}
\label{sec:fundamentos_teoricos_esc_adapt}

La escalabilidad, adaptabilidad y multiplataforma son conceptos fundamentales en el diseño y desarrollo de aplicaciones, 
especialmente en caso de estudio de este proyecto, una plataforma OTT con soporte multicliente y multiplataforma.

\subsubsection{Escalabilidad}
\label{subsec:fundamentos_teoricos_esc_adapt_escalabilidad}

La escalabilidad es la capacidad de un sistema para crecer. En este crecimiento se incluye: el aumento de la capacidad de la
aplicacion para manejar más usuarios, datos y transacciones, adaptarse a las nuevas necesidades de los mercados y tecnologías,
mejorar tanto las capacidades como el rendimiento del sistema y ampliar el catálogo de servicios y funcionalidades.

La plataforma debe ser capaz de escalar tanto horizontal como verticalmente. La escalabilidad horizontal se refiere a la capacidad
de añadir más servidores a la infraestructura para distribuir la carga de trabajo y mejorar el rendimiento. Por otro lado, la escalabilidad
vertical implica la mejora de los recursos de un servidor existente, como la memoria, el almacenamiento o la potencia de procesamiento.

La plataforma también debe estar pensada para futuras actualizaciones y mejoras, de modo que pueda adaptarse a los cambios en las
necesidades del negocio y las preferencias de los usuarios sin afectar a la estabilidad y el rendimiento del sistema. La escalabilidad
es esencial para garantizar que la plataforma pueda crecer de manera sostenible y seguir siendo competitiva en un entorno en constante
evolución.

\subsubsection{Adaptabilidad}
\label{subsec:fundamentos_teoricos_esc_adapt_adaptabilidad}

La adaptabilidad es la capacidad de un sistema para ajustarse a diferentes contextos, entornos y dispositivos. En el caso de una
plataforma OTT, la adaptabilidad implica la capacidad de ofrecer una experiencia de usuario consistente y de alta calidad en una
amplia variedad de dispositivos, sistemas operativos y navegadores.

La adaptabilidad es esencial para garantizar que los usuarios puedan acceder al contenido de la plataforma desde cualquier dispositivo
y en cualquier momento, sin importar las limitaciones técnicas o las preferencias personales. La plataforma debe ser capaz de adaptarse
automáticamente a las características de cada dispositivo, como el tamaño de la pantalla, la resolución, la velocidad de conexión y la
capacidad de procesamiento, para ofrecer una experiencia de usuario óptima y satisfactoria.

La adaptabilidad también implica la capacidad de la plataforma para adaptarse a los cambios en el mercado, las tecnologías y las
preferencias de los usuarios. La plataforma debe ser flexible y modular, de modo que pueda integrar nuevas funcionalidades, servicios
y contenidos de manera rápida y sencilla, sin afectar a la estabilidad y el rendimiento del sistema.

\subsubsection{Multiplataforma}
\label{subsec:fundamentos_teoricos_esc_adapt_multiplataforma}

La multiplataforma es la capacidad de un sistema para funcionar en diferentes dispositivos, sistemas operativos y navegadores. En el
caso de una plataforma OTT, la multiplataforma implica la capacidad de ofrecer una experiencia de usuario consistente y de alta calidad
en una amplia variedad de dispositivos, como ordenadores, smartphones, tablets, smart TVs y consolas de videojuegos.

La multiplataforma es esencial para garantizar que los usuarios puedan acceder al contenido de la plataforma desde cualquier dispositivo
y en cualquier momento, sin importar las limitaciones técnicas o las preferencias personales, aumentando así el número de usuarios a los que
se puede llegar y mejorando la visibilidad y la rentabilidad de la plataforma.

La multiplataforma también implica la capacidad de la plataforma para integrarse con otros sistemas y servicios, como redes sociales, sistemas
de pago, motores de recomendación y herramientas de análisis, para ofrecer una experiencia de usuario completa y personalizada, adaptada a las
necesidades y preferencias de cada usuario.

\subsubsection{Conclusiones}
\label{subsec:fundamentos_teoricos_esc_adapt_conclusiones}

En resumen, la escalabilidad, adaptabilidad y multiplataforma son conceptos fundamentales en el diseño y desarrollo de una plataforma OTT, ya
que permiten garantizar que la plataforma pueda crecer de manera sostenible, adaptarse a los cambios en el mercado y las tecnologías, y ofrecer
una experiencia de usuario consistente y de alta calidad en una amplia variedad de dispositivos y contextos.