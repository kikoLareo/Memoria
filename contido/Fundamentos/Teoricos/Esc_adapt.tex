\subsection{Escalabilidad, adaptabilidad y multiplataforma}
\label{sec:fundamentos_teoricos_esc_adapt}

La escalabilidad, adaptabilidad y multiplataforma son conceptos fundamentales en el diseño y desarrollo de 
aplicaciones, especialmente en el caso de estudio de este proyecto: una plataforma OTT con soporte multicliente 
y multiplataforma y una aplicación de análisis de datos para varias aplicaciones.

\subsubsection{Escalabilidad}
\label{subsec:fundamentos_teoricos_esc_adapt_escalabilidad}

La escalabilidad se refiere a la capacidad de un sistema para crecer y adaptarse a nuevas necesidades, sin 
comprometer su rendimiento o estabilidad. En este proyecto, la escalabilidad se centra principalmente en la 
capacidad de la plataforma para ampliar su catálogo de servicios y funcionalidades, adaptándose a las demandas 
de los usuarios y a las necesidades del mercado.

En lugar de enfocarse en la infraestructura de servidores, la escalabilidad aquí implica la posibilidad de 
integrar nuevas funcionalidades de manera modular, permitiendo que la plataforma evolucione y mejore continuamente. 
Esta capacidad de expansión es crucial para mantener la relevancia de la plataforma en un entorno en constante 
cambio, permitiendo la incorporación de nuevas características sin afectar las existentes.

Además, la plataforma debe estar preparada para futuras actualizaciones y mejoras, de modo que pueda adaptarse 
a los cambios en las preferencias de los usuarios y en las tendencias del mercado. La escalabilidad funcional 
es esencial para asegurar que la plataforma pueda crecer de manera sostenible y seguir siendo competitiva, al 
tiempo que ofrece una experiencia de usuario coherente y de alta calidad.


\subsubsection{Adaptabilidad}
\label{subsec:fundamentos_teoricos_esc_adapt_adaptabilidad}

La adaptabilidad es la capacidad de un sistema para ajustarse a diferentes contextos, entornos y dispositivos. 
En una plataforma OTT, la adaptabilidad implica ofrecer una experiencia de usuario consistente y de alta calidad 
en una amplia variedad de dispositivos, sistemas operativos y navegadores.

Es crucial que los usuarios puedan acceder al contenido de la plataforma desde cualquier dispositivo y en cualquier 
momento, sin importar las limitaciones técnicas o las preferencias personales. La plataforma debe ser capaz de 
adaptarse automáticamente a las características de cada dispositivo, como el tamaño de la pantalla, la resolución, 
la velocidad de conexión y la capacidad de procesamiento, para ofrecer una experiencia de usuario óptima.

Además, la adaptabilidad implica la capacidad de la plataforma para ajustarse a los cambios en el mercado, las 
tecnologías y las preferencias de los usuarios. La plataforma debe ser flexible y modular, permitiendo la integración 
de nuevas funcionalidades, servicios y contenidos de manera rápida y sencilla, sin comprometer la estabilidad y 
el rendimiento del sistema.

\subsubsection{Multiplataforma}
\label{subsec:fundamentos_teoricos_esc_adapt_multiplataforma}

La multiplataforma es la capacidad de un sistema para funcionar en diferentes dispositivos, sistemas operativos 
y navegadores. En una plataforma OTT, la multiplataforma implica ofrecer una experiencia de usuario consistente 
y de alta calidad en una amplia variedad de dispositivos, como ordenadores, smartphones, tablets, smart TVs y 
consolas de videojuegos.

Esta característica es fundamental para garantizar que los usuarios puedan acceder al contenido de la plataforma 
desde cualquier dispositivo y en cualquier momento, aumentando así la cantidad de usuarios a los que se puede llegar 
y mejorando la visibilidad y rentabilidad de la plataforma.

Además, la multiplataforma incluye la capacidad de la plataforma para integrarse con otros sistemas y servicios, 
como redes sociales, sistemas de pago, motores de recomendación y herramientas de análisis, para ofrecer una 
experiencia de usuario completa y personalizada, adaptada a las necesidades y preferencias de cada usuario.

\subsubsection{Conclusiones}
\label{subsec:fundamentos_teoricos_esc_adapt_conclusiones}

En resumen, la escalabilidad, adaptabilidad y multiplataforma son conceptos fundamentales en el diseño y 
desarrollo de una plataforma OTT, ya que permiten garantizar que la plataforma pueda crecer de manera sostenible, 
adaptarse a los cambios en el mercado y en las tecnologías, y ofrecer una experiencia de usuario consistente y de 
alta calidad en una amplia variedad de dispositivos y contextos.
