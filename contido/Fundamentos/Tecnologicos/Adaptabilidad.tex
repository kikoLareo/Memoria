\subsection{Desarrollo multiplataforma}
\label{subsec:adaptabilidad_multiplataforma}

El desarrollo de una plataforma OTT que funcione de manera eficiente en múltiples dispositivos requiere un 
enfoque que combine la flexibilidad de las tecnologías web con herramientas específicas para adaptar la aplicación 
a distintos sistemas operativos y dispositivos. La aplicación se ha desarrollado utilizando JavaScript (JS), HTML 
y CSS, cuyas ventajas y desventajas se han discutido en la sección \ref{sec:TechWeb}.

\subsubsection{Adaptación a Web}
\label{subsubsec:adaptabilidad_web}

Gracias a la utilización de JavaScript, HTML y CSS, la adaptación de la aplicación a la web es sencilla y 
directa. La aplicación se ha desarrollado siguiendo los principios de diseño responsivo, lo que permite que 
la interfaz se ajuste automáticamente al tamaño de la pantalla del dispositivo en el que se visualiza. Además, 
la aplicación se ha probado en los principales navegadores web (Chrome, Firefox, Safari, Edge) para garantizar 
su compatibilidad y correcto funcionamiento en distintos entornos.

\subsubsection{Adaptación a Smart TVs}
\label{subsubsec:adaptabilidad_smart_tvs}

La adaptación de la aplicación a televisores inteligentes es un proceso más complejo, ya que estos dispositivos 
suelen tener sistemas operativos propietarios con limitaciones y peculiaridades específicas. Cada fabricante de 
televisores inteligentes tiene su propio sistema operativo y plataforma de desarrollo, así como su propia manera 
de empaquetar, ejecutar y distribuir aplicaciones. Por ello, la adaptación de la aplicación a televisores 
inteligentes requiere un estudio y enfoque específicos para cada plataforma.

\subsubsection{Tizen}
\label{subsubsec:adaptabilidad_tizen}

Tizen \cite{Tizen} es un sistema operativo de código abierto desarrollado por Samsung Electronics y la Linux 
Foundation. Tizen utiliza tecnologías web como HTML, CSS y JavaScript para el desarrollo de aplicaciones, lo que 
facilita la adaptación de la aplicación a este sistema operativo. Las peculiaridades de Tizen, como su sistema de 
empaquetado y distribución de aplicaciones, se han tenido en cuenta durante el desarrollo para garantizar su 
compatibilidad y correcto funcionamiento. La aplicación ha sido probada utilizando la plataforma Tizen Studio, 
que permite emular el funcionamiento en un televisor Samsung con Tizen OS e instalar y ejecutar la aplicación 
en un dispositivo real. El formato de empaquetado para Tizen es un archivo \texttt{.wgt}, que contiene todos 
los recursos y archivos necesarios para la ejecución y distribución de la aplicación en Tizen OS.

\subsubsection{WebOS}
\label{subsubsec:adaptabilidad_webos}

WebOS \cite{WebOS} es un sistema operativo de código abierto utilizado por LG Electronics para sus televisores 
inteligentes. WebOS también utiliza tecnologías web como HTML, CSS y JavaScript para el desarrollo de aplicaciones, 
permitiendo una fácil adaptación a este sistema operativo. LG proporciona un CLI (Command Line Interface) para la 
creación, empaquetado y distribución de aplicaciones para WebOS. Además, está disponible una extensión para Visual 
Studio Code que facilita aún más el desarrollo. La aplicación ha sido probada en un televisor LG con WebOS 
utilizando ambas herramientas, garantizando su compatibilidad y funcionamiento. El formato de empaquetado para 
WebOS es un archivo \texttt{.ipk}.

\subsubsection{Android TV}
\label{subsubsec:adaptabilidad_android_tv}

Android TV \cite{AndroidTV} es una plataforma de televisión inteligente desarrollada por Google, basada en el 
sistema operativo Android. La adaptación de la aplicación a este sistema operativo varía en comparación con las 
anteriores, ya que Android no permite el uso nativo de tecnologías como JavaScript, HTML y CSS. Por ello, es 
necesario un proceso de conversión para hacer la aplicación compatible con Android TV.

\paragraph{Conversión a Android TV: Cordova}
\label{par:adaptabilidad_android_tv_conversion}

Para adaptar la aplicación a Android TV, se ha utilizado la herramienta Cordova \cite{Cordova}, que permite utilizar 
tecnologías estándar en lugar de las tecnologías nativas de determinadas plataformas, como en el caso de Android.

Apache Cordova es una plataforma de desarrollo móvil que permite a los desarrolladores crear aplicaciones 
multiplataforma utilizando tecnologías web estándar como HTML5, CSS3 y JavaScript. Una de las principales ventajas 
de Cordova es que permite acceder a las funcionalidades nativas del dispositivo a través de APIs, facilitando la 
creación de aplicaciones que pueden ejecutarse en múltiples plataformas sin necesidad de escribir código específico 
para cada una. Esto no solo reduce el tiempo de desarrollo, sino que también simplifica el mantenimiento y la 
actualización de las aplicaciones.

Cordova cuenta con una amplia comunidad de desarrolladores y una gran cantidad de plugins que extienden sus capacidades. 
Estos plugins permiten integrar funcionalidades adicionales como el acceso a la cámara, el almacenamiento local, 
las notificaciones push, y más. La flexibilidad y extensibilidad de Cordova lo convierten en una opción popular 
para el desarrollo de aplicaciones móviles híbridas, especialmente en proyectos donde el tiempo y los recursos son 
limitados. La capacidad de Cordova para adaptarse a diferentes plataformas, incluyendo Android TV, demuestra su 
versatilidad y potencia como herramienta de desarrollo.

La aplicación ha sido probada en un dispositivo Android TV, garantizando su compatibilidad y funcionamiento. El 
formato de empaquetado para Android TV es un archivo \texttt{.apk}.

\subsubsection{Empaquetado y ejecución de la aplicación}
\label{subsubsec:adaptabilidad_empaquetado_ejecucion}

Para las aplicaciones de TV, se ha creado una plantilla que contiene los archivos necesarios para la ejecución 
de la aplicación en los distintos sistemas operativos de televisores inteligentes. La estructura de la aplicación 
permite que, en el momento de empaquetar para prueba o distribución, sea sencillo añadir el código a la plantilla 
y ejecutar las herramientas necesarias para cada sistema operativo.
