\subsection{Desarrollo multiplataforma}
\label{subsec:adaptabilidad_multiplataforma}

El desarrollo de una plataforma OTT que funcione de manera eficiente en múltiples dispositivos 
requiere un enfoque que combine la flexibilidad de las tecnologías web con herramientas específicas 
para adaptar la aplicación a distintos sistemas operativos y dispositivos. La aplicación se ha 
desarrollado utilizando JavaScript (JS), HTML, y CSS, tecnologías que permiten una fácil adaptación 
a la web y una base sólida para su conversión a otras plataformas, como televisores inteligentes y Android TV.

\subsubsection{Adaptación a Web}
\label{subsubsec:adaptabilidad_web}

Gracias a la utilización de JavaScript, HTML y CSS, la adaptación de la aplicación a la web es
sencilla y directa. La aplicación se ha desarrollado siguiendo los principios de diseño responsivo,
lo que permite que la interfaz se adapte automáticamente al tamaño de la pantalla del dispositivo
en el que se visualiza. Además, la aplicación se ha probado en los principales navegadores web
(Chrome, Firefox, Safari, Edge) para garantizar su compatibilidad y funcionamiento en distintos
entornos.


\subsubsection{Adaptación a Smart TVs}
\label{subsubsec:adaptabilidad_smart_tvs}

La adaptación de la aplicación a televisores inteligentes es un proceso más complejo, ya que
estos dispositivos suelen tener sistemas operativos propietarios y limitaciones y pecualiariades
específicas. Cada fabricante de televisores inteligentes tiene su propio sistema operativo y
plataforma de desarrollo, así como su propia manera de empequetar, ejecutar y distribuir
aplicaciones. Por ello, la adaptación de la aplicación a televisores inteligentes requiere un
estudio y enfoque específico para cada plataforma. 

\subsubsection{Tizen}
\label{subsubsec:adaptabilidad_tizen}

Tizen \cite{Tizen} es un sistema operativo de código abierto desarrollado por Samsung Electronics y la Linux
Foundation. Tizen utiliza tecnologías web como HTML, CSS y JavaScript para el desarrollo de
aplicaciones, lo que facilita la adaptación de la aplicación a este sistema operativo. Las pecualiariades
de Tizen, como su sistema de empaquetado y distribución de aplicaciones, se han tenido en cuenta
durante el desarrollo de la aplicación para garantizar su compatibilidad y funcionamiento. De este modo,
la aplicación ha sido probada utilizando la plataforma Tizen studio la cual nos permite emular el
funcionamiento de la aplicación en un televisor Samsung con Tizen OS y, más importante y confiable, 
instalar y ejecutar la aplicación en un televisor Samsung. El formato de empaquetado de la aplicación
para Tizen es un archivo .wgt, el cual contiene todos los recursos y archivos necesarios para la
ejecución y distribución de la .

\subsubsection{WebOS}
\label{subsubsec:adaptabilidad_webos}

WebOS \cite{WebOS} es un sistema operativo de código abierto utilizado por LG Electronics para sus televisores
inteligentes. WebOS utiliza tecnologías web como HTML, CSS y JavaScript para el desarrollo de
aplicaciones, permitiendo la adaptación de la aplicación a este sistema operativo. Lg proporciona
un CLI (Command Line Interface) para la ejecución de las funcionalidades necesarias para la creación, 
empaquetado y distribución de aplicaciones para WebOS. Además, también está disponible una extensión para
Visual Studio Code que facilita aún más el desarrollo. La aplicación ha sido probada en un televisor LG
con WebOS a través de ambas herramientas, garantizando su compatibilidad y funcionamiento. El formato de
empaquetado de la aplicación para WebOS es un archivo .ipk.

\subsubsection{Android TV}
\label{subsubsec:adaptabilidad_android_tv}

Android TV \cite{AndroidTV} es una plataforma de televisión inteligente desarrollada por Google. Está basado
en el sistema operativo Android. La adaptación de la aplicación a este sistema operativo varia en comparación
con las anteriores. Android no permite el uso de las tecnologías de desarrollo js, html y css de manera nativa
por lo que se requiere de un proceso de conversión de la aplicación a un formato compatible con Android TV.

Para ello se ha utilizado la herramienta Cordova \cite{Cordova}, que permite la utilización de las tecnologías
Css, Html y JavaScript en lugar de las tecnologías nativas de determinadas plataformas como Ios y Android.
Cordova permite la creación de aplicaciones híbridas, es decir, aplicaciones que combinan tecnologías web y
nativas. La aplicación se ha convertido a un proyecto Cordova y se ha añadido la plataforma de Android TV.

La aplicación ha sido probada en un dispositivo Android TV, garantizando su compatibilidad y funcionamiento.
El formato de empaquetado de la aplicación para Android TV es un archivo .apk.


\subsubsection{Empaquetado y ejecución de la aplicación}
\label{subsubsec:adaptabilidad_empaquetado_ejecucion}

En el caso de las aplicaciones de TV lo que se ha hecho es crear una carpeta que sirve de plantilla para
la creación de los archivos necesarios para la ejecución de la aplicación en los distintos sistemas operativos
de las TV. La aplicación se ha ido desarrollando con una estructura que permite que en el momento de querer empaquetar
una aplicación para probar o distribuir un código, sea sencillo añadir ese código a la plantilla y ejecutar con 
las herramientas necesarias para cada sistema operativo.


