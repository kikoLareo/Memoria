\subsection{Arquitectura Software: Comunicación entre componentes y desarrollo multicliente}
\label{subsec:arquitectura_software}

\subsubsection{Comunicación entre componentes}
\label{subsec:arquitectura_software_comunicacion}

Una de las claves de la arquitectura software \ref{subsec:fundamentos_teoricos_arquitectura} es la comunicación 
entre componentes y microservicios \ref{subsec:fundamentos_teoricos_arquitectura_microservicios}. En esta 
aplicación, dicha comunicación se realiza a través de una API REST, que es una interfaz de programación de 
aplicaciones que sigue los principios de diseño del estilo arquitectónico de transferencia de estado 
representacional (REST).

REST es un estilo de arquitectura basado en la comunicación mediante HTTP, caracterizado por su sencillez, 
escalabilidad y adaptabilidad a diversos entornos. En una API REST, los recursos son identificados por URLs 
y se accede a ellos a través de los métodos HTTP estándar (GET, POST, PUT, DELETE). Esto permite que los 
clientes de la API realicen operaciones sobre los recursos de manera sencilla y estandarizada.

En esta aplicación, la API REST facilita la comunicación entre los diferentes microservicios de la 
plataforma OTT, proporcionando a la interfaz los medios para obtener la información necesaria para 
su visualización, y a los microservicios una manera de recibir las peticiones de la interfaz y 
devolver la información correspondiente.

\subsubsection{Desarrollo multicliente}
\label{subsec:arquitectura_software_multicliente}

Una de las ventajas que ofrece la API REST en esta aplicación es la flexibilidad para decidir qué 
información obtener. Dependiendo de los parámetros añadidos a la URL, la API devolverá la información 
solicitada o realizará una acción específica sobre un cliente determinado.

Esta diferenciación permite, a través de la misma API, gestionar múltiples clientes sin necesidad de 
desarrollar nuevas APIs o microservicios para cada caso.
