\subsection{Arquitectura Software: Comunicación entre componentes y desarrollo multicliente}
\label{subsec:arquitectura_software}

\subsubsection{Comunicación entre componentes}
\label{subsec:arquitectura_software_comunicacion}

Una de las claves de la arquitectura software \ref{subsec:fundamentos_teoricos_arquitectura} es la 
comunicación entre componentes y microservicios \ref{subsec:fundamentos_teoricos_arquitectura_microservicios}.
En el caso de esta aplicación esta comunicación se realiza a través de una API REST, que es es una interfaz 
de programación de aplicaciones (API) que se ajusta a los principios de diseño del estilo arquitectónico de 
transferencia de estado representacional (REST). 

REST es un estilo de arquitectura que se basa en la comunicación a través de HTTP, y que se caracteriza por
ser sencillo, escalable y que se puede utilizar en cualquier entorno. En una API REST, los recursos son
identificados por URLs y se acceden a través de los métodos HTTP estándar (GET, POST, PUT, DELETE).
Esto permite que los clientes de la API puedan realizar operaciones sobre los recursos de forma sencilla y
estandarizada.

En el caso de esta aplicación, la API REST se encarga de la comunicación entre los diferentes microservicios
de la plataforma OTT, ofreciendo a la interfaz una forma de obtener la información necesaria para mostrar
en la aplicación, y a los microservicios una forma de recibir las peticiones de la interfaz y devolver la
información necesaria.

\subsubsection{Desarrollo multicliente}
\label{subsec:arquitectura_software_multicliente}

Una de las ventajas que nos proporciona el API REST que utilizamos en esta aplicación es la de tener el poder
de decisión sobre la información que queremos obtener. Así, en función de los parametros que se añadan a la
URL, la API nos devolverá la información solicitada o realizará una acción sobre un cliente en concreto.

Esta diferenciación nos permite, a través de la misma API, trabajar sobre diferentes clientes sin necesidad
de crear una API nueva o un microservicio a medida en cada caso. 