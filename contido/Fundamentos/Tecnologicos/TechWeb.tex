\subsection{Tecnologías Web: JavaScript, HTML y CSS}
\label{sec:TechWeb}

La plataforma OTT está construida sobre la base de tecnologías web fundamentales: JavaScript (JS), 
HTML y CSS. Estas tecnologías permiten el desarrollo de una interfaz de usuario interactiva y
 adaptable, que puede ser fácilmente desplegada en diferentes plataformas, incluyendo la web, 
 smart TVs y dispositivos móviles.

\subparagraph{JavaScript (JS)}
JavaScript es el motor que impulsa la interactividad en la plataforma. Permite manipular el DOM, 
gestionar eventos del usuario, realizar llamadas asíncronas a la API de la CDN para obtener contenido 
dinámico y, sobre todo, implementar la lógica de negocio de la plataforma. JavaScript es un lenguaje de 
programación versátil y flexible, ideal para la creación de aplicaciones web complejas y altamente interactivas.

\subparagraph{HTML}
HTML define la estructura de la interfaz de usuario, organizando el contenido de manera semántica y 
accesible. Su compatibilidad con todos los navegadores web y dispositivos asegura que la plataforma 
pueda ser utilizada en una amplia gama de entornos, desde navegadores de escritorio hasta smart TVs.

\subparagraph{CSS}
CSS es responsable de la presentación visual de la plataforma, permitiendo una estilización coherente 
y atractiva en todas las plataformas soportadas. A través de técnicas de diseño responsivo y el uso 
de transiciones y animaciones, CSS asegura que la interfaz se adapte y ofrezca una experiencia de 
usuario óptima en cualquier dispositivo.

\subsubsection{Ventajas y desventajas de las tecnologías web}
\label{subsec:TechWeb_ventajas_desventajas}

Las tecnologías web como JavaScript, HTML y CSS presentan numerosas ventajas, como su amplia compatibilidad, 
flexibilidad y la posibilidad de desarrollar aplicaciones que pueden ejecutarse en múltiples plataformas sin 
necesidad de cambios significativos. Estas tecnologías son altamente versátiles, lo que permite una rápida 
adaptación a nuevos requisitos y la incorporación de funcionalidades avanzadas de manera eficiente. Además, 
el uso de estándares web asegura que la plataforma sea accesible y usable en una gran variedad de dispositivos y entornos.

Sin embargo, también existen desventajas. El rendimiento de las aplicaciones web puede no igualar al de las 
aplicaciones nativas en ciertos dispositivos, especialmente en entornos con recursos limitados. Además, la 
gestión del comportamiento en múltiples plataformas puede requerir un esfuerzo adicional para asegurar una 
experiencia de usuario coherente.

La elección de estas tecnologías para el desarrollo de la plataforma OTT se justifica por su capacidad de 
ofrecer una solución multiplataforma eficiente y adaptable, permitiendo un desarrollo más rápido y una 
mayor flexibilidad en la personalización y mantenimiento de la plataforma, a pesar de las limitaciones 
inherentes a las tecnologías web.
