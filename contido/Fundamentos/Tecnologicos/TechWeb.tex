
\subsection{Tecnologías Web: JavaScript, HTML y CSS}
\label{sec:TechWeb}

La plataforma OTT está construida sobre la base de tecnologías web fundamentales: JavaScript (JS), 
HTML, y CSS. Estas tecnologías permiten el desarrollo de una interfaz de usuario interactiva y 
adaptable, que puede ser fácilmente desplegada en diferentes plataformas, incluyendo web, smart TVs 
y dispositivos móviles.

\subparagraph{JavaScript (JS)}
JavaScript es el motor que impulsa la interactividad en la plataforma. Permite manipular el DOM, 
gestionar eventos del usuario,  realizar llamadas asíncronas a la API de la CDN para obtener contenido 
dinámico y, sobre todo, implementar la lógica de negocio de la plataforma. JavaScript es un lenguaje
de programación versátil y flexible, que permite la creación de aplicaciones web complejas y
altamente interactivas.

\subparagraph{HTML}
HTML define la estructura de la interfaz de usuario, organizando el contenido de manera semántica y 
accesible. Su compatibilidad con todos los navegadores web y dispositivos asegura que la plataforma 
pueda ser utilizada en una amplia gama de entornos, desde navegadores de escritorio hasta smart TVs.

\subparagraph{CSS}
CSS es responsable de la presentación visual de la plataforma, permitiendo una estilización coherente 
y atractiva en todas las plataformas soportadas. A través de técnicas de diseño responsivo y el uso de 
transiciones y animaciones, CSS asegura que la interfaz se adapte y ofrezca una experiencia de usuario 
óptima en cualquier dispositivo.