\subsection{Rendimiento y Optimización}
\label{subsec:rendimiento_optimizacion}

El rendimiento es un aspecto crítico en el desarrollo de la plataforma OTT, ya que influye directamente 
en la experiencia del usuario. Para garantizar un rendimiento óptimo, se han implementado diversas estrategias 
de optimización tanto en el frontend como en las interacciones entre el frontend y el backend.

En el frontend, se han optimizado los movimientos y transiciones utilizando técnicas avanzadas de CSS y se han 
creado funcionalidades a través de JavaScript para asegurar una experiencia de usuario fluida y agradable. Estas 
optimizaciones permiten que las animaciones, transiciones y otros efectos visuales se ejecuten de manera eficiente, 
mejorando la percepción de velocidad y la interactividad de la aplicación. Además, se ha trabajado en la estructura 
HTML y CSS para asegurar una carga eficiente de los recursos visuales, evitando bloqueos o ralentizaciones innecesarias.

Otra área clave de optimización ha sido la gestión eficiente de las llamadas a la API y el manejo de los datos 
recibidos. Se han implementado técnicas de concurrencia y paralelismo para acelerar las operaciones de red y 
minimizar el tiempo de espera al cargar o actualizar contenido dinámico. Estas optimizaciones aseguran que la 
interfaz de usuario sea ágil y responsiva, incluso en entornos con conexiones de red variables.

En cuanto al backend, aunque no es el enfoque principal de este proyecto, se han mencionado optimizaciones 
generales como el uso de caching y la optimización de consultas a las bases de datos para mejorar la eficiencia. 
Sin embargo, estas mejoras en el backend no están directamente relacionadas con el trabajo desarrollado en este proyecto.


\subsection{Gestión de Versiones (Git)}
\label{subsec:gestion_versiones}

La gestión de versiones en este proyecto se ha llevado a cabo utilizando Git, un sistema de control de versiones 
distribuido que permite a los desarrolladores trabajar de manera colaborativa y gestionar el historial de cambios 
en el código fuente. Git facilita la creación de ramas para desarrollar nuevas funcionalidades o realizar correcciones 
sin afectar al código principal.

El uso de Git también permite un control preciso sobre las versiones del software, lo que es esencial para mantener 
la estabilidad de la plataforma durante las actualizaciones y el despliegue de nuevas funcionalidades. 

\subsection{Uso de APIs}
\label{subsec:uso_apis}

El uso de APIs (Interfaz de Programación de Aplicaciones) es fundamental en la arquitectura de la plataforma OTT. 
Las APIs permiten la comunicación eficiente entre los diferentes componentes de la aplicación, facilitando la 
integración con servicios externos y el intercambio de datos entre microservicios.

En este proyecto, la API REST es la principal herramienta de comunicación, permitiendo a los clientes acceder a 
los recursos de la plataforma a través de métodos HTTP estándar. Las APIs también juegan un papel crucial en la 
integración de funcionalidades como la autenticación de usuarios, la gestión de contenidos y la recopilación de 
datos analíticos. La implementación de estas APIs ha sido diseñada para ser escalable y flexible, permitiendo la 
fácil incorporación de nuevas funcionalidades y servicios en el futuro.

No solo se han utilizado APIs para la comunicación entre los distintos componentes de la plataforma, sino que
también se han integrado APIs de terceros para enriquecer la experiencia del usuario. Por ejemplo, se han
incorporado APIs para la obtención de los datos de las aplicaciones (Matomo) o el almacenamiento de información 
y para la generación de análisis sobre gráficas (MongoDb Y OpenAi).