\subsection{Adaptación a los diferentes sistemas operativos de televisión}
\label{sec:adaptacion}

El mercado de las televisiones inteligentes ha experimentado un gran crecimiento en los últimos años, lo que ha ampliado 
las opciones de aplicaciones y servicios disponibles para los usuarios. Sin embargo, este mercado es relativamente nuevo y 
se le exigen prestaciones similares a las de los dispositivos móviles u ordenadores personales. Cumplir con esas exigencias 
no es sencillo debido a varias razones: la capacidad de procesamiento de las televisiones, el soporte de los sistemas operativos 
a las tecnologías web y la diversidad de sistemas operativos disponibles.

\paragraph{Capacidad de procesamiento de las televisiones}
En los últimos años, la capacidad de procesamiento de las televisiones ha mejorado significativamente. Sin embargo, 
aún no es suficiente para ofrecer una experiencia de usuario comparable a la de un ordenador personal o un dispositivo móvil. 
Además, las tecnologías web son cada vez más complejas y exigentes, lo que hace que las televisiones no siempre puedan 
soportarlas. Al comparar la capacidad de procesamiento de una televisión de alta gama con la de un ordenador o un móvil, 
aunque las televisiones han mejorado, siguen estando por detrás de los dispositivos de consumo masivo en términos de rendimiento.

Las especificaciones de una televisión de alta gama actual (3-4 GB de RAM, procesador de 4 núcleos) son comparables a las de 
dispositivos móviles de gama media-alta de hace algunos años, que hoy en día ya cuentan con 6-8 GB de RAM y procesadores 
de 8 núcleos. De manera similar, los ordenadores más básicos ahora ofrecen entre 8 y 16 GB de RAM.

\paragraph{Soporte de los sistemas operativos a las tecnologías web}
Este problema está parcialmente relacionado con el anterior. Los sistemas operativos de las televisiones inteligentes no están 
tan adaptados a las tecnologías web como en otros dispositivos. Esto puede deberse tanto a la capacidad de procesamiento limitada 
como al hecho de que las televisiones entraron relativamente tarde en el mercado de los dispositivos inteligentes, lo que ha 
retrasado el desarrollo de aplicaciones y servicios optimizados para estos sistemas.

\paragraph{Diversidad de sistemas operativos}
Un problema adicional en el mercado de las televisiones inteligentes es la diversidad de sistemas operativos disponibles. Aunque en 
el mercado de los dispositivos móviles existen varios sistemas operativos, la mayoría de los móviles utiliza Android o iOS, lo que 
facilita el trabajo de los desarrolladores al crear aplicaciones. En cambio, la diversidad de sistemas operativos en las televisiones 
es mucho mayor, lo que obliga a los desarrolladores a adaptar sus aplicaciones a cada uno de ellos.

\subsubsection{Adaptación a los diferentes sistemas operativos de televisión}
\label{sec:adaptacion}

El mayor desafío de esta aplicación es su adaptación a los diferentes sistemas operativos de televisión. Desde las 
primeras fases, el objetivo principal ha sido asegurar su accesibilidad en televisores, dado que este era el 
objetivo a corto plazo. Sin embargo, siempre se tuvo en cuenta que debía ser una aplicación multiplataforma. 
Aunque la implementación y las pruebas han estado enfocadas en optimizar el rendimiento en televisores, la 
aplicación está diseñada para funcionar en ordenadores y dispositivos móviles. Reactivando ciertas funcionalidades, 
como el click o hover del ratón, la aplicación podría funcionar en web sin mayores problemas. No obstante, para 
dispositivos móviles, las pruebas han sido limitadas y aún se requieren nuevas adaptaciones, ya que su optimización 
es un objetivo a más largo plazo. Mientras tanto, la prioridad sigue siendo asegurar la consistencia en televisores, 
a medida que se prepara su lanzamiento en las tiendas de aplicaciones de los distintos sistemas operativos de televisión.

Adaptar una misma aplicación a partir del mismo código supone ajustarse a las capacidades y limitaciones de cada 
SO. Si un SO no soporta ciertas funcionalidades, lo ideal es buscar una alternativa que funcione en todos 
los dispositivos por igual, y en caso de no ser posible, se deberá buscar una solución específica para ese SO.

\subsubsection{Adaptaciones realizadas}
\label{sec:adaptaciones}

A continuación se detallan las adaptaciones generales realizadas para mantener la consistencia y el correcto
funcionamiento de la aplicación en los diferentes sistemas operativos de televisión.

\paragraph{Interfaz}
La interfaz debe ser consistente en cualquier dispositivo en el que se utilice la aplicación. Para ello, se ha 
utilizado un diseño "responsive" que se adapta a cualquier resolución de pantalla. Esto es necesario si queremos usarla 
en dispositivos de distintas familias, pero también en televisores de distintas marcas y modelos. Un ejemplo de ello son 
las televisiones utilizadas en este proyecto para pruebas: LG y Samsung, con la misma resolución de 1920x1080, y TCL (Android TV), 
con una resolución de 960x540. Aunque todas las televisiones son de gamas similares, las diferencias en resolución son evidentes, 
lo que hace esencial asegurarse de que la aplicación sea consistente en cualquier televisión disponible en el mercado. 

Para lograr esto, todo el diseño y las hojas de estilo CSS utilizan medidas relativas: porcentajes, "vh" y "vw". 
Se evita completamente el uso de medidas absolutas como "px", y también se evita el uso de "rem" y "em" para asegurar que 
el diseño sea coherente en todos los dispositivos.

\paragraph{Soporte de tecnologías web}
El soporte de tecnologías web es un aspecto crucial. Como se explicó en otras secciones, las tecnologías web utilizadas en 
este proyecto son JavaScript, HTML y CSS. Aunque todos los SO utilizados soportan estas tecnologías, no lo hacen completamente. 
Durante el desarrollo, se intentó minimizar el uso de librerías externas para garantizar la mantenibilidad del código y evitar 
problemas de compatibilidad en actualizaciones futuras. Un ejemplo de este desafío fue la eliminación de la funcionalidad 
de DOMParser, que no es compatible con Tizen, lo que obligó a implementar una solución manual para procesar archivos XML.

\paragraph{Navegación}
Un ejemplo de adaptación específica para cada SO es la navegación. Los comandos de los mandos a distancia varían ligeramente 
entre dispositivos, por lo que se creó una adaptación específica para cada uno. Cada SO cuenta con un archivo de configuración 
que traduce los códigos de los botones del mando a un formato común que la aplicación puede interpretar correctamente.

\paragraph{Reproducción de video}
La reproducción de video es una de las funcionalidades más importantes de la aplicación. En el caso de las televisiones inteligentes, 
las opciones de reproductores de video son más limitadas que en las páginas web. Por ejemplo, mientras que WebOS y AndroidTV 
soportan VideoJs, Tizen no, y en su lugar utiliza Shaka Player, que no es compatible con los otros SO. Para solucionar esto, 
la aplicación detecta el SO y ajusta el reproductor de video en consecuencia.

\paragraph{Detección del estado de la red}
La detección del estado de la red también varía entre sistemas operativos. Aunque se intentó unificar esta funcionalidad, 
no fue posible obtener resultados consistentes en todos los casos. En Tizen se utiliza la librería webApis, mientras que en 
AndroidTV y WebOS se emplea la funcionalidad "navigator.connection".

\paragraph{Cerrar la aplicación}
La forma de cerrar la aplicación varía entre sistemas. Se investigó la documentación de cada uno y se implementó 
la lógica necesaria para cerrar correctamente la aplicación según el SO detectado.

\subsubsection{Desafíos encontrados}
\label{sec:desafios}

Durante el desarrollo de la aplicación, surgieron varios desafíos que complicaron su adaptación. A continuación, 
se detallan algunos de los desafíos más importantes y las soluciones aplicadas.

El primer desafío fue la falta de información y soluciones para este tipo de aplicaciones. La mayoría de la información 
disponible proviene de la documentación oficial, que a menudo es limitada y no cubre todos los casos posibles. 
Los foros y comunidades de desarrolladores también son útiles, pero en muchos casos no ofrecen respuestas a 
problemas específicos en el desarrollo para televisores.

\paragraph{Foco automático en AndroidTV}
En las primeras pruebas en AndroidTV, se encontró un problema con el enfoque automático de los elementos en la pantalla. 
Esto provocaba movimientos inesperados en la interfaz al mover el mando. La solución fue desactivar este enfoque 
automático mediante un "listener" que interceptaba el evento de enfoque y lo desactivaba inmediatamente.

\paragraph{Botón de retorno con funcionamiento predefinido en AndroidTV}
Otro problema en AndroidTV fue el funcionamiento del botón de retorno. En la última actualización del SO, el botón 
cerraba automáticamente la aplicación si no había una página anterior en el historial. Como esta aplicación usa 
un diseño de página única, se interceptó el evento de retorno y se anuló su comportamiento predeterminado.

\paragraph{Splash screen en AndroidTV}
Finalmente, se encontró un problema con el "splash screen" predeterminado de Cordova en AndroidTV, que no se podía 
desactivar. Aunque se logró cambiar la imagen mostrada, el formato aún presenta problemas. Se sigue buscando una 
solución definitiva para este problema.
