\subsection{Adaptación a los diferentes sistemas operativos de televisión}
\label{sec:adaptacion}

El mercado de las televisiones inteligentes está sufriendo en los últimos años un gran crecimiento, lo que provoca 
que las opciones de aplicaciones y servicios que se pueden ofrecer a los usuarios sean cada vez mayores. Sin embargo,
es un mercado relativamente nuevo y al que se le está exigiendo prestaciones similares a las que se pueden encontrar
en los dispositivos móviles o ordenadores personales. Sin embargo, cumplir esas exigencias no es sencillo debido a 
varias razones: la capacidad de procesamiento de las televisiones, el soporte de los sistemas operativos a las  
tecnologías web y la diversidad de sistemas operativos que se pueden encontrar en el mercado.

\paragraph{Capacidad de procesamiento de las televisiones}
Aunque la mejoría y el aumento de la capacidad de procesamiento en los últimos años ha sido más que notable, no es 
suficiente para poder ofrecer una experiencia de usuario similar a la que se puede encontrar en un ordenador personal
 o un dispositivo móvil. Tampoco ayuda que las tecnologías web sean cada vez más pesadas y complejas y que el resto de 
 dispositivos sí que puedan soportarlas. Si comparamos la capacidad de procesamiento de una televisión con la de un 
 ordenador personal o de un dispositivo móvil, podemos ver que una televisión de alta gama nueva tiene una capacidad 
 de procesamiento que, aunque impresionante para su categoría, aún se queda corta en comparación con otros dispositivos 
 de consumo masivo.

 Las características que ofrece una televisión de alta gama nueva (3/4 GB de RAM, procesador de 4 núcleos...) son las 
 mismas características que ofrecían hace ya unos años los dispositivos móviles de gama media-alta (hoy cuentan
 con 6/8 GB de RAM y procesadores de 8 núcleos) y los ordenadores más básicos (hoy no bajan de 8/16 GB de RAM). 

\paragraph{Soporte de los sistemas operativos a las tecnologías web}
Este problema en parte está relacionado con el anterior. Los sistemas operativos de las televisiones inteligentes
no están tan adaptados a las tecnologías web como otros casos. Esto puede ser debido a varios motivos, como 
la capacidad de procesamiento que se comentaba o la entrada tardía de las televisiones en el mercado de los
dispositivos inteligentes y por lo tanto a la oferta de aplicaciones y servicios. 

\paragraph{Diversidad de sistemas operativos}
Otro problema que se encuentra en el mercado de las televisiones inteligentes es la diversidad de sistemas operativos
que se pueden encontrar. Aunque en el mercado de los dispositivos móviles también se pueden encontrar varios sistemas
operativos, la mayoría de los dispositivos móviles cuentan con Android o iOS, lo que facilita la tarea de los
desarrolladores a la hora de crear aplicaciones y servicios. En el caso de las televisiones, la diversidad de sistemas
operativos es mucho mayor, lo que provoca que los desarrolladores tengan que adaptar sus aplicaciones y servicios a
cada uno de los sistemas operativos que se encuentran en el mercado.

\subsubsection{Adaptación a los diferentes sistemas operativos de televisión}
\label{sec:adaptacion}

El mayor desafío de esta aplicación es su adaptación a los diferentes sistemas operativos de televisión. Desde las 
primeras fases, el objetivo principal ha sido asegurar su accesibilidad en televisores, dado que este era el 
objetivo a corto plazo. Sin embargo, siempre se tuvo en cuenta que debía ser una aplicación multiplataforma. 
Aunque la implementación y las pruebas han estado enfocadas en optimizar el rendimiento en televisores, la 
aplicación está diseñada para funcionar en ordenadores y dispositivos móviles. Reactivando ciertas funcionalidades, 
como el click o hover del ratón, la aplicación podría funcionar en web sin mayores problemas. No obstante, para 
dispositivos móviles, las pruebas han sido limitadas y aún se requieren nuevas adaptaciones, ya que su optimización 
es un objetivo a más largo plazo. Mientras tanto, la prioridad sigue siendo asegurar la consistencia en televisores, 
a medida que se prepara su lanzamiento en las tiendas de aplicaciones de los distintos sistemas operativos de televisión.

Adaptar una misma aplicación a partir del mismo código supone adaptarse a las capacidades y limitaciones de cada
SO. Si un SO no soporta ciertas funcionalidades, lo ideal es buscar una alternativa conjunta que funcione en todos
los dispositivos por igual, y en caso de no ser posible, se deberá buscar una solución específica para ese SO.

\subsubsection{Adaptaciones realizadas}
\label{sec:adaptaciones}

A continuación se detallan las adaptaciones generales realizadas para mantener la consistencia y el correcto
funcionamiento de la aplicación en los diferentes sistemas operativos de televisión.

\paragraph{Interfaz}
La interfaz debe ser consistente en cualquier dispositivo en el que se utilice la aplicación. Para ello, se ha 
utilizado un diseño "responsive" que se adapte a cualquier resolución de pantalla. Esto es necesario si queremos utilizarlo
en dispositivos de distintas familias, pero también en televisores de distintas marcas y modelos. Un ejemplo de ello son 
las televisiones que se están utilizando en este proyecto como televisores de pruebas. La televisión LG y la televisión Samsung
utilizadas tienen la misma resolución, 1920x1080, mientras que la TCL (Android TV) tiene una resolución de 960x540. Las 
tres televisiones son nuevas y de gamas similares, pero la resolución de la TCL, que es la que mejor resolución de 
reproducción de video tiene, es la mitad que las otras dos. Y esto son 3 televisiones que se toman de ejemplo, pero 
si se quiere llegar al mayor número de usuarios posible, hay que asegurarse que la aplicación es consistente en cualquier
televisión que se pueda encontrar en el mercado. 

Para conseguir esto, todo el diseño y todas las hojas de estilo de CSS utilizan medidas relativas: porcentajes, "vh" y "vw". 
Se evita a toda costa el uso de medidas absolutas, como "px" (no se utiliza nunca en este código), y también se ha 
evitado utilizar "rem" y "em" ya que con diferencias tan grandes no se puede asegurar que el diseño sea consistente en
todos los dispositivos.

\paragraph{Soporte de tecnologías web}
El soporte de tecnologías web es un punto importante a tener en cuenta. Como ya se explicó en otras secciones, las
tecnologías web utilizadas para este proyecto son JavaScript, HTML y CSS. Estas tecnologías son soportadas por todos
los SO utilizados, pero no al completo. A lo largo del desarrollo se ha intentado utilizar las minimas librerias posibles
para asegurar la mantenibilidad del código y problemas futuros con las actualizaciones de las librerias. Pero estos no 
son los motivos por los que no se han utilizado librerias, existen soluciones a esos problemas en caso de necesitar las librerias.
El caso es que estas librerias pueden no ser compartibles con los So utilizados y, ante el desconocimiento en algunos casos
y la comprobación de que on funcionan en otros, se ha optado por no utilizarlas. El día que se esta escribiendo este documento sin 
ir más lejos, se ha tenido que elimnar el uso de la funcionalidad de DOMParser, ya que no es soportada por Tizen, y se tuvo
que implementar a mano una solución para poder leer los archivos XML que se reciben con la información de los directos.

\paragraph{Navegación}
Este es un ejemplo de adaptación especifica para cada SO. Los comandos que recibe la aplicación cuando se pulsa un botón
del mando no son iguales en todos los dispositivos, no varian mucho, pero hay diferencias y se ha optado por hacer una
adaptación específica para cada SO. Para cada SO existe un archivo de configuración con los códigos de los botones del mando
y estos se traducen a un formato común que entiende la aplicación para realizar las acciones correspondientes.

\paragraph{Reproducción de video}
La reproducción de video es una de las partes más importantes de la aplicación y por lo tanto hay que asegurar que
funciona correctamente, sino la aplicación no tendría sentido. A diferencia de las páginas web, en las que hay mayor 
libertad de elección del reproductor de video, en el caso de las televisiones inteligentes estas opciones son más limitadas.
En nuestra aplicación en concreto, los SO utilizados hasta el momento no utilizan el mismo reproductor, y es que a diferencia 
de WebOs y AndroidTv, Tizen no soporta el reproductor VideoJs y por el contrario utiliza Shaka Player, el cual no es soportado
por los otros dos SO. Este es otro ejemplo de adaptación específica ya que para solucionar este problema la solución pasa por 
detectar el SO y crear un reproductor de video u otro. 

\paragraph{Detección del estado de la red}
La detección del estado de la red es una parte importante de la aplicación, ya que si no hay conexión a internet no se
puede acceder a los contenidos. En este caso, la detección del estado de la red es una adaptación específica para cada SO.
Tras intentar utilizar las mismas funcionalidades para todos los SO, la detección del estado de red no era consistente
en todos los casos, por lo que hubo que buscar una solución específica para cada SO. En el caso de Tizen se utiliza la 
libreria webApis, en el caso de AndroidTv y webOs se permite utilizar las funcionalidades de navigator y más concretamente
navigator.connection.

\paragraph{Cerrar la aplicación}
Como era de esperar, la forma de cerrar la aplicación también es diferente en cada caso. Para ello, se buscó información
en la documentación correspondiente y se desarrolló la aplicación para detectar el SO y cerrar la aplicación de la forma
correcta en cada caso.


\subsubsection{Desafios encontrados}
\label{sec:desafios}

A lo largo del desarrollo de la aplicación se han encontrado varios desafíos que han complicado la adaptación de la
aplicación. A continuación se detallan algunos de los desafíos más importantes/raros encontrados y cómo se han solucionado.

El primero de los desafios encontrados fue la falta de información y de soluciones sobre problemas en este tipo de aplicaciones.
La mayoría de la información que se encuentra en internet sobre aplicaciones para televisores inteligentes es proviene de
las documentación oficial que en muchos casos es escasa y no cubre todos los casos posibles. Los foros y comunidades de
desarrolladores son una buena fuente de información, pero en muchos casos no se encuentran soluciones a los problemas
que se pueden encontrar en el desarrollo en estos dispositivos.

\paragraph{Foco automático en AndroidTv}
Cuando se comenzó a probar la aplicación en AndroidTv, se encontró un problema un tanto peculiar. La aplicación 
estaba detectando elementos automáticamente y con cada movimiento del mando no solo realizaba el movimiento esperado,
si no que buscaba otro elemento en el que enfocarse y podia provocar movimiento indeseados al desplazar la pantalla 
o mover otros elementos. Para solucionar este problema habia que desactivar ese foco. Se buscaron varias soluciones: 
utilizar la propiedad "tabindex", probar distintas combinaciones z-index de los elementos, estudiar las configuraciones
internas y las Apis que utiliza Cordova para AndroidTv, pero ninguna de ellas funcionó. La solución fue utilizar 
un "listener" que detectara cuando un elemento obtenia el foco y lo desactivara inmediatamente. En las televisiones
el foco que se utiliza es un foco "artificial" ya que al elemento enfocado se le añade la clase "focused" y no 
se utiliza la propiedad "focus" de HTML. Esta función que desactivaba el foco se ejecutaba en el evento "focusin"
y tuvo que ser ajustada más tarde para permitir el foco en los elementos que lo necesitaban como la caja del buscador.

\paragraph{Botón de retorno con funcionamiento predefinido en AndroidTv}
Otro de los problemas encontrados en AndroidTv fue el funcionamiento del botón de retorno. En la última actualización de 
AndroidTv, el botón de retorno sufrió un cambio en su funcionamiento. Al ser pulsado, si no encontraba ninguna 
página anterior en el historial, cerraba la aplicación. Esto no era el comportamiento esperado, ya que esta aplicación 
usa un diseño de página única, solapando las páginas en lugar de cargar una nueva. Para solucionar este problema, se 
tuvo que añadir un "listener" al botón de retorno que detectara cuando se pulsaba y que cancelara el evento por defecto,
evitando que se cerrara la aplicación. El problema fue que conseguir interceptar el evento antes de que se cerrara la
aplicación no fue sencillo, y hubo que probar varias soluciones hasta encontrar la correcta.

\paragraph{Splash screen en AndroidTv}
Otros de los problemas encontrados con Cordova en AndroidTv fue que desde las últimas actualizaciones implementa 
un "splash screen" por defecto que no se puede desactivar. Este "splash screen" se muestra al inicio de la aplicación
y justo a continuación el splash propio de la aplicación. Esto no es un problema en sí, pero el "splash screen" de
Cordova tiene una personalización muy limitada y compleja y por el momento solo se ha conseguido cambiar la imagen
de que se muestra, pasando del logo de Cordova al logo de la aplicación. Pero aún así, no es una solución definitiva
ya la imagen se ve con un formato extraño. Se ha intentado desactivar el "splash screen" y modificarlo de todas 
las formas posibles que indican tanto en la documentación como en los foros, pero conseguir el resultado al 100\%

