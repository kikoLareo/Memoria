\subsection{APIs y librerías utilizadas}
\label{sec:apis-librerias}

\subsubsection{Librerias}
\label{sec:librerias}

Para la implementación de la aplicación de análisis de datos se han utilizado diversas librerías que facilitaron el desarrollo de la misma.
Estas librerías establecen una serie de funciones y métodos que permiten realizar operaciones de forma más sencilla y rápida que si se
tuviesen que implementar desde cero. 

Una de estas librerías es \textit{Axios}. Esta librería facilita la realización de peticiones HTTP desde el cliente a un servidor.
Esta popular herramienta está basada en promesas que simplifica la comunicación con APIs. Permite realizar solicitudes \textit{GET},
\textit{POST}, \textit{PUT}, \textit{DELETE}, entre otras, de manera eficiente y con un manejo simplificado de las respuestas y errores.
Su facilidad de uso y su capacidad para manejar tanto solicitudes asíncronas como configuraciones avanzadas, como el establecimiento 
de cabeceras personalizadas y la gestión de tiempos de espera, han sido fundamentales para la integración con las APIs REST de la 
plataforma. Además, \textit{Axios} soporta la interceptación de solicitudes y respuestas, lo que facilita la implementación de
mecanismos de autenticación y la gestión centralizada de errores, mejorando así la robustez y seguridad de la aplicación.

\begin{verbatim}
    const config = {
    method: 'post',
    url: `${BASE_URL}/insertOne`,
    headers: {
      'Content-Type': 'application/json',
      'api-key': API_KEY,
    },
    data: data
  };

  try {
    const response = await axios(config);
    ...

\end{verbatim}

Otra libreria utilizada fue ChartJs cuyo uso se ha explicado en la sección \ref{sec:desarrollo-aplicacion}. Esta librería permite la
 creación de gráficas de forma sencilla y rápida. Ofrece una amplia variedad de gráficos, como barras, líneas, radar, polar,
entre otros, y permite personalizar los con colores, títulos, leyendas, entre otros.

\subsubsection{APIs : OpenAi y MongoDB}
\label{sec:apis}

Además de la API de Matomo, se han utilizado otras APIs para dotar de funcionalidades a la aplicación de análisis de datos.
Una de estas funcionalidades es el análisis y explicación de los datos recopilados a través de la API de \textit{OpenAi}. Esta funcionalidad
permite a los usuarios obtener una explicación de los datos recopilados, lo que facilita la interpretación de los mismos y la toma
de decisiones. 

Esta API utiliza tanto el contexto de la página como los datos de la gráfica a analizar para generar una explicación en lenguaje natural y
comprensible para cualquier usuario sin conocimientos técnicos. Para ello, hay varios elementos necesarios: contexto, datos y petición. 

\paragraph{Contexto: } Información general de la página web. Este contexto se va creando y detallando con el tiempo ya que cada vez
que se realiza una petición a la API, además del análisis, se pide que se actualice ese contexto para siguientes peticiones para poder dotar
a la explicación de un mayor detalle y precisión. 

\paragraph{Datos: } Datos de la gráfica a analizar. Estos datos son los que se recopilan a través de la API de Matomo y se envían a la API
de \textit{OpenAi} para su análisis.

\paragraph{Petición: } Promt generado a partir de información de la página (Contexto) y los datos de la gráfica (descripción, variables, valores, etc).
Este promt se envía a la API de \textit{OpenAi} para que genere una explicación en lenguaje natural.

Tanto el contexto como los datos son almacenados para alimentar a futuras llamadas y crear promts más precisos y detallados. Para almacenar
estos datos se ha utilizado una base de datos no relacional, en concreto \textit{MongoDB}. 

\paragraph{MongoDB: } MongoDB es una base de datos no relacional que permite almacenar datos en formato JSON. Es una base de datos
escalable y flexible que permite almacenar grandes cantidades de datos y realizar consultas de forma rápida y eficiente.
Para la integración de MongoDB con la aplicación de análisis de datos se ha utilizado el servicio en la nube de MongoDB Atlas.
Este servicio ofrece una base de datos en la nube que permite almacenar y consultar datos de forma segura y eficiente. Además, MongoDB Atlas
ofrece una serie de funcionalidades avanzadas, como la replicación de datos, la recuperación ante desastres y la escalabilidad automática,
que garantizan la disponibilidad y el rendimiento de la base de datos.

MongoDB Atlas se ha utilizado a través de su api para almacenar los datos de contexto y datos de las gráficas para su posterior análisis
y generación de explicaciones. 

\paragraph{API de MongoDB}
\begin{lstlisting}[language=Java]
  const BASE_URL = 'https://eu-west-2.aws.data.mongodb-api.com
    /app/data-hrcfvpe/endpoint/data/v1/action';

  exports.handler = async (event, context) => {
    const { collection, query } = JSON.parse(event.body);
  
    const data = JSON.stringify({
      collection: collection,
      database: 'kanaloa',
      dataSource: 'kanaloa',
      filter: query,
    
    });
  
    const config = {
      method: 'post',
      url: `${BASE_URL}/findOne`,
      headers: {
        'Content-Type': 'application/json',
        'api-key': API_KEY,
      },
      data: data
    };
  
    try {
      const response = await axios(config);
      return {
        statusCode: 200,
        body: JSON.stringify(response.data.document),
      };
    } catch (error) {
      console.error(error);
      return {
        statusCode: 500,
        body: 'Error fetching data',
      };
    }
  };
\end{lstlisting}
    

