\subsection{APIs y librerías utilizadas}
\label{sec:apis-librerias}

\subsubsection{Librerías}
\label{sec:librerias}

Para la implementación de la aplicación de análisis de datos se han utilizado diversas librerías que han facilitado el desarrollo. 
Estas librerías proporcionan funciones y métodos que permiten realizar operaciones de forma más rápida y eficiente que si se 
tuvieran que implementar desde cero.

Una de estas librerías es \textit{Axios}, la cual facilita la realización de peticiones HTTP desde el cliente a un servidor. 
Esta herramienta, basada en promesas, simplifica la comunicación con APIs, permitiendo realizar solicitudes \textit{GET}, 
\textit{POST}, \textit{PUT} y \textit{DELETE}, entre otras, de manera eficiente y con un manejo sencillo de respuestas y errores. 
Su facilidad de uso y capacidad para manejar solicitudes asíncronas y configuraciones avanzadas, como cabeceras personalizadas 
y gestión de tiempos de espera, han sido clave para la integración con las APIs REST de la plataforma. Además, \textit{Axios} 
soporta la interceptación de solicitudes y respuestas, lo que facilita la implementación de mecanismos de autenticación y gestión 
centralizada de errores, mejorando así la robustez y seguridad de la aplicación.

\begin{verbatim}
    const config = {
    method: 'post',
    url: `${BASE_URL}/insertOne`,
    headers: {
      'Content-Type': 'application/json',
      'api-key': API_KEY,
    },
    data: data
  };

  try {
    const response = await axios(config);
    ...
\end{verbatim}

Otra librería utilizada fue \textit{ChartJs}, explicada en la sección \ref{sec:desarrollo-aplicacion}. Esta librería permite la 
creación de gráficos de forma sencilla y rápida. Ofrece una amplia variedad de gráficos, como barras, líneas y gráficos circulares, 
que se pueden personalizar con colores, títulos y leyendas.

\subsubsection{APIs: OpenAi y MongoDB}
\label{sec:apis}

Además de la API de Matomo, se han integrado otras APIs para dotar de funcionalidades adicionales a la aplicación de análisis de datos. 
Una de estas funcionalidades es el análisis y explicación de los datos recopilados mediante la API de \textit{OpenAi}. Esta funcionalidad 
permite a los usuarios obtener una interpretación de los datos, lo que facilita la toma de decisiones.

La API de \textit{OpenAi} utiliza tanto el contexto de la página como los datos de las gráficas para generar una explicación en lenguaje 
natural, comprensible para cualquier usuario sin conocimientos técnicos. Los elementos necesarios para este proceso son: contexto, datos y petición.

\paragraph{Contexto:} Información general del sitio web. Este contexto se va creando y enriqueciendo con cada petición a la API, 
permitiendo ofrecer explicaciones más detalladas y precisas en futuras consultas.

\paragraph{Datos:} Datos específicos de la gráfica a analizar, recopilados a través de la API de Matomo y enviados a la API de \textit{OpenAi} 
para su interpretación.

\paragraph{Petición:} \textit{Prompt} generado a partir de la información del contexto y los datos de la gráfica, que se envía a la API de \textit{OpenAi} 
para generar una explicación en lenguaje natural.

Tanto el contexto como los datos se almacenan para alimentar futuras llamadas y crear \textit{prompts} más precisos. Para el almacenamiento de estos datos 
se utiliza \textit{MongoDB}, una base de datos no relacional.

\paragraph{MongoDB:} MongoDB es una base de datos no relacional que permite almacenar datos en formato JSON. Su escalabilidad y flexibilidad permiten 
almacenar grandes cantidades de datos y realizar consultas de manera eficiente. Para este proyecto, se ha utilizado el servicio en la nube \textit{MongoDB Atlas}, 
que ofrece funcionalidades avanzadas como replicación de datos, recuperación ante desastres y escalabilidad automática, garantizando disponibilidad y rendimiento.

MongoDB Atlas se utiliza para almacenar los datos de contexto y las gráficas, permitiendo su análisis y generación de explicaciones.

\paragraph{API de MongoDB}

\begin{lstlisting}[language=Java]
  const BASE_URL = 'https://eu-west-2.aws.data.mongodb-api.com
    /app/data-hrcfvpe/endpoint/data/v1/action';

  exports.handler = async (event, context) => {
    const { collection, query } = JSON.parse(event.body);
  
    const data = JSON.stringify({
      collection: collection,
      database: 'kanaloa',
      dataSource: 'kanaloa',
      filter: query,
    
    });
  
    const config = {
      method: 'post',
      url: `${BASE_URL}/findOne`,
      headers: {
        'Content-Type': 'application/json',
        'api-key': API_KEY,
      },
      data: data
    };
  
    try {
      const response = await axios(config);
      return {
        statusCode: 200,
        body: JSON.stringify(response.data.document),
      };
    } catch (error) {
      console.error(error);
      return {
        statusCode: 500,
        body: 'Error fetching data',
      };
    }
  };
\end{lstlisting}
