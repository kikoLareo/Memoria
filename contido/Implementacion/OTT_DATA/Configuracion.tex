\subsection{Instalación, configuración e integración de la API de Matomo}
\label{sec:configuracion-matomo}

Para el desarrollo de la aplicación de análisis de datos, se ha utilizado la herramienta de análisis web Matomo. Aunque existen 
otras opciones como Google Analytics, se ha optado por Matomo en este primer desarrollo, ya que es la herramienta más utilizada 
en las distintas aplicaciones de la empresa.

Matomo \cite{matomo} es una herramienta de análisis web de código abierto que permite a los administradores de sitios web 
obtener información detallada sobre los visitantes. Ofrece funcionalidades como la recopilación de datos, generación de informes, 
y la personalización de los mismos. Una de las principales características es su API REST, que permite a los desarrolladores 
acceder a los datos recopilados y realizar diversas operaciones sobre ellos.

A través de la API, se puede obtener la información necesaria para construir la aplicación de análisis de datos, utilizando estos 
datos como entrada para generar gráficas y tablas.

Matomo ofrece una integración sencilla con las aplicaciones en las que se requiera la recopilación de datos. Los pasos a seguir son:

\begin{enumerate}
    \item Registrarse en la plataforma de Matomo.
    \item Dar de alta un nuevo sitio web en la plataforma.
    \item Descargar el código de seguimiento y añadirlo a la aplicación.
\end{enumerate}

Tras estos pasos, Matomo comenzará a recopilar los datos de los visitantes de la aplicación. Además, es posible configurar opciones 
más avanzadas, como eventos personalizados, objetivos o la creación de segmentos. Para obtener datos sobre las reproducciones de vídeo, 
el reproductor debe configurarse para enviar eventos a Matomo cada vez que se reproduzca un vídeo.

\paragraph{Ejemplo de código de seguimiento de Matomo}

\begin{lstlisting}[language=HTML]
    <!-- Matomo -->
    <script>
    var _paq = window._paq = window._paq || [];
    /* tracker methods like "setCustomDimension" should be called before "trackPageView" */
    _paq.push(['trackPageView']);
    _paq.push(['enableLinkTracking']);
    (function() {
        var u="https://tiivii-ott.matomo.cloud/";
        _paq.push(['setTrackerUrl', u+'matomo.php']);
        _paq.push(['setSiteId', 'IdSite']);
        var d=document, g=d.createElement('script'), s=d.getElementsByTagName('script')[0];
        g.async=true; g.src='https://cdn.matomo.cloud/tiivii-ott.matomo.cloud/matomo.js'; s.parentNode.insertBefore(g,s);
    })();
    </script>
    <!-- End Matomo Code -->
\end{lstlisting}

Una vez que Matomo está recopilando los datos, se pueden realizar consultas a la API para obtener la información necesaria. 
Para ello, es necesario obtener un \textit{token} de acceso a la API, disponible en la configuración de la cuenta de Matomo, 
y registrar la URL de la página web desde la que se realizará la consulta. Con esta configuración y sabiendo qué datos se quieren 
obtener \ref{sec:diseno-estudio} y qué métodos utilizar, se puede empezar a realizar las consultas a la API.
