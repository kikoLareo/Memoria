\subsection{Instalación, configuración e integración de la API de Matomo}
\label{sec:configuracion-matomo}

Para la construcción y desarrollo de la aplicación de análisis de datos, se ha utilizado la herramienta de análisis web Matomo. 
Aunque existen otras herramientas de análisis web, como Google Analytics, para un primer desarrollo se ha optado únicamente por 
esta herramienta, ya que es la más utilizada en las distintas aplicaciones de la empresa. 

Matomo \cite{matomo} es una herramienta de análisis web de código abierto que permite a los administradores de sitios web
obtener información sobre los visitantes de sus aplicaciones y páginas web. Ofrece una amplia gama de funcionalidades, como la
recopilación de datos, la generación de informes y la personalización de los mismos. Una de las funcionalidades que ofrece es 
la utilización de una API REST, que permite a los desarrolladores acceder a los datos recopiados por Matomo y realizar operaciones
sobre ellos. 

Con estas consultas, se puede obtener la información necesaria para la construcción de la aplicación de análisis de datos.
A través de la API se obtienen los datos que que servirán como entrada para la construcción de las gráficas y tablas.

Matomo ofrece una rápida y sencilla instalación e integración con las distintas aplicaciones en las que se requiera la 
recogida de datos. Los pasos a seguir son: 

\begin{enumerate}
    \item Darse de alta en la plataforma de Matomo.
    \item Dar de alta un nuevo sitio web en la plataforma.
    \item Descargar el código de seguimiento y añadirlo a la aplicación.
\end{enumerate}

Con estos pasos, Matomo comenzará a recopilar los datos de los visitantes de la aplicación. Si se desea obtener información más
detallada y concreta existen más opciones de configuración, como la creación de eventos personalizados, la configuración de
objetivos o la creación de segmentos. Para conocer datos sobre las reproducciones de los vídeos, habrá que configurar el reproductor
de vídeo para que envíe eventos a Matomo cada vez que se reproduzca un vídeo.

\paragraph{Ejemplo de código de seguimiento de Matomo}

\begin{lstlisting}[language=HTML]
    <!-- Matomo -->
    <script>
    var _paq = window._paq = window._paq || [];
    /* tracker methods like "setCustomDimension" should be called before "trackPageView" */
    _paq.push(['trackPageView']);
    _paq.push(['enableLinkTracking']);
    (function() {
        var u="https://tiivii-ott.matomo.cloud/";
        _paq.push(['setTrackerUrl', u+'matomo.php']);
        _paq.push(['setSiteId', 'IdSite']);
        var d=document, g=d.createElement('script'), s=d.getElementsByTagName('script')[0];
        g.async=true; g.src='https://cdn.matomo.cloud/tiivii-ott.matomo.cloud/matomo.js'; s.parentNode.insertBefore(g,s);
    })();
    </script>
    <!-- End Matomo Code -->
\end{lstlisting}

Una vez Matomo ya está recopilando los datos, se puede comenzar a realizar consultas a la API para obtener la información
necesaria. Para ello, se debe obtener un token de acceso a la API disponible en la configuración de la cuenta de Matomo y 
registrar la URL de la página web en la que se está realizando la consulta. Con esta configuración lista, y conociendo los
datos que se quieren obtener \ref{sec:diseno-estudio} y a través de qué métodos, se puede comenzar a realizar las consultas 
a la API. 
