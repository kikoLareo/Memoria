\chapter{Futuros Objetivos}
\label{chap:futuros_objetivos}

Los objetivos futuros, tanto a corto como a largo plazo, ya han sido explicados en mayor o menor medida a lo largo de la memoria.
Estos objetivos abordan futuras mejoras y ampliaciones de la plataforma OTT, así como la exploración de nuevas tecnologías y
funcionalidades que puedan aportar valor añadido a la aplicación. 

Uno de los objetivos principales es la expansión de la aplicación a la mayor cantidad de dispositivos y plataformas posibles.
Para esto, se planea seguir trabajando en la adaptación de la aplicación a nuevas plataformas y sistemas operativos. Los objetivos a 
corto plazo son conseguir una aplicación funcional en los \textit{marketplaces} tanto de Vidaa OS (Hisense) como de FireTV, y a largo plazo, tener las 
aplicaciones móviles y de tabletas en los \textit{marketplaces} de Google Play y App Store.

Otro objetivo es seguir optimizando el rendimiento de la aplicación, tanto en términos de velocidad y fluidez como en el consumo de recursos, para 
dotar a la aplicación de una experiencia de usuario a la altura de las mejores aplicaciones OTT del mercado. Para ello, hay que continuar con 
el análisis de nuevas tecnologías y técnicas de optimización, así como con la mejora de los microservicios y la infraestructura de la aplicación
para garantizar un rendimiento óptimo en todo momento.

Un objetivo muy importante, que se quiere llevar a cabo a corto plazo, es la implementación de una mayor personalización para los 
clientes. Esto incluye la posibilidad de personalizar la interfaz de usuario, los contenidos y las funcionalidades de la aplicación, manteniendo 
un solo código y permitiendo a los clientes diferenciarse cada vez más entre ellos, siempre procurando reducir el tiempo de desarrollo y
mantenimiento de la aplicación.

También se continuará implementando funcionalidades de productos de la empresa, como la previsualización de los contenidos, que ofrece a los 
usuarios un video corto con un breve resumen de un contenido, generado mediante inteligencia artificial. Este producto está siendo
utilizado actualmente por clientes como RTVE, y se planea implementarlo en las próximas semanas en este proyecto.

En cuanto a la aplicación de análisis de datos, el plan es optimizar y maximizar la información recogida en las distintas aplicaciones, y 
continuar con la expansión de la misma. Por el momento, está habilitada para uso interno, pero se planea, en un futuro cercano,
ofrecerla a los clientes para que puedan analizar los datos de sus aplicaciones y tomar decisiones basadas en ellos.

