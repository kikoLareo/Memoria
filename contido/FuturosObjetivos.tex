\chapter{Futuros Objetivos}
\label{chap:futuros_objetivos}

Los objetivos futuros, tanto a corto como a largo plazo ya han sido explicados en mayor o menor medida a lo largo de la memoria.
Estos objetivos abordan futuras mejoras y ampliaciones de la plataforma OTT, así como la exploración de nuevas tecnologías y
funcionalidades que puedan aportar valor añadido a la aplicación. 

Uno de los objetivos principales es la expansión de la aplicación a la mayor cantidad de dispositivos y plataformas posibles.
Para esto se planea seguir trabajando en la adaptación de la aplicación a nuevas plataformas y sistemas operativos. Los objetivos a 
corto plazo son conseguir una aplicación funcional en los marketplaces tanto de Vidaa Os  (Hisense) como de FireTv y a largo plazo tener las 
aplicaciones de movil y tableta en los marketplaces de Google Play y App Store.

Otro objetivo es seguir optimizando el rendimiento de la aplicación, tanto en términos de velocidad y fluidez como de consumo de recursos para 
dotar a la aplicación de una experiencia de usuario a la altura de las mejores aplicaciones OTT del mercado. Para ello, hay que continuar con 
el análisis de nuevas tecnologías y técnicas de optimización, así como con la mejora de los microservicios y la infraestructura de la aplicación
para garantizar un rendimiento óptimo en todo momento.

Un objetivo muy importantes el cual quiero llevar a cabo a muy corto plazo es la implementación en la personalización que podemos ofrecer a los 
clientes. Esto incluye la posibilidad de personalizar la interfaz de usuario, los contenidos y las funcionalidades de la aplicación manteniendo 
un solo código y permitiendo a los clientes diferenciarse cada vez más unos de otros, siempre procurando reducir el tiempo de desarrollo y
mantenimiento de la aplicación.

Se continuarán implementado funcionalidades de productos de la empresa como puede ser  la previsualización de los contenidos, el cual ofrece a los 
usuarios un video corto con un breve resumen de un contenido, generado este a través de inteligencia artificial. Este producto esta siendo
utilizado en estos momentos por clientes como Rtve y se planea implementarlo en las próximas semanas en este proyecto.

En cuanto a la aplicación de análisis de datos, el plan es optimizar y máximizar la información recogida en las distintas aplicaciones y 
continuar con la expansión de la aplicación. Por el momento está habilitada para el uso interno, pero se planea en un futuro cercano
ofrecerla a los clientes para que puedan analizar los datos de sus aplicaciones y tomar decisiones basadas en ellos.

