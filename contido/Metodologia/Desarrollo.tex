\section{Metodología de Desarrollo}
\label{sec:metodologia_desarrollo}

El desarrollo de la plataforma OTT se llevó a cabo siguiendo una metodología ágil. Esta metodología
permite una mayor flexibilidad en el desarrollo del proyecto, y facilita la adaptación a los cambios
que puedan surgir durante el desarrollo del proyecto. Se asegura así la evolución y mejora controlada
y continua del producto final.

\subsection{Implementacion progresiva e incremental}
\label{subsec:implementacion_progresiva}

El proceso de desarrollo se estructuró en torno a la implementación de funcionalidades de manera 
progresiva e incremental. En lugar de intentar desarrollar todas las características al mismo tiempo, 
el proyecto se dividió en ciclos más pequeños, cada uno enfocado en la entrega de una funcionalidad específica.

Estos ciclos se componen de las siguientes fases:
\begin{itemize}
    \item \textbf{Análisis:} En esta fase se escoge la funcionalidad a implementar y se identifican los 
    requisitos de la misma. La selección de la funcionalidad a implementar se realiza en base a la
    prioridad de la misma y a la dependencia de otras funcionalidades.
    \item \textbf{Diseño:} En esta fase se diseña la arquitectura de la funcionalidad a implementar. Se
    identifican los componentes necesarios y se definen las interacciones entre ellos.
    \item \textbf{Implementación:} En esta fase se implementa la funcionalidad siguiendo el diseño
    previamente definido.
    \item \textbf{Pruebas:} En esta fase se realizan pruebas unitarias y de integración para asegurar
    que la funcionalidad implementada cumple con los requisitos definidos en la fase de análisis.
\end{itemize}

\subsection{Adaptación continua y flexibilidad}
\label{subsec:adaptacion_continua}

Una de las ventajas de la metodología ágil es la adaptación continua y la flexibilidad que ofrece, permitiendo
hacer frente a los cambios y nuevas necesidades que puedan surgir durante el desarrollo del proyecto y ajustar
las prioridades y el enfoque del proyecto en caso de ser necesario.

Durante el desarrollo de la plataforma OTT, se realizaron reuniones periódicas los clientes para revisar
el progreso del proyecto y validar las funcionalidades implementadas. Estos clientes
pudieron proponer cambios y mejoras en la plataforma, que fueron evaluados y estudiados. Debido a la naturaleza
multicliente del proyecto, en todo momento se busco que los cambios o mejoras realizadas en base a peticiones de un 
cliente en específico no afectaran a los demás clientes. Uno de los objetivos principales que tiene el proyecto
es seguir aumentando la flexibilidad y personalización de la plataforma para cada cliente.


\subsection{Reuniones de Validación y Planificación}
\label{subsec:reuniones_validacion}

Después de cada ciclo de desarrollo, se llevaron a cabo reuniones de validación para revisar las 
funcionalidades completadas y planificar las siguientes etapas. Estas reuniones fueron fundamentales 
para asegurar que el proyecto se mantuviera alineado con los objetivos del cliente y para permitir la 
rápida adaptación a nuevos requisitos. En estas sesiones, se discutía el estado actual del proyecto, 
se evaluaban posibles mejoras, y se decidía la próxima funcionalidad a implementar. Este enfoque 
colaborativo garantizó una toma de decisiones informada y una respuesta rápida a los cambios necesarios.


\subsection{Ventajas de la metodología ágil}
\label{subsec:ventajas_agil}

La metodología ágil ofrece una serie de ventajas que permiten un desarrollo más eficiente y controlado del proyecto.
Algunas de las ventajas más destacadas son las siguientes:

\begin{itemize}
    \item \textbf{Mayor flexibilidad:} La metodología ágil permite adaptarse a los cambios y nuevas necesidades
    que puedan surgir durante el desarrollo del proyecto.
    \item \textbf{Mayor control:} Al dividir el proyecto en ciclos más pequeños, se facilita el seguimiento y
    control del progreso del proyecto.
    \item \textbf{Mayor calidad:} Al realizar pruebas continuas a lo largo del desarrollo del proyecto, se asegura
    que el producto final cumple con los requisitos y está libre de errores.
    \item \textbf{Mayor satisfacción del cliente:} Al involucrar al cliente en el proceso de desarrollo y permitirle
    proponer cambios y mejoras, se asegura que el producto final cumple con sus expectativas.
\end{itemize}