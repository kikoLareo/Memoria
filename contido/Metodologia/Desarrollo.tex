\section{Metodología de Desarrollo}
\label{sec:metodologia_desarrollo}

El desarrollo tanto de la plataforma OTT como de la aplicación de análisis de datos se llevó a cabo siguiendo 
una metodología ágil. Esta metodología permitió una mayor flexibilidad en el desarrollo de ambos proyectos, 
facilitando la adaptación a los cambios y asegurando la evolución y mejora continua de los productos finales.

\subsection{Implementación progresiva e incremental}
\label{subsec:implementacion_progresiva}

El proceso de desarrollo se estructuró en ciclos pequeños y manejables, cada uno enfocado en la implementación de 
una funcionalidad específica. Cada ciclo incluía las siguientes fases:

\begin{itemize}
    \item \textbf{Análisis:} Se identificaban los requisitos de la funcionalidad a implementar, priorizando según su importancia y dependencia de otras funcionalidades.
    \item \textbf{Diseño:} Se diseñaba la arquitectura necesaria, definiendo componentes e interacciones.
    \item \textbf{Implementación:} La funcionalidad se desarrollaba según el diseño especificado.
    \item \textbf{Pruebas:} Se realizaban pruebas unitarias y de integración para asegurar que la funcionalidad cumpliera con los requisitos.
\end{itemize}

\subsection{Adaptación continua y reuniones de validación}
\label{subsec:adaptacion_reuniones}

Una de las principales ventajas de la metodología ágil fue la capacidad de adaptación continua. Reuniones periódicas 
con los clientes permitieron revisar el progreso, validar las funcionalidades y ajustar las prioridades según fuera 
necesario. Estas sesiones garantizaron que el proyecto se mantuviera alineado con las expectativas del cliente y 
permitieron una respuesta rápida a cualquier cambio o nueva necesidad.

\subsection{Ventajas de la metodología ágil}
\label{subsec:ventajas_agil}

El enfoque ágil ofreció varias ventajas clave para el desarrollo del proyecto:
\begin{itemize}
    \item \textbf{Flexibilidad y Adaptación:} Posibilidad de ajustar el proyecto a nuevas necesidades durante su desarrollo.
    \item \textbf{Control y Seguimiento:} División del proyecto en ciclos pequeños facilitó el seguimiento detallado del progreso.
    \item \textbf{Calidad y Satisfacción del Cliente:} Pruebas continuas y participación activa del cliente aseguraron un producto final que cumplió con las expectativas y estándares de calidad.
\end{itemize}
