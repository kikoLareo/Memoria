\subsection{Diseño del proyecto}
\label{sec:metodologia_diseno}

El objetivo principal del proyecto fue desarrollar una aplicación OTT que, a partir de un único código fuente, 
pudiera adaptarse a diversos dispositivos y cumplir con las necesidades de diferentes clientes. Además, se planificó 
el desarrollo de una aplicación complementaria para el análisis de datos de uso, destinada a mejorar la comprensión 
del comportamiento de los usuarios y optimizar la experiencia de uso.

La primera fase del proyecto consistió en el análisis y diseño de ambas plataformas, estableciendo los siguientes 
objetivos generales:
\begin{itemize}
    \item Crear una plataforma OTT que permita a los clientes tener su propia aplicación de distribución de contenido.
    \item Ofrecer una alta capacidad de personalización para que los clientes adapten la aplicación a su marca y contenido.
    \item Desarrollar una plataforma escalable y adaptable a las diversas necesidades de cada cliente.
    \item Asegurar que el código funcione en la mayor cantidad de dispositivos posible.
    \item Desarrollar una aplicación de análisis de datos que recopile y visualice información sobre el uso de la plataforma OTT, facilitando la toma de decisiones y la mejora continua del servicio.
\end{itemize}

\paragraph{Meta principal:} Evitar la necesidad de crear un código fuente distinto para cada cliente y dispositivo.

Con la idea general del proyecto clara, se definieron los requisitos, funcionalidades, y el alcance tanto de la 
plataforma OTT como de la aplicación de análisis de datos. Se realizó un análisis de mercado, priorizando el 
soporte inicial para los sistemas Android, Tizen y WebOS, con miras a ampliar a otros sistemas en el futuro.

También se revisaron productos similares desarrollados previamente por la empresa, para establecer una base 
sólida en términos de características visuales y funcionales, que luego se mejorarían y evolucionarían.

Se planificaron las tareas y funcionalidades prioritarias para lograr un producto mínimo viable lo antes 
posible, con el fin de comenzar las pruebas y presentaciones tanto de la plataforma OTT como de la aplicación 
de análisis de datos.
