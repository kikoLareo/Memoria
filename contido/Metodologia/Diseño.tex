\subsection{Diseño del proyecto}
\label{sec:metodologia_diseno}

Crear un proyecto que a partir de un mismo código fuenta genere una aplicación OTT con la información y funcionalidades
para cada cliente, adaptandose a la mayor cantidad de dispositivos posibles. Esta fue la premisa principal del proyecto
y el objetivo que se me planteó cuando comencé en la empresa. 

La primera etapa del proyecto fue el analisis y diseño de la plataforma. Se definieron los objetivos generales del proyecto:

\begin{itemize}
    \item Crear una plataforma OTT que permita a los clientes de la empresa tener su propia aplicación de distribución de contenido.
    \item Permitir a los clientes personalizar su aplicación lo máximo posible con su propia marca y contenido.
    \item Crear una plataforma que sea escalable y que pueda adaptarse lo máximo posible a las necesidades de cada cliente.
    \item Crear un código que permita ejecutar la aplicación en la mayor cantidad de dispositivos posibles.
\end{itemize}

\paragraph{Y lo más importante:} No tener que crear un código fuente para cada cliente y dispositivo. 

Una vez clara la idea general del proyecto, se procedió a la definición de los requisitos y funcionalidades de la plataforma
(páginas, menús, funcionalidades, etc.), así como el alcance del proyecto. Se realizó un análisis de mercado y 
se decidió comenzar dando soporte por el momento en los sistemas operativos Android, Tizen y WebOS, con la premisa 
de que el código debe estar preparado para ampliarse a otros sistemas operativos en el futuro.

Otro punto importante fue el estudio de los productos similares ya existentes creados por la empresa, para comenzar 
creando la base del producto con las mismas caracteristicas visuales y funcionales, para luego ir evolucionando y 
mejorando el producto.

Se realizó una planificación inicial del proyecto, definiendo las tareas y funcionalidades prioritarias, con la idea de 
tener un producto mínimo viable lo antes posible para comenzar a presentarlo y probarlo. 


