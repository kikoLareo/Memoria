\section{Descripción de las Etapas de Desarrollo}
En esta sección, se describen las principales etapas del desarrollo del proyecto, 
desde la planificación inicial hasta el despliegue final. Cada etapa jugó un papel 
crucial en la evolución del proyecto, permitiendo una implementación organizada y eficiente de la plataforma OTT.

\subsection{Fase de Análisis y Planificación}
\subsubsection{Recolección de Requisitos}
En esta etapa, se realizó un análisis exhaustivo para identificar los requisitos 
funcionales y no funcionales del proyecto. Esto incluyó reuniones con los clientes
para entender las expectativas y necesidades de cada uno, así como un análisis de mercado para
asegurar que las plataformas cumplieran con los estándares actuales. 

\subsection{Fase de Diseño}
\subsubsection{Diseño de Arquitectura}
Durante esta etapa, se definió la arquitectura general de la plataforma  OTT con sus componentes. Se utilizó 
una arquitectura basada en microservicios para garantizar la escalabilidad y flexibilidad del 
sistema e integración con los servicios existentes en la empresa. También se diseñó la estructura de 
comunicación entre los componentes y se eligieron las tecnologías más adecuadas para cada parte del sistema.

Se estudiarón los conjuntos de datos y las APIs necesarias para creación de la plataforma de análisis de datos.

\subsubsection{Diseño de la Interfaz de Usuario}
Se comenzó a trabajar sobre una interfaz básica para la plataforma OTT que permitiera el desarrollo de las funcionalidades
principales de la plataforma y poco a poco se fue refinando y mejorando en función de las peticiones
de los clientes y las pruebas de usabilidad realizadas. A través de reuniones con los clientes y 
analizando sus peticiones y prototipos, se fueron contruyendo los diseños de manera conjunta hasta
llegar a un diseño final que satisfaciera las necesidades de los usuarios.

Se diseño la interfaz de usuario para la plataforma de análisis de datos, con el objetivo de que los usuarios
pudieran visualizar y analizar los datos de manera sencilla y efectiva sin necesidad de conocimientos técnicos.

\subsection{Fase de Desarrollo}
\subsubsection{Implementación de Funcionalidades}
Esta fase fue el núcleo del proyecto, donde se desarrollaron las funcionalidades clave de la 
plataforma OTT. Cada funcionalidad se implementó de manera incremental, siguiendo un ciclo 
ágil de análisis, diseño, codificación y pruebas. Se utilizaron tecnologías web como JavaScript, 
HTML y CSS para asegurar la compatibilidad multiplataforma.

Para la plataforma de análisis de datos, se implementaron las funcionalidades necesarias para
la carga, procesamiento y visualización de los datos, así como de análisis y generación de informes.

\subsubsection{Optimización y Refactorización}
A medida que se añadían nuevas funcionalidades, el código se refactorizaba y optimizaba para 
mejorar el rendimiento y la mantenibilidad. Este proceso incluyó la simplificación de la lógica 
de negocio, la mejora de la estructura del código, y la optimización de la carga y la respuesta de la aplicación.

\subsection{Fase de Pruebas}
\subsubsection{Pruebas Unitarias y de Integración}
Se realizaron pruebas unitarias para cada componente, asegurando que las funciones individuales 
funcionaran correctamente. Luego, se realizaron pruebas de integración para verificar que los 
componentes interactuaran de manera efectiva y sin conflictos.

\subsubsection{Pruebas de Compatibilidad}
Se llevaron a cabo pruebas exhaustivas en diferentes dispositivos y sistemas operativos 
(web, Tizen, WebOS, Android TV) para asegurar que la aplicación funcionara correctamente 
en todos los entornos previstos. Esto incluyó pruebas de rendimiento para garantizar que 
la aplicación se ejecutara sin problemas en dispositivos con capacidades de hardware limitadas.


\subsection{Conclusión}	  
Estas etapas se llevaron a cabo de manera iterativa y colaborativa, con reuniones periódicas
con los clientes para validar el progreso y ajustar las prioridades según fuera necesario.
Así para cada funcionalidad nueva se realizaba un ciclo de análisis, diseño, implementación y pruebas
para asegurar que la funcionalidad cumpliera con los requisitos y expectativas del cliente.