\section{Descripción de las Etapas de Desarrollo}
En esta sección, se describen las principales etapas del desarrollo del proyecto, 
desde la planificación inicial hasta el despliegue final. Cada etapa jugó un papel 
crucial en la evolución del proyecto, permitiendo una implementación organizada y eficiente de la plataforma OTT.

\subsection{Fase de Análisis y Planificación}
\subsubsection{Recolección de Requisitos}
En esta etapa, se realizó un análisis exhaustivo para identificar los requisitos 
funcionales y no funcionales del proyecto. Esto incluyó reuniones con los \textit{stakeholders} 
para entender las expectativas y necesidades del cliente, así como un análisis de mercado para
 asegurar que la plataforma cumpliera con los estándares actuales.


\subsubsection{Estudio de Viabilidad}
Se evaluó la viabilidad técnica del proyecto, considerando las tecnologías disponibles, 
las capacidades del equipo, y las restricciones de tiempo y presupuesto. Este estudio 
permitió definir el alcance realista del proyecto y establecer prioridades.

\subsubsection{Planificación Inicial}
Se elaboró un cronograma detallado con hitos clave, que incluyó la planificación de \textit{sprints} 
y la asignación de recursos para las diferentes fases del desarrollo.

\subsection{Fase de Diseño}
\subsubsection{Diseño de Arquitectura}
Durante esta etapa, se definió la arquitectura general de la plataforma. Se decidió utilizar 
una arquitectura basada en microservicios para garantizar la escalabilidad y flexibilidad del 
sistema. También se diseñó la estructura de comunicación entre los componentes y se eligieron 
las tecnologías más adecuadas para cada parte del sistema.

\subsubsection{Diseño de la Interfaz de Usuario}
Se crearon prototipos y \textit{wireframes} para la interfaz de usuario utilizando herramientas 
como Figma o Adobe XD. Estos diseños fueron iterados y refinados a través de revisiones con los 
\textit{stakeholders}, asegurando que la interfaz cumpliera con las expectativas de usabilidad y estética.

\subsection{Fase de Desarrollo}
\subsubsection{Implementación de Funcionalidades}
Esta fase fue el núcleo del proyecto, donde se desarrollaron las funcionalidades clave de la 
plataforma OTT. Cada funcionalidad se implementó de manera incremental, siguiendo un ciclo 
ágil de análisis, diseño, codificación y pruebas. Se utilizaron tecnologías web como JavaScript, 
HTML y CSS para asegurar la compatibilidad multiplataforma.

\subsubsection{Optimización y Refactorización}
A medida que se añadían nuevas funcionalidades, el código se refactorizaba y optimizaba para 
mejorar el rendimiento y la mantenibilidad. Este proceso incluyó la simplificación de la lógica 
de negocio, la mejora de la estructura del código, y la optimización de la carga y la respuesta de la aplicación.

\subsection{Fase de Pruebas}
\subsubsection{Pruebas Unitarias y de Integración}
Se realizaron pruebas unitarias para cada componente, asegurando que las funciones individuales 
funcionaran correctamente. Luego, se realizaron pruebas de integración para verificar que los 
componentes interactuaran de manera efectiva y sin conflictos.

\subsubsection{Pruebas de Compatibilidad}
Se llevaron a cabo pruebas exhaustivas en diferentes dispositivos y sistemas operativos 
(web, Tizen, WebOS, Android TV) para asegurar que la aplicación funcionara correctamente 
en todos los entornos previstos. Esto incluyó pruebas de rendimiento para garantizar que 
la aplicación se ejecutara sin problemas en dispositivos con capacidades de hardware limitadas.

\subsection{Fase de Despliegue y Mantenimiento}
\subsubsection{Despliegue en Producción}
Una vez que todas las funcionalidades principales fueron implementadas y probadas, se procedió 
al despliegue de la plataforma en los entornos de producción. Se utilizaron \textit{pipelines} 
de CI/CD para automatizar el proceso de despliegue, minimizando los riesgos de errores y asegurando 
una transición suave a producción.

\subsubsection{Plan de Mantenimiento}
Se desarrolló un plan de mantenimiento para garantizar la continuidad y estabilidad de la 
plataforma tras su lanzamiento. Esto incluyó la planificación de actualizaciones regulares, 
la gestión de incidentes, y la respuesta a problemas reportados por los usuarios. También se 
incluyó un plan para la incorporación de nuevas funcionalidades según las necesidades futuras de los clientes.
