\subsection{Diseño de la arquitectura del sistema}
\label{subsec:diseno:ott:arquitectura}

Como ya se ha mencionado en la sección \ref{subsec:fundamentos_teoricos_arquitectura}
la arquitectura de esta aplicación esta diseñada para integrarse en la arquitectura general 
de la empresa y se basa en una arquitectura basada en microservicios. 

\subsubsection{Arquitectura basada en microservicios}
\label{subsec:diseno:ott:arquitectura_microservicios}

La arquitectura basada en microservicios es un enfoque para el diseño de aplicaciones que consiste en
un conjunto de pequeños servicios, los cuales se ejecutan en su propio proceso y se comunican con
mecanismos ligeros (normalmente una API de recursos HTTP como es el caso de este proyecto) \cite{Microservices}.

Cada microservicios está especializado en una tarea concreta y trabaja de forma independiente. De esta manera
tendremos las funcionalidades de la plataforma OTT distribuidas en diferentes microservicios, aislando las funcionalidades
y permitiendo que cada uno de ellos pueda ser desarrollado, desplegado y escalado de forma independiente. Esto facilita
la evolución de la plataforma, ya que se puede mejorar cada microservicio con la confianza de que si se hace correctamente
no afectará al resto de la plataforma. Lo mismo ocurre con la tolerancia a fallos, ya que si la arquitectura está bien
diseñada, un fallo en un microservicio no debería afectar al resto de la plataforma, o debería hacerlo lo menos posible.
Además, cada microservicio puede ser reutilizado en diferentes proyectos, lo que facilita la creación de nuevas aplicaciones
y la integración con otros sistemas.

En el caso de la plataforma OTT, se han utilizado los siguientes microservicios creados por la empresa:

\begin{itemize}
    \item \textbf{IDEN - Identificación de usuarios:} encargado de la gestión de los usuarios de la plataforma, incluyendo el registro,
    autenticación y autorización de los mismos, así como la gestión de perfiles, intereses, preferencias y historial. 
    \item \textbf{Directus - CMS:} encargado de la gestión y almacenamiento de los contenidos de la plataforma, de los
    metadatos de los contenidos y de los ficheros multimedia.
    \item \textbf{CAS - Servicio de acceso condicional:} encargado de la gestión de los accesos condicionales a los contenidos
    protegidos de la plataforma, incluyendo la gestión de licencias, DRM, cifrado y protección de contenidos.
    \item \textbf{Orquestador:} Su objetivo es realizar las comprobaciones y procesos que dependan de más de uno de los microservicios de la aplicación
    \item \textbf{Importer:} Microservico utilizada en los casos en los que el cliente posee una Base de Datos con todos los contenidos
    e información de la plataforma OTT y se necesita importar a nuestra BD interna que es la que alimenta a la plataforma.
    \item \textbf{Player:} Utilizado en otras aplicaciones para la obtención del un HTML que contiene el reproductor de video. Esto
    no es posible por el momento de utilizar en la aplicación OTT multiplataforma porque para cada SO es necesario reproductores 
    distintos y el backend todavía no tiene soporte. Sin embargo, a raíz del desarrollo de esta aplicación se ha creado un endpoint
    que devuelve la URL de un video concreto para poder reproducirlo. Esta URL en un principio era construida por la aplicación 
    OTT, sin embargo, para poder aliviar a la plataforma de tareas de este estilo se ha modificado el microservicio para que sea él quien
    construya la URL. Se va a trabajar ahora en adaptación del microservicio para que pueda devolver un HTML con el reproductor necesario
    para cada SO.
\end{itemize}


Estos microservicios son utilizados a través de una API REST, que permite la comunicación entre los diferentes componentes de la
plataforma. Cada microservicio se comunica con los demás a través de esta API, enviando y recibiendo datos en formato JSON.
De esta manera, el frontend de la aplicación está en continua comunicación con el backend, solicitando y enviando datos a través de
las distintas rutas de la API. Cada microservicio ofrece un catalogo de peticiones que se pueden realizar, y el frontend las utiliza
cuando el usuario interactúa con la aplicación. 

\subsubsection{Objetivos de la arquitectura}
\label{subsec:diseno:ott:arquitectura_objetivos}

La arquitectura utilizada en la aplicación esta creada con el objetivo de crear una aplicación escalable, flexible y fácil de mantener.
Uno de los pilares fundamentales son los microservicios ya comentados, que permiten aislar las funcionalidades de la aplicación y
desarrollarlas de forma independiente. Sin embargo, los microservicios por si solos no son suficientes para garantizar que la 
aplicación sea escalable y flexible, sino que el uso que se haga de ellos y la forma en la que se comuniquen entre ellos también
es importante. Por ello, se han seguido una serie de buenas prácticas y patrones de diseño que garantizan que la aplicación sea
robusta y fácil de mantener. Algunos de los objetivos de la arquitectura son:

\begin{itemize}
    \item \textbf{Escalabilidad:} Durante la etapa de diseño, la escalabilidad fue un objetivo clave. 
    Se diseñó la arquitectura del sistema para permitir un crecimiento tanto en la capacidad de usuarios 
    como en la adición de nuevas funcionalidades. Esto se logró mediante el diseño modular y el desacoplamiento
     de componentes, lo que permite que cada parte de la aplicación funcione de manera independiente y pueda 
     escalar horizontalmente (añadiendo más instancias) y verticalmente (optimizando los recursos) según sea necesario.

    El diseño modular fue esencial para manejar la naturaleza multicliente de la plataforma, permitiendo que 
    cada cliente personalice la aplicación sin afectar a los demás. Se implementaron patrones de diseño como 
    microservicios y técnicas de gestión eficiente de la comunicación entre servicios, asegurando que el sistema 
    mantenga su rendimiento y estabilidad a medida que crece.

    \item \textbf{Flexibilidad:} El diseño de la plataforma se centró en crear un sistema flexible que pudiera 
    adaptarse a las necesidades específicas de cada cliente. Esto incluyó la capacidad de personalizar la interfaz 
    de usuario y ajustar las funcionalidades sin modificar el código base. La flexibilidad se logró mediante la 
    implementación de una arquitectura basada en componentes, lo que permite que los módulos sean activados o desactivados 
    según los requisitos del cliente.

    Además, se diseñó la plataforma para ser interoperable con otros sistemas y servicios, facilitando la integración con 
    APIs de terceros y garantizando que la plataforma pueda operar en diversos entornos tecnológicos.

    \item \textbf{Interoperabilidad:} Desde la etapa de diseño, se priorizó la interoperabilidad del sistema, asegurando 
    que la plataforma pueda integrarse y comunicarse eficazmente con otros sistemas y servicios externos. El diseño se 
    centró en utilizar estándares de la industria y tecnologías que permitan una integración sin problemas, asegurando 
    que la plataforma sea compatible con una amplia gama de servicios, desde APIs hasta sistemas de gestión de contenido y análisis de datos.

    La capacidad de manejar diferentes protocolos de comunicación y formatos de datos fue integrada desde el inicio del 
    diseño, permitiendo que la plataforma se adapte fácilmente a nuevas integraciones y expansiones futuras.

    \item \textbf{Adaptación Multiplataforma:} Durante la fase de diseño, se puso un gran énfasis en asegurar que la 
    plataforma OTT fuera capaz de adaptarse a una amplia variedad de dispositivos y sistemas operativos, incluyendo navegadores 
    web, dispositivos móviles y smart TVs. El diseño responsivo y el uso de frameworks multiplataforma permitieron reutilizar 
    gran parte del código base, optimizando el desarrollo y garantizando una experiencia de usuario coherente.

    El diseño también incluyó configuraciones específicas para cada plataforma, permitiendo que la aplicación detecte y se 
    ajuste automáticamente al entorno en el que se ejecuta, lo que garantiza un rendimiento óptimo y la máxima funcionalidad 
    en todos los dispositivos.

    \item \textbf{Mantenibilidad:} La mantenibilidad del sistema fue un objetivo central durante la etapa de diseño. Se 
    diseñó la arquitectura modularmente, permitiendo que los componentes sean actualizados o reemplazados sin afectar el 
    sistema completo. Además, se implementaron prácticas de diseño que facilitan el mantenimiento continuo, como la documentación 
    detallada y la automatización de pruebas y despliegues.

    La elección de herramientas y tecnologías que soporten la integración continua (CI/CD) fue parte integral del diseño, 
    asegurando que las actualizaciones y mejoras puedan ser implementadas rápidamente, con un mínimo de interrupciones.

    \item \textbf{Rendimiento y Optimización:} El rendimiento óptimo fue un objetivo fundamental desde la etapa de diseño. Se 
    diseñaron estrategias para manejar grandes volúmenes de tráfico y datos sin degradar la experiencia del usuario. Esto incluyó 
    la optimización de la comunicación entre componentes, el uso de caché, y la distribución de la carga de trabajo a través de 
    múltiples instancias y CDNs.

    La capacidad de monitorear y ajustar el rendimiento en tiempo real también se integró en el diseño, asegurando que la 
    plataforma pueda mantener un alto rendimiento bajo condiciones de alta demanda.

    \item \textbf{Experiencia de Usuario (UX):} El diseño de la plataforma se centró en ofrecer una experiencia de 
    usuario excepcional. Esto implicó la creación de una interfaz intuitiva, accesible y coherente en todos los 
    dispositivos soportados. El diseño se basó en principios centrados en el usuario, con un enfoque en la personalización y la facilidad de uso.

    Durante el diseño, se realizaron múltiples iteraciones de pruebas de usabilidad para garantizar que la interfaz 
    no solo fuera funcional, sino también agradable y eficiente para los usuarios finales. El diseño también consideró la 
    capacidad de personalizar la experiencia del usuario, ofreciendo recomendaciones y configuraciones ajustadas a las preferencias individuales.
\end{itemize}
