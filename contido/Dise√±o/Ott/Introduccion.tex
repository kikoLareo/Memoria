\subsection{Introducción}
\label{sec:diseno:ott:introduccion}

En este capítulo se describirá el diseño de la aplicación \textit{ott}. Se comenzará con una 
descripción de la arquitectura del sistema, seguido de la interfaz de usuario y los diagramas UML 
utilizados en el desarrollo de la aplicación.

El proceso de diseño de la aplicación \textit{ott} ha sido un tanto peculiar, ya que en un principio
se trabajo sobre diseños de interfaz ya realizados en otras plataformas para otros clientes, buscando crear
una base sobre la que trabajar para desarrollar todas las funcionalidades, conociendo de antemano
que no sería el diseño final, y posteriormente, una vez la aplicación comenzó a estar en una fase más avanzada 
con gran cantidad de las funcionalidades ya implementadas, se comenzaron las reuniones con los distintos 
clientes en las que además de recoger sus impresiones y sugerencias y estudiar nuevas funcionalidades
a implementar, se comenzó a detallar a su gusto la interfaz y la experiencia de usuario de la aplicación.

A lo largo del desarrollo de la aplicación, se han modificado y añadido nuevas funcionalidades. 
Gracias a la flexibilidad de la metodología ágil, se ha podido ir adaptando la aplicación a estas
nuevas funcionalidades, a las necesidades de los clientes y a las nuevas tendencias del mercado, de tal 
manera que cada vez que se terminaba una funcionalidad o una iteración, se realizaban reuniones internas o 
con los clientes para validar y planificar las siguientes etapas. Una vez determinado el objetivo de la 
nueva iteración, comenzaba el ciclo de análisis, diseño, implementación y pruebas. Esto permite una
adaptación continua y una mayor flexibilidad en el desarrollo del proyecto. 


