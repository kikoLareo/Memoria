\subsection{Introducción al diseño de la aplicación \textit{ott}}
\label{subsec:diseno:ott:introduccion}

En este capítulo se describirá el diseño de la aplicación \textit{OTT}. Se comenzará con una 
descripción de la arquitectura del sistema, seguido de la interfaz de usuario y los diagramas UML 
utilizados en el desarrollo de la aplicación.

El proceso de diseño de la aplicación \textit{OTT} ha sido particular, ya que en un principio
se trabajó sobre diseños de interfaz ya utilizados en otras plataformas para otros clientes. 
El objetivo inicial era crear una base sólida sobre la que desarrollar las funcionalidades, 
sabiendo de antemano que este no sería el diseño final. Posteriormente, una vez que la aplicación 
estaba en una fase avanzada con una gran cantidad de funcionalidades implementadas, comenzaron las 
reuniones con los distintos clientes. En estas reuniones, además de recoger sus impresiones y 
sugerencias, se estudiaron nuevas funcionalidades a implementar, y se ajustaron tanto la interfaz 
como la experiencia de usuario para adaptarlas a las preferencias de cada cliente.

Durante el desarrollo de la aplicación, se han modificado y añadido funcionalidades conforme a las 
necesidades que fueron surgiendo. Gracias a la flexibilidad de la metodología ágil, fue posible 
adaptar la aplicación a estas nuevas funcionalidades, así como a las demandas de los clientes y 
las nuevas tendencias del mercado. Cada vez que se completaba una funcionalidad o una iteración, 
se llevaban a cabo reuniones internas o con los clientes para validar el trabajo y planificar 
las siguientes etapas. Una vez definido el objetivo de la nueva iteración, se iniciaba el ciclo de 
análisis, diseño, implementación y pruebas, lo que permitía una adaptación continua y mayor 
flexibilidad en el desarrollo del proyecto.
