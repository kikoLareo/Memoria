\subsection{Diseño de la arquitectura del sistema}
\label{subsec:diseno:ott:arquitectura}

Como ya se ha mencionado en la sección \ref{subsec:fundamentos_teoricos_arquitectura}, 
la arquitectura de esta aplicación está diseñada para integrarse en la arquitectura general 
de la empresa y se basa en una arquitectura de microservicios.

\subsubsection{Arquitectura basada en microservicios}
\label{subsec:diseno:ott:arquitectura_microservicios}

La arquitectura basada en microservicios es un enfoque de diseño que organiza una aplicación en un conjunto 
de pequeños servicios independientes, cada uno especializado en una funcionalidad específica y ejecutado en 
su propio proceso. Estos servicios se comunican a través de mecanismos ligeros como APIs REST \cite{Microservices}.

Cada microservicio está desacoplado del resto, lo que permite que se desarrollen, desplieguen y escalen de 
manera independiente. Este enfoque facilita la evolución de la plataforma, ya que cada microservicio puede ser 
mejorado sin afectar al resto de los componentes. Además, la arquitectura proporciona una mayor tolerancia a fallos, 
ya que un error en un microservicio no debería afectar a toda la plataforma. Este diseño modular también permite 
la reutilización de microservicios en otros proyectos, simplificando la creación de nuevas aplicaciones y su 
integración con otros sistemas.

En el caso de la plataforma OTT, se han utilizado los siguientes microservicios creados por la empresa:

\begin{itemize}
    \item \textbf{IDEN - Identificación de usuarios:} encargado de la gestión de los usuarios de la plataforma, 
    incluyendo el registro, autenticación y autorización, así como la gestión de perfiles, intereses y preferencias.
    
    \item \textbf{Directus - CMS:} responsable de la gestión y almacenamiento de contenidos, metadatos y archivos multimedia.

    \item \textbf{CAS - Servicio de acceso condicional:} encargado de gestionar accesos condicionales, incluyendo licencias, DRM, cifrado y protección de contenidos.

    \item \textbf{Orquestador:} encargado de realizar las comprobaciones y procesos que requieren la interacción de múltiples microservicios.

    \item \textbf{Importer:} utilizado cuando el cliente necesita importar contenido desde su base de datos externa a la plataforma interna de la OTT.

    \item \textbf{Player:} encargado de proporcionar la URL de reproducción de vídeo. Aunque no se utiliza el reproductor nativo debido a las limitaciones del backend, este microservicio genera una URL para reproducir el contenido en cada SO.
\end{itemize}

Estos microservicios se comunican entre sí a través de una API REST, enviando y recibiendo datos en formato JSON. 
El frontend interactúa continuamente con el backend mediante esta API, solicitando y enviando datos a las rutas 
definidas para cada microservicio. Este modelo de comunicación permite una integración fluida entre los distintos 
componentes del sistema.


\subsubsection{Objetivos de la arquitectura}
\label{subsec:diseno:ott:arquitectura_objetivos}

La arquitectura utilizada en la aplicación está diseñada para ser escalable, flexible y fácil de mantener. 
Uno de los pilares fundamentales es el uso de microservicios, que permite aislar las funcionalidades de la aplicación 
y desarrollar cada una de ellas de forma independiente. No obstante, la correcta utilización de los microservicios 
y la forma en que se comunican entre ellos es clave para garantizar que la aplicación mantenga estos principios. 
A continuación, se detallan los principales objetivos de la arquitectura:

\begin{itemize}
    \item \textbf{Escalabilidad:} Desde el diseño, la escalabilidad fue un objetivo prioritario. 
    La arquitectura permite un crecimiento tanto en la capacidad de usuarios como en la adición de nuevas funcionalidades, 
    mediante el uso de un diseño modular y el desacoplamiento de componentes. Esto facilita que cada parte del sistema 
    pueda escalar horizontalmente (añadiendo más instancias) o verticalmente (optimizando recursos) según sea necesario.

    \item \textbf{Flexibilidad:} La plataforma está diseñada para adaptarse a las necesidades específicas de cada cliente, 
    permitiendo personalizar la interfaz y ajustar funcionalidades sin modificar el código base. La flexibilidad se logra 
    mediante una arquitectura basada en componentes, que permite activar o desactivar módulos según los requisitos del cliente. 
    Además, se garantiza la interoperabilidad con otros sistemas y servicios, facilitando la integración con APIs de terceros.

    \item \textbf{Interoperabilidad:} Se priorizó la capacidad del sistema para integrarse eficazmente con otros sistemas 
    y servicios externos, utilizando estándares de la industria que aseguren una integración sin problemas. Esto permite 
    que la plataforma sea compatible con una amplia variedad de servicios, desde APIs hasta sistemas de gestión de contenido 
    y análisis de datos, facilitando su futura expansión.

    \item \textbf{Adaptación Multiplataforma:} Se diseñó la arquitectura para asegurar que la plataforma OTT pueda adaptarse 
    a una gran variedad de dispositivos y sistemas operativos, como navegadores web, dispositivos móviles y smart TVs. 
    El diseño responsivo y la reutilización de gran parte del código base aseguran una experiencia de usuario coherente 
    en todos los dispositivos.

    \item \textbf{Mantenibilidad:} La arquitectura modular facilita que los componentes se actualicen o reemplacen sin 
    afectar al resto del sistema. Las mejores prácticas en diseño y la automatización de pruebas y despliegues aseguran 
    que el sistema sea fácil de mantener y actualizar a lo largo del tiempo.

    \item \textbf{Rendimiento y Optimización:} La plataforma está diseñada para manejar grandes volúmenes de tráfico y datos 
    sin degradar la experiencia del usuario. Se han implementado estrategias como la optimización de la comunicación entre 
    componentes, el uso de caché y la distribución de la carga de trabajo mediante CDNs para garantizar un rendimiento óptimo.

    \item \textbf{Experiencia de Usuario (UX):} El diseño de la plataforma se centra en ofrecer una experiencia de usuario 
    excepcional, con una interfaz intuitiva, accesible y coherente en todos los dispositivos. Se realizaron iteraciones 
    de pruebas de usabilidad para garantizar una interfaz funcional y agradable, que pueda personalizarse según las preferencias 
    individuales de los usuarios.
\end{itemize}
