\subsection{Estudio y diseño de la aplicación de análisis de datos}
\label{sec:diseno-estudio}

\subsubsection{Diseño general}
\label{sec:diseno-general}

El diseño de la aplicación de análisis de datos se ha realizado siguiendo una metodología de diseño centrada en el usuario. 
Para ello, se ha realizado un análisis de las necesidades de los usuarios, se han definido los requisitos de la aplicación 
y se ha diseñado la interfaz de usuario.

Dentro de las necesidades de los usuarios y de los requisitos de la aplicación se destaca la necesidad de que la visualización
de los datos sea clara y sencilla y que para utilizar la aplicación no sea necesario tener conocimientos avanzados de análisis de datos.

En cuanto al diseño de la interfaz de usuario, se ha optado por un enfoque sencillo y minimalista, utilizando colores suaves y 
una tipografía clara y legible. La interfaz ha sido diseñada para ser intuitiva y fácil de usar, permitiendo al usuario acceder 
rápidamente a todas las funcionalidades de la aplicación. El diseño se centra en la visualización de gráficos y datos, presentando 
la información de manera ordenada y consistente, para que el usuario pueda comprenderla rápidamente. Todas las gráficas y tablas 
comparten un estilo uniforme y limpio, evitando elementos recargados que puedan distraer la atención.

De forma similar a la OTT, el diseño de la aplicacion de análisis de datos se ha realizado siguiendo un enfoque modular,
dividiendo la aplicación en diferentes componentes independientes que se comunican entre sí. Esto permite que la aplicación sea
más fácil de mantener y extender, ya que cada componente puede ser modificado o reemplazado sin afectar al resto de los componenetes 
ni a la interfaz. Además, el diseño modular facilita la reutilización de código y la integración de nuevas funcionalidades.

\subsubsection{Estudio de las agrupaciones de datos}
\label{sec:diseno-agrupaciones}

Una vez definidos los requisitos de la aplicación y el diseño general, era importante estudiar que agrupaciones de datos 
se podian lograr con las funcionalidades que ofrece Matomo y que opciones de visualización se podían implementar. 

Lo primero fue detectar que tipos de gráficos eran necesarios soportar y diseñar. Tras un análisis de la entrega de los datos
que hace Matomo se decidió comenzar con el soporte para gráficos lineales, de barras, circulares y tablas. Así, en función 
de la utilidad, periodo y enfoque de los datos, se podrán seleccionar diferentes tipos de gráficos para visualizar la información.

\paragraph{Páginas y secciones}
Lo siguiente era estudiar que páginas o apartados de la aplicación de análisis de datos se podían implementar. La primera página en la que 
se pensó fue la página de inicio o dashboard, en la que se mostrarán los gráficos que presenten una información más general y resumida
de los datos. 

Otra página en la que se pensó durante el análisis fue una página que permitiera al usuario comparar gráficas provenientes de diferentes
módulos. La idea era que el cliente tuviera la oportunidad de comparar varias gráficas de diferentes módulos en una misma página para 
poder analizar la información en busca de patrones, tendencias o correlaciones entre los datos.

El resto de páginas son dedicadas a agrupaciones de datos que tienen relación entre sí. La creación de estas páginas se enfocó en 
que fuera más dinámica y que permitiera la creación fácil y rápida de nuevas secciones. 

