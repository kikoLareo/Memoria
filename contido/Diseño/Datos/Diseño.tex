\subsection{Estudio y diseño de la aplicación de análisis de datos}
\label{sec:diseno-estudio}

El diseño de la aplicación de análisis de datos se ha realizado siguiendo una metodología de diseño centrada en el usuario. 
Para ello, se ha realizado un análisis de las necesidades de los usuarios, se han definido los requisitos de la aplicación 
y se ha diseñado la interfaz de usuario.

Dentro de las necesidades de los usuarios y de los requisitos de la aplicación se destaca la necesidad de que la visualización
de los datos sea clara y sencilla y que para utilizar la aplicación no sea necesario tener conocimientos avanzados de análisis de datos.

En cuanto al diseño de la interfaz de usuario, se ha optado por un enfoque sencillo y minimalista, utilizando colores suaves y 
una tipografía clara y legible. La interfaz ha sido diseñada para ser intuitiva y fácil de usar, permitiendo al usuario acceder 
rápidamente a todas las funcionalidades de la aplicación. El diseño se centra en la visualización de gráficos y datos, presentando 
la información de manera ordenada y consistente, para que el usuario pueda comprenderla rápidamente. Todas las gráficas y tablas 
comparten un estilo uniforme y limpio, evitando elementos recargados que puedan distraer la atención.

De forma similar a la OTT, el diseño de la aplicacion de análisis de datos se ha realizado siguiendo un enfoque modular,
dividiendo la aplicación en diferentes componentes independientes que se comunican entre sí. Esto permite que la aplicación sea
más fácil de mantener y extender, ya que cada componente puede ser modificado o reemplazado sin afectar al resto de los componenetes 
ni a la interfaz. Además, el diseño modular facilita la reutilización de código y la integración de nuevas funcionalidades.

\subsubsection{Estudio de las agrupaciones de datos}
\label{sec:diseno-agrupaciones}

Una vez definidos los requisitos de la aplicación y el diseño general, era importante estudiar que agrupaciones de datos 
se podian lograr con las funcionalidades que ofrece Matomo y que opciones de visualización se podían implementar. 

Lo primero fue detectar que tipos de gráficos eran necesarios soportar y diseñar. Tras un análisis de la entrega de los datos
que hace Matomo se decidió comenzar con el soporte para gráficos lineales, de barras, circulares y tablas. Así, en función 
de la utilidad, periodo y enfoque de los datos, se podrán seleccionar diferentes tipos de gráficos para visualizar la información.

\paragraph{Páginas y secciones}
Lo siguiente era estudiar que páginas o apartados de la aplicación de análisis de datos se podían implementar. La primera página en la que 
se pensó fue la página de inicio o dashboard, en la que se mostrarán los gráficos que presenten una información más general y resumida
de los datos. 

Otra página en la que se pensó durante el análisis fue una página que permitiera al usuario comparar gráficas provenientes de diferentes
módulos. La idea era que el cliente tuviera la oportunidad de comparar varias gráficas de diferentes módulos en una misma página para 
poder analizar la información en busca de patrones, tendencias o correlaciones entre los datos.

El resto de páginas son dedicadas a agrupaciones de datos que tienen relación entre sí. La creación de estas páginas se enfocó en 
que fuera más dinámica y que permitiera la creación fácil y rápida de nuevas secciones. Estas páginas están enfocadas a mostrar 
datos de los visitantes de la web desde diferentes puntos de vista. Para una OTT de estas características, pensada para ser 
utilizada en varias plataformas y dispositivos, una de las secciones interesantes es la de dispositivos, que muestra información
sobre los dispositivos desde los que se accede a la web, mostrando sistemas operativos, marcas, modelos, etc. Otro punto interesante 
obviamente son las reproducciones, aquí se buscará mostrar información sobre la visualización de los contenidos, como el número de 
reproducciones, el tiempo de visualización, videos más vistos, etc.

Teniendo claro que enfoques y datos queriamos mostrar, se comenzó con el diseño de las páginas y secciones y la búsqueda de las 
funcionalidades de la API de Matomo que nos permitieran obtener los datos necesarios para mostrar la información deseada.
Tras estudiar estas funcionalidades, y teniendo en cuenta cuales estaban implementadas y optimizadas en todos los códigos de
las disitintas aplicaciones de la empresa, se crearon las pantallas de reproducciones, dispositivos y videos y visitas, además de
la de inicio y comparador.

\begin{itemize}
    \item \textbf{Inicio:} En esta página se mostrarán los gráficos más generales y resumidos de los datos, como el número de visitas, 
    reproducciones, dispositivos, etc en los últimos instántes. La idea es que el usuario pueda obtener una visión general de los datos en un solo vistazo.
    \item \textbf{Reproducciones:} En esta página se mostrarán los gráficos relacionados con las reproducciones de los videos, como el 
    número de reproducciones, el tiempo medio de visualización, visitantes únicos, etc. La idea es que el usuario pueda analizar la 
    información relacionada con las reproducciones de los videos y obtener insights sobre el comportamiento de los usuarios.
    \item \textbf{Dispositivos:} En esta página se mostrarán los gráficos relacionados con los dispositivos desde los que se accede a la web, 
    como los sistemas operativos, las marcas, los modelos y tipos. La idea es que el usuario pueda analizar la información relacionada con los 
    dispositivos y obtener insights sobre las preferencias de los usuarios.
    \item \textbf{Videos:} Esta página será una lista de todos los contenidos de la plataforma con información para cada uno de ellos. Entre 
    esta información el usuario puede ver el número de reproducciones, el tiempo medio de visualización, tasa de finalización, etc. Además, 
    la tabla permite ordenar los datos en función de los diferentes campos. 
    \item \textbf{Visitas:} Esta sección estará compuesta de varias páginas que muestran información sobre las visitas a la web. Las páginas 
    existentes por el momento son: resumen, tiempo, frecuencia de visitas e interés de los visitantes.
    \item \textbf{Comparador:} En esta página se mostrarán los gráficos de diferentes módulos para que el usuario pueda compararlos y analizar 
    la información en busca de patrones, tendencias o correlaciones entre los datos. La idea es que el usuario pueda obtener insights sobre la 
    relación entre los diferentes módulos y tomar decisiones informadas.
\end{itemize}


\subsubsection{Funcionalidades de la aplicación}
\label{sec:diseno-funcionalidades}

El enfoque de la aplicación de análisis de datos es proporcionar al usuario una herramienta sencilla y eficaz para analizar los datos
de la OTT. Para ello, además de la visualización de gráficos y tablas, otro de los requisitos de la aplicación es que los clientes
tengan la posibilidad de obtener una explicación de los datos y de las gráficas que se muestran. Para esto los usuarios dispondrán de 
una opción que les permita obtener una breve explicación de lo que se está mostrando en la gráfica y otra opción gracias a la cual 
al ser seleccionada el usuario obtendrá un análisis más detallado de los datos a través de la API de OpenAi. Esta funcionalidad utilizará
un contexto general de la aplicación de la cual se estan visualizando los datos y la información de la gráfica para obtener un análisis
más detallado y preciso. Este contexto se ira mejorando y detallando con el tiempo y la utilización de la plataforma y será almacenado
a través de la API de MongoDB praa poder ser suministrado a la API de OpenAi cuando sea necesario. 





