\chapter{Conclusiones}
\label{chap:conclusiones}

El desarrollo de la plataforma OTT ha supuesto un desafío técnico y organizativo significativo, dada la complejidad del 
proyecto y la necesidad de adaptarse a múltiples plataformas y clientes. A lo largo del proyecto, se han logrado 
importantes avances en términos de escalabilidad, flexibilidad y rendimiento, lo que ha permitido crear una aplicación 
robusta y adaptable, capaz de satisfacer las necesidades de diferentes usuarios y dispositivos.

\section{Logros Alcanzados}
\label{sec:conclusiones:logros}

Uno de los principales logros del proyecto ha sido la integración y adaptación a una arquitectura basada en microservicios, 
la cual ha demostrado ser altamente escalable y flexible. Aunque no implementé directamente los microservicios, el 
desafío consistió en integrar de manera eficiente estos componentes existentes en la aplicación y asegurar su correcta 
interacción. Esto incluyó la adaptación de la aplicación al uso de estos microservicios, permitiendo que cada uno de 
ellos funcione de manera independiente, lo que facilita el mantenimiento del sistema y la adición de nuevas funcionalidades.
Además, se contribuyó a la mejora de algunos microservicios específicos, optimizando su rendimiento e integración. 
A través de este proceso, se logró una integración efectiva con APIs y servicios externos, garantizando la interoperabilidad 
del sistema en diferentes entornos tecnológicos.

El enfoque en la experiencia de usuario (UX) ha dado lugar a una interfaz intuitiva y coherente en todas las plataformas 
soportadas. Se ha logrado un diseño minimalista que facilita la navegación y comprensión de los datos por parte del usuario, 
lo que se ha reflejado en la fluidez de la interacción y la satisfacción de los usuarios finales.

Otro logro clave ha sido el desarrollo de un plan de pruebas integral que incluyó pruebas unitarias, de integración y de 
sistema. Este enfoque permitió identificar y corregir errores de manera temprana, asegurando un alto nivel de calidad 
en el producto final. Las pruebas realizadas en dispositivos reales, en lugar de emuladores, han sido fundamentales para 
garantizar que la aplicación funcione correctamente en las condiciones de uso previstas.

\section{Desafíos y Soluciones}
\label{sec:conclusiones:desafios}

El proyecto no estuvo exento de desafíos. Uno de los mayores retos fue garantizar que la aplicación funcionara de manera 
óptima en una amplia gama de dispositivos y sistemas operativos. La diferencia en las capacidades de procesamiento y las 
peculiaridades de cada plataforma obligaron a realizar ajustes específicos en el código y a adoptar enfoques de diseño 
multiplataforma más sofisticados.

\section{Aprendizajes}
\label{sec:conclusiones:mejoras}

A lo largo del desarrollo de este proyecto, se han obtenido valiosas lecciones que pueden aplicarse en futuros desarrollos. 
Algunas de las lecciones más importantes tienen que ver con la recogida de información sobre problemas, tecnologías y
soluciones, la importancia de la colaboración y la comunicación en equipos multidisciplinarios, y la necesidad de adaptarse
a los cambios y desafíos que surgen durante el desarrollo de un proyecto. Otro aprendizaje ha sigo la importancia de la
comunicación con los clientes y la retroalimentación continua, lo que ha permitido ajustar y mejorar la aplicación en función
de las necesidades y preferencias de los usuarios.


\section{Conclusión Final}
\label{sec:conclusiones:final}

En resumen, el desarrollo de la plataforma OTT ha sido un proyecto exitoso que ha cumplido con los objetivos propuestos, 
proporcionando una aplicación escalable, flexible y de alto rendimiento. Aunque se enfrentaron desafíos significativos, 
las soluciones implementadas han permitido superar estos obstáculos y entregar un producto que no solo satisface las 
necesidades actuales de los clientes, sino que también está preparado para evolucionar y adaptarse a futuros requerimientos. 
Este proyecto no solo representa un avance técnico importante, sino que también establece una base sólida para desarrollos 
futuros en el ámbito de las plataformas de distribución de contenido.
