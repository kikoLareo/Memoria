\chapter{Pruebas}
\label{chap:pruebas}

\section{Introducción}
\label{sec:introduccion_pruebas}

En este capítulo se describen las pruebas realizadas durante el desarrollo del proyecto. Estas pruebas tienen como
objetivo verificar el correcto funcionamiento de la aplicación y comprobar que cumple con los requisitos establecidos
en la fase de análisis. 

\subsection{Plan de pruebas unitarias, integración y sistema}
\label{subsec:plan_pruebas}

Desde el inicio del proyecto, se planificó un enfoque de pruebas exhaustivo que abarcara pruebas unitarias, 
de integración y de sistema, adaptado a la naturaleza de la aplicación y sus requisitos específicos. Dado naturaleza
de la aplicación, las pruebas se realizaron principalmente de manera manual, llevadas a 
cabo por diferentes miembros del equipo de desarrollo y calidad.

\subsubsection{Pruebas Unitarias}
Las pruebas unitarias se enfocaron en validar componentes individuales de la aplicación en sus fases iniciales de desarrollo. 
Estas pruebas se realizaron en paralelo al desarrollo de cada funcionalidad, comprobando que cada módulo y microservicio funcionara 
correctamente de forma aislada. El objetivo era identificar y corregir errores en las unidades más pequeñas de código antes de 
integrarlas en el sistema global.

\subsubsection{Pruebas de Integración}
Una vez validadas las unidades individuales, se realizaron pruebas de integración para verificar que los diferentes componentes 
del sistema, incluyendo microservicios y módulos de la aplicación, interactuaran correctamente entre sí. Estas pruebas fueron 
esenciales para garantizar que la comunicación entre los distintos componentes y partes de la aplicación funcionara sin problemas
y que no se produjeran errores de integración. 

\subsubsection{Pruebas de Sistema}
Las pruebas de sistema se llevaron a cabo utilizando la aplicación completa en dispositivos reales. Aunque inicialmente se consideró 
el uso de emuladores, se constató que estos no ofrecían una representación precisa de las capacidades de procesamiento y 
comportamiento de los dispositivos reales. Por ello, se optó por realizar pruebas directamente en los dispositivos finales, 
asegurando que la aplicación funcionara de manera fluida y sin errores en condiciones reales de uso.

El proceso de pruebas de sistema incluyó tanto pruebas funcionales, que verificaron que todas las funcionalidades de la
 aplicación operaban como se esperaba, como pruebas de rendimiento, que aseguraron que la aplicación mantenía un comportamiento 
 óptimo bajo diferentes cargas y en distintos dispositivos.

\subsubsection{Enfoque de Pruebas Continua}
Durante el desarrollo de cada funcionalidad, las pruebas fueron iterativas y continuas. Se realizaban pruebas parciales conforme 
se avanzaba en la implementación, asegurando que cada nueva parte integrada no introducía errores ni afectaba negativamente al 
rendimiento. Al finalizar cada funcionalidad, se llevaban a cabo pruebas más exhaustivas, tanto por el desarrollador como por 
otros miembros del equipo, para garantizar que la aplicación en su conjunto funcionaba correctamente antes de proceder con nuevas adiciones.

Este enfoque de pruebas continuas permitió identificar y resolver problemas de manera temprana, reduciendo el riesgo de errores 
acumulados en las etapas finales del desarrollo. Además, este método aseguró una alta calidad del producto final al abordar las 
posibles fallas en cada fase del desarrollo.

\subsection{Pruebas de aceptación}
\label{subsec:pruebas_aceptacion}

Al estar desarrollando aplicaciones para terceros, las pruebas de la aplicación OTT realizadas por el equipo de desarrollo no son suficientes
para garantizar la calidad del producto final. Por ello, se planificaron pruebas de aceptación que involucraran a los
clientes, con el objetivo de validar que la aplicación cumplía con sus expectativas y requerimientos.

Las pruebas de aceptación se llevaron a cabo en diferentes etapas del desarrollo, permitiendo a los clientes evaluar y
proporcionar retroalimentación sobre la aplicación en su estado actual. Esta retroalimentación fue fundamental para
ajustar y mejorar la aplicación en función de las necesidades y preferencias de los clientes, asegurando que el producto
final cumpliera con sus expectativas. 

Las pruebas de aceptación se realizaron a través de un servidor de pruebas donde estaba alojada la aplicación, permitiendo
a los clientes acceder a la aplicación y probarla en un entorno controlado. Una vez el cliente completaba las pruebas, se
realizaba una reunion telemática para obtener los resultados de las pruebas y el feedback del cliente. Durante estas reuniones
se discutían los resultados y se acordaban los cambios y mejoras necesarios para la aplicación.

\subsection{Pruebas realizadas por los marketplaces}
\label{subsec:pruebas_marketplaces}

Además de las pruebas de aceptación realizadas por los clientes, la aplicación OTT también es sometida a pruebas por parte
de los marketplaces en los que se quiere publicar. Cada marketplace tiene sus propios requisitos y estándares de calidad que la
aplicación debe cumplir para ser aprobada y publicada en su plataforma. Por ello, se realizan pruebas específicas para
cada marketplace, asegurando que la aplicación cumple con los requisitos técnicos y de calidad establecidos por cada uno.

Una vez completadas las pruebas de cada marketplace y corregidas las deficiencias detectadas, se envía la aplicación en el formato
adecuado para cada plataforma, junto con la documentación y la información requerida. A continuación, se espera la aprobación o los 
resultados de las distintas pruebas que realizan los marketplaces, que pueden incluir pruebas de rendimiento, seguridad, usabilidad,
entre otras.

\section{Resultados de las pruebas}
\label{sec:resultados_pruebas}

Para este proyecto ya se han realizado pruebas con versiones del código en varios marketplaces. Uno de ellos fue el de LG. Los 
resultados de su prueba arrojaron un problema con la reproducción de los vídeos en la aplicación en un primer envio (fue corregido de 
inmediato en la siguiente versión ya que no era un problema grave). Al reenviar la aplicación, las funcionalidades pasaron satisfactoriamente
las pruebas de funcionalidades y de rendimiento. Sin embargo, se encontró otro problema importante: la aplicación no se inicia en 
versiones más antiguas de webOs 5.0, esta incluida. Este problema se está intentando solucionar, pero no es un problema grave ya que
a los clientes se ofreció una aplicación compatible y funcional con televisiones con alguno de las últimas 4 versiones de webOs
(webOs 6.0, 22, 23, 24) por lo que será un error que se intentará solucionar pero que no afecta a la publicación de la aplicación.


