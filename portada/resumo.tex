%%%%%%%%%%%%%%%%%%%%%%%%%%%%%%%%%%%%%%%%%%%%%%%%%%%%%%%%%%%%%%%%%%%%%%%%%%%%%%%%

\pagestyle{empty}
\begin{abstract}
  Este trabajo presenta el desarrollo e implementación de una plataforma OTT (Over-The-Top) multiplataforma, 
  diseñada para funcionar en una amplia gama de dispositivos, desde televisores inteligentes hasta ordenadores y 
  dispositivos móviles. La aplicación utiliza una arquitectura basada en microservicios, lo que garantiza escalabilidad, 
  flexibilidad y fácil mantenimiento. El objetivo principal del proyecto fue crear una plataforma robusta y adaptable, 
  capaz de integrarse con diversos sistemas operativos y satisfacer las necesidades de múltiples clientes.

A lo largo del proyecto, se han abordado importantes desafíos técnicos, como la adaptación de la aplicación a diferentes 
capacidades de procesamiento y la optimización del rendimiento en distintos entornos, tales como WebOS, Tizen y Android TV. 
Para ello, se han realizado pruebas exhaustivas, tanto a nivel unitario como de integración y sistema, que han permitido 
validar la funcionalidad y el rendimiento de la aplicación en condiciones reales de uso.

Además de la plataforma OTT, el proyecto incluye una aplicación de análisis de datos, desarrollada con la API de Matomo, 
que permite realizar un seguimiento detallado del comportamiento de los usuarios. Esta herramienta facilita la toma de 
decisiones basada en datos, permitiendo a los clientes obtener informes personalizados sobre el rendimiento de sus contenidos.

Los resultados obtenidos durante el desarrollo de este proyecto demuestran la capacidad de la plataforma para adaptarse a 
las demandas del mercado y proporcionar una experiencia de usuario óptima. Finalmente, se destacan los objetivos futuros, 
entre los que se incluyen la expansión a nuevos dispositivos, la mejora continua del rendimiento y la personalización de 
la interfaz para diferentes clientes.

  \vspace*{25pt}
  \begin{segundoresumo}
    Este traballo presenta o desenvolvemento e implementación dunha plataforma OTT (Over-The-Top) multiplataforma, deseñada 
    para funcionar nunha ampla gama de dispositivos, dende televisores intelixentes ata ordenadores e dispositivos móbiles. 
    A aplicación utiliza unha arquitectura baseada en microservizos, o que garante escalabilidade, flexibilidade e un mantemento 
    sinxelo. O obxectivo principal do proxecto foi crear unha plataforma robusta e adaptable, capaz de integrarse con diversos 
    sistemas operativos e satisfacer as necesidades de múltiples clientes.

    Ao longo do proxecto, abordáronse importantes desafíos técnicos, como a adaptación da aplicación a diferentes capacidades de 
    procesamento e a optimización do rendemento en distintos contornos, tales como WebOS, Tizen e Android TV. Para iso, realizáronse 
    probas exhaustivas, tanto a nivel unitario como de integración e sistema, que permitiron validar a funcionalidade e o rendemento 
    da aplicación en condicións reais de uso.

    Ademais da plataforma OTT, o proxecto inclúe unha aplicación de análise de datos, desenvolvida coa API de Matomo, que permite 
    realizar un seguimento detallado do comportamento dos usuarios. Esta ferramenta facilita a toma de decisións baseada en datos, 
    permitindo aos clientes obter informes personalizados sobre o rendemento dos seus contidos.

    Os resultados obtidos durante o desenvolvemento deste proxecto demostran a capacidade da plataforma para adaptarse ás demandas 
    do mercado e proporcionar unha experiencia de usuario óptima. Finalmente, destácanse os obxectivos futuros, entre os que se 
    inclúen a expansión a novos dispositivos, a mellora continua do rendemento e a personalización da interface para diferentes clientes.
  \end{segundoresumo}
\vspace*{25pt}
\begin{multicols}{2}
\begin{description}
\item [\palabraschaveprincipal:] \mbox{} \\[-20pt]
  \blindlist{itemize}[7] % substitúe este comando por un itemize
                         % que relacione as palabras chave
                         % que mellor identifiquen o teu TFG
                         % no idioma principal da memoria (tipicamente: galego)
\end{description}
\begin{description}
\item [\palabraschavesecundaria:] \mbox{} \\[-20pt]
  \blindlist{itemize}[7] % substitúe este comando por un itemize
                         % que relacione as palabras chave
                         % que mellor identifiquen o teu TFG
                         % no idioma secundario da memoria (tipicamente: inglés)
\end{description}
\end{multicols}

\end{abstract}
\pagestyle{fancy}

%%%%%%%%%%%%%%%%%%%%%%%%%%%%%%%%%%%%%%%%%%%%%%%%%%%%%%%%%%%%%%%%%%%%%%%%%%%%%%%%
